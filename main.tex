%==================================================
% 修士論文_五十嵐康佑
%==================================================
% \documentclass[uplatex,a4paper,11pt]{jsreport}
% \documentclass[uplatex,a4paper,11pt]{jsreport}
\documentclass[dvipdfmx,a4paper,11pt]{jsreport}

% ---- Packages ----
% pdfLaTeX 用の設定は使わない(uplatex では不要)
% \usepackage[utf8]{inputenc}
% \usepackage[T1]{fontenc}
% \usepackage{lmodern}

\usepackage[dvipdfmx]{geometry}
\geometry{left=25mm,right=25mm,top=25mm,bottom=25mm}

% ★ LuaLaTeX 用のパッケージは削除する
% \usepackage{luatexja}
% \usepackage[haranoaji]{luatexja-preset} % 日本語フォント設定

\usepackage{setspace}
%\onehalfspacing

\usepackage{amsmath,amssymb,amsthm}
\usepackage{bm}
\usepackage[dvipdfmx]{graphicx}  % uplatex + dvipdfmx 用
\usepackage{booktabs}
\usepackage{threeparttable}
\usepackage{here}
\usepackage[dvipdfmx]{hyperref}  % 同上
\usepackage{url}     % URL 表示用

% 1. 章タイトルのデザイン変更(改行させず1行にする・余白を詰める)
\usepackage{titlesec}

% 章タイトルのフォーマット定義
% [hang] : ラベル(第1章)とタイトル(序論)を1行に並べる設定
\titleformat{\chapter}[hang]
  {\normalfont\huge\bfseries} % フォント(太字・巨大)
  {第\thechapter 章}         % ラベルの形式
  {1em}                       % ラベルとタイトルの間の空白
  {}

% 章タイトルの周囲の余白調整
% \titlespacing*{コマンド}{左余白}{上余白}{下余白}
\titlespacing*{\chapter}{0pt}{-20pt}{20pt} 
% ※ 上余白をマイナスにすると、ページのかなり上の方から始まります。
%    デフォルトより詰めたい場合は -10pt や 0pt などで調整してください。

% 2. 本文の行間の調整
% 現在 \onehalfspacing(約1.5倍)が入っていますが、これが「広い」原因です。
% 削除するか、以下のように \setstretch で数値を指定して調整してください。

% \onehalfspacing  % ← これを削除またはコメントアウト
\setstretch{1.1}   % 1.0が標準。0.9〜1.1くらいでお好みに調整してください。

% ---------- 定理環境など(必要なら) ----------
\newtheorem{theorem}{定理}[chapter]
\newtheorem{lemma}{補題}[chapter]
\newtheorem{proposition}{命題}[chapter]
\theoremstyle{definition}
\newtheorem{definition}{定義}[chapter]

% ---------- タイトルなど ----------
\title{紹介状を持たない患者に対する選定療養費徴収義務化が外来患者の受診行動に与えた影響}
\author{五十嵐 康佑}
\date{\today}  % 提出日が決まっていれば固定の日付を書く

%==================================================
\begin{document}
%==================================================

% --- 表紙 ---
\begin{titlepage}
  \begin{center}
    \vspace*{2cm}
    {\Large 修士学位論文}\\[1cm]
    {\huge \textbf{紹介状を持たない患者に対する選定療養費徴収義務化が外来患者の受診行動に与えた影響}}\\[0.5cm]
    {\huge \textbf{(The Impact of the Mandatory Charge of Additional Fees for Non-Referral Patients on Outpatients' Behavior)}}\\[3cm]

    {\large 経済学研究科}\\[0.5cm]
    % {\large 2025年度}\\[1cm]

    {\large 経済専攻経済学コース}\\[0.5cm]
    % {\large 2025年度}\\[1cm]

    {\large 学籍番号:29-246004}\\[0.5cm]
    % {\large 2025年度}\\[1cm]

    {\Large 氏名:五十嵐 康佑}\\[0.5cm]
    % {\large 指導教員:XXXX 教授}\\[3cm]

    {\Large 提出日:令和8年1月6日}\\[0.5cm]
    % {\large 指導教員:XXXX 教授}\\[3cm]

  \end{center}
\end{titlepage}

% --- 前付け(目次・概要など) ---
\pagenumbering{roman} % ローマ数字

\tableofcontents      % 目次
% \listoffigures        % 図目次(不要ならコメントアウト)
% \listoftables         % 表目次(不要ならコメントアウト)

% \chapter*{概要}       % 日本語の要約
% \addcontentsline{toc}{chapter}{概要} % 目次に載せる


% \chapter*{Abstract}   % 英語要約
% \addcontentsline{toc}{chapter}{Abstract}
% Here you write the English abstract of your master thesis.

\clearpage
\pagenumbering{arabic} % ここから本文はアラビア数字

%==================================================
% 本文
%==================================================

\chapter{序論}
\label{chap:intro}

\section{背景:情報の非対称性と医療資源配分}
\label{sec:information_asymmetry}
日本の医療制度は国民皆保険制度とフリーアクセスを二大支柱として発展し、世界最高水準の平均寿命と
保健衛生水準を達成してきた。特にフリーアクセス制度下では、患者は居住地や症状の軽重に関わらず、自身の選好に基づいて自由に医療機関
を選択することが可能である。この仕組みは、医療へのアクセス障壁を下げることで早期発見・早期治療に寄与してきた一方で、医療
資源配分の非効率性という深刻な副作用をもたらしており、その例として挙げられるのが、大病院への患者集中問題である。
本来高度急性期医療や専門医療を担うべき特定機能病院や地域医療支援病院(以下、大病院)に対し、軽微な傷病や慢性疾患の安定期にある
患者が、「念のため」「安心だから」という理由で選好して受診する傾向が常態化している。このような受診行動は大病院における長時間待機
や医療従事者の疲弊を招くだけでなく、高度医療を必要とする重症患者へのリソース投入を阻害する要因となる。経済学の古典的論文である
 \cite{Arrow1963} が指摘するように、医療サービス市場には情報の非対称性と不確実性が存在する。患者は自身の症状が軽症か
 重症かを正確に判断する知識を持たず、またどの医療機関が適切な治療を提供できるかについても不完全な情報しか持たない。そのため患者
 は「大病院である」という規模や設備、評判を医療の質を保証するシグナルとして利用し、たとえ軽症であっても効用最大化行動の結果と
 して大病院を選好する。この患者行動は個人の視点では合理的であっても、社会全体で見れば高度医療資源の浪費という負の外部性を
 引き起こしている。この問題に対処するため、諸外国(イギリスや北欧諸国等)では家庭医(General Practitioner: GP)登録制度に
 よるゲートキーピング機能を導入し、専門医への受診にはGPの紹介状を必須とするケースが多い\cite{Okubo2021}。しかし、かかり
 つけ医の事前登録制度を持たない日本において同様の強権的な受診制限を導入することはフリーアクセスの理念と衝突するため困難である。
 そこで日本独自の政策手段として導入されたのが価格メカニズムを通じた誘導、すなわち「紹介状なしで大病院を受診する患者に対する選定
 療養費の徴収義務化」である。2016年から始まったこの制度は、紹介状を持たずに特定の大病院を受診する患者に対して保険診療の一部負担金
 とは別に、全額自己負担となる特別料金(定額負担)の支払いを義務付けるものである。これは、患者が直面する金銭的コストを引き上げるこ
 とで不必要な大病院受診を抑制し、診療所・クリニックへの逆紹介を促すことを目的としており、法的なゲートキーパー制度の代わりに、
 価格インセンティブを用いて患者の行動変容と医療機関の機能分化を達成しようとする試みである。しかし、この制度が政策意図通り
 に機能しているかは自明ではない。前述の通り、医療における情報の非対称性が存在する場合、患者は価格よりも大病院における診療の質の
 高さを優先する可能性がある。実際、\cite{KawaguchiSugahara2015}によれば5,000円から10,000円程度の負担増がなければ軽症患者
 の行動変容は起きにくいと予測されている。日本のような法的なゲートキーパー制度を導入することが難しい社会において、定額の追加費用がどの程度患者の行動変容
 を引き起こすかは、実証的に検証されるべき重要な問いである。
 
 \section{本研究の目的とリサーチクエスチョン}
\label{sec:research_question}
本研究の目的は、紹介状を持たない外来患者に対する選定療養費徴収義務化がどの程度患者の受診行動を変容させたかを定量的に明らか
にすることである。具体的には、2022年10月に実施された制度改正を自然実験と見なして利用する。この改正によって、特定機能病院・一般
病床200床以上の地域医療支援病院に加えて一般病床200床以上の紹介受診重点医療機関が選定療養費の徴収義務対象となる医療機関となり、
同時に最低徴収金額も増額された。本研究ではこの外生的な制度変更を利用して「選定療養費の徴収義務化および
その拡大は、対象病院における初診外来件数を有意に減少させたか? また、その効果は時間の推移によってどの程度異なるか」というリサーチ
クエスチョンに対する回答を出すことを目的とする。

\section{本研究のアプローチ}
\label{sec:approach}
本研究では、全国健康保険協会(協会けんぽ)の2015年度〜2023年度のレセプトデータを使用した病院×月単位のパネルデータを構築し、差の
差分析およびイベントスタディを用いて政策効果を推定する。制度の対象となる病院(処置群)とそうでない病院(対照群)の分類について、レセプ
トデータ上では各病院が特定機能病院・地域医療支援病院・紹介受診重点医療機関に該当するかどうかは判別不可能であるが、本研究では診療
行為記録内の選定療養費算定の有無を用いることで、各病院がどの時点から制度の対象となったかを特定する。これにより、2022年10月の
ショックを受けた病院群(処置群)と、影響を受けなかった病院群(対照群)を構成し、診療科数などの時間を通じて変化する可能性のある病院固有の要素をコント
ロールした上で政策効果を推定する。また、分析のアウトカムとしては制度の影響を最も直接的に受けると考えられる初診
外来件数に焦点を当てる。\cite{Izumida2004}は自己負担率の引き上げが受診確率を抑制する一方で、その後の治療プロセスには影響しない
ことを示している。これにならい、選定療養費が初診における受診抑制に対して機能したかを見ることで、政策の効果を検証する。

\section{本研究の貢献と意義}
\label{sec:approach}
本研究の貢献は、主に以下の3点である。\\
第一に、選定療養費徴収義務化の効果に関するレセプトデータを利用した全国規模での検証である。本制度に関する既存の実証研究は極めて
少ない。\cite{Iba2023}は2016年の制度導入が紹介率を向上させたことを示したが、その分析対象は茨城県のデータに限られていた。
また、\cite{Sugahara2015Rep}はアンケート調査に基づくシミュレーションであり、実際の行動変容を捉えたものではない。本研究は、
全国規模のレセプトデータを用いることで2022年の大規模な対象拡大の効果を検証する初の試みの一つである。\\
第二に、医療需要の価格弾力性に関する新たな知見の提示である。\cite{Manning1987} や\cite{Shigeoka2014}などの
先行研究の多くは、自己負担率の変化が医療需要に与える影響を分析してきた。これに対し、本研究が対象とする選定療養費は、受診一回あたり
に課される定額負担であり、かつ「紹介状がない場合」という条件付きのコストである。このような非線形かつ条件付きの価格設定が患者の
受診行動にどのような影響を与えるかを示した研究は蓄積が少ない。\\
第三に、今後の医療政策への示唆である。かかりつけ医機能の法制化が議論される中、価格メカニズムによる誘導がどの程度有効かを示すこと
は極めて重要である。もし、現在の価格設定(医科初診の場合・最低金額7,000円)で十分な抑制効果が見られないとすれば、さらなる価格の
引き上げや、より強力なゲートキーピング制度の導入が必要であることが示唆される。逆に、過度な受診抑制が発生していれば、アクセス阻害の
懸念が生じる。本研究の結果は、次期診療報酬改定や医療提供体制の見直しに向けたエビデンスの基盤となりうると考える。

\section{主要な結果の要約}
\label{sec:main_results}
分析の結果、以下の知見が得られた。\\
第一に、イベントスタディの結果2022年10月の制度拡大直後から3ヶ月程度は対象病院の初診外来件数, 総外来件数に統計的に有意な減少トレンドが確認された。一部の
医療機関(紹介受診重点医療機関)については新たに認定されてから制度適用までに6か月の経過措置が設けられていたものの、多くの対象病院において
即座に受診抑制効果が現れたことが示唆される。第二に、DID推定の結果選定療養費の義務化は平均して初診外来件数を約 \textbf{12.3}\% 減少
させる効果を持っていたことが明らかになった。この結果は、紹介状を持たない患者の一部が追加費用を忌避して地域の診療所等へ受診先を変更した
(あるいは受診を控えた)ことを示唆しており、制度が一定の効果を発揮したと評価できる。一方で、再診外来件数および総外来件数への影響は限定的(約1.7\%の減少, 約4\%の減少)で
あった。この結果からは、\cite{Izumida2004} が示したように、負担増は初診を抑制するものの、慢性疾患などで通院を続ける
再診患者を含めた病院全体の患者フローを大きく変えるには至っていないということができる。

\section{本論文の構成}
\label{sec:structure}
本論文の構成は以下の通りである。第2章では、日本の医療制度におけるフリーアクセスの課題と、選定療養費制度の変遷について、特にかかり
つけ医機能との関連から詳述する。第3章では、医療需要の価格弾力性および患者の受診選択行動、選定療養費制度の評価に関する先行研究を概観し、本研究の位置付け
を明確にする。第4章では、使用するレセプトデータの詳細と記述統計を示す。第5章では、DIDおよびイベントスタディを用いた実証モデルの
特定化について説明する。第6章で推定結果を示し、第7章でその解釈と政策的含意、および研究の限界について考察を行う。最後に第8章で結論
を述べる。

%--------------------------------------------------

\chapter{制度背景}
\label{chap:institution_background}
本章では、本研究の分析対象である紹介状なし外来患者に対する選定療養費制度の背景と変遷について記述する。まず、日本の医療制度の特徴で
あるフリーアクセスとその課題について整理し、主要国との比較を交えながらなぜ価格メカニズムによる介入が必要とされたのかを論じる。
次に選定療養費制度の導入から2016年の徴収義務化導入、そして2022年の大規模改定に至るまでの経緯を、厚生労働省の資料等に基づき概観
し、本研究が利用する自然実験としての制度的枠組みを記載する。最後に、本制度が患者行動に与える影響に関する経済学的な理論仮説を
提示する。

\section{日本の医療制度とフリーアクセス}
\label{sec:freeaccess}
\subsection{フリーアクセスとその課題}
\label{sec:fa_merit_demerit}
日本の医療制度は、国民皆保険制度とフリーアクセスを二大支柱として発展してきた。フリーアクセスの下では患者は自身の傷病の軽重や居住
地に関わらず、全国どの医療機関であっても自由に選択し受診することが可能である。また、原則として医療機関側が正当な理由なく診療を拒否
することは医師法によって禁じられている。この高いアクセス性は国民の健康維持や疾患の早期発見に大きく寄与し、世界最高水準の平均寿命
と保健衛生水準に貢献してきた。しかし一方で、フリーアクセスは医療資源配分における深刻な非効率性を引き起こすリスクを持つ。非効率性の
例として、医療機関のリソースと患者の病状のミスマッチが挙げられる。本来大学病院や特定機能病院・地域医療支援病院などの
大病院は、高度急性期医療や専門的治療を提供するために高額な医療機器や専門スタッフを集約させた施設である。これら希少なリソースは
重症患者や難治性疾患の治療に優先的に配分されるべきであるが、フリーアクセスの下では本来は一次医療(プライマリ・ケア)で対応
可能な軽症患者が大病院を選好して受診する現象が生じやすい。これは、患者が大病院の持つ安心感・ブランドを選好する一方で自身の受診が
引き起こす混雑やリソースの占有といった社会的費用(負の外部性)を考慮しないために発生する「共有地の悲劇」のような状況と言える。\cite{Tsuchikura2015}によれば、この
結果大病院における長時間待機や勤務医の過重労働による疲弊が常態化し、医療提供体制の持続可能性が危ぶまれている。

\subsection{ゲートキーパー機能の国際比較}
\label{sec:international_gp}
この問題に関連して、日本ではかかりつけ医のゲートキーパーとしての機能が欠如している。\cite{Okubo2021}が指摘するように、イギリスの
NHS制度やフランス、北欧諸国に見られるように多くの先進国では家庭医(General Practitioner: GP)登録制度が採用されている。
たとえばイギリスでは救急の場合を除いて患者はまず登録したGPを受診する義務があり、専門医や大病院へのアクセスはGPが必要と判断して発行
する紹介状を持つ患者に厳格に制限されている。これにより、医療の機能分化が制度的に担保されている。それとは対照的に、日本には制度化
されたかかりつけ医の事前登録システムが存在しない。これは患者にとっての利便性を高める反面、専門知識を持たない患者が自己判断で過剰に
高度な医療機関を選択するリスクを持つ。

\subsection{選定療養費制度の経済学的意義}
\label{sec:economical_mean}
このように、フリーアクセスの維持と医療資源の効率的配分はトレードオフの関係にある。大病院への患者集中を解消して病院の機能分化を推進
するためには患者の受診行動を変容させる必要があるものの、フリーアクセスが制限される予定が当面ない日本においては紹介状を持たない
患者の受診を一律に拒否するような強権的な規制を導入することは、政治的・社会的に困難である。このジレンマに対する解決策として導入され
たのが、選定療養費制度を用いた価格メカニズムによる誘導である。選定療養費とは保険診療と保険外診療の併用(混合診療)を例外的に認める
制度の一つである。紹介状のない外来受診に対する選定療養費制度は、紹介状を持たずに大病院のような高度医療機関を受診する行為に対して
定額の追加負担を課すことで患者に金銭的ディスインセンティブを与えるものである。経済学的に解釈すれば、選定療養費は大病院の混雑が引き
起こす外部性を内部化するためのピグー税のようなものと見なすことができる。日本では現在、強制的なゲートキーピング制度を導入できない
制約の中で価格シグナルを通じて患者を診療所・クリニックへと誘導し、間接的に医療提供体制の機能分化を達成しようとしている。

\section{選定療養費制度の変遷}
\label{sec:sentei_policy}
本節では、紹介状を持たない外来受診に対する選定療養費制度の変遷について概観する。本制度は、当初は医療機関の裁量による任意の徴収で
あったが、段階的な診療報酬改定を経て対象病院の拡大と徴収の義務化および最低徴収金額の引き上げが行われてきた。

\subsection{制度の導入と初期の拡大(2016年以前)}
\label{sec:sentei_policy_early}
2016年度以前においては、一般病床200床以上の病院は地方厚生局への届出を行うことで紹介状を持たない初診患者から任意の金額の選定療養費
を徴収することが認められていた。しかし、\cite{MHLW2015}によればこの段階では徴収はあくまで医療機関の自由裁量であり、地域住民へ
の配慮や患者減少への懸念から実際に徴収を行う病院は限定的であった。また、徴収を行っている場合でもその金額は平均して2,000円台にとど
まり、受診抑制効果は限定的であった。

\subsection{徴収の義務化:2016年改革}
\label{sec:sentei_policy_2016}
2016年4月の診療報酬改定時には、特定機能病院および一般病床500床以上の地域医療支援病院を対象として紹介状なしの外来受診に対する選定
療養費の徴収が義務化された。対象となった医療機関について、「特定機能病院」とは、高度の医療を提供するとともに高度の医療技術の開発
・評価、および医療に関する研修を行う能力を有する病院として厚生労働大臣の承認を受けた施設(主に大学病院等)である。一方「地域医療
支援病院」とは、地域医療の第一線を担うかかりつけ医(診療所・中小病院)を支援する能力を有する病院として都道府県知事の承認を受けた
施設であり、救急医療や共同利用機能の中核を担う。義務化に伴い、最低徴収金額は初診時5,000円(歯科は3,000円)、再診時2,500円
(歯科は1,500円)と設定された。これは、当時のコンジョイント分析等の結果\cite{KawaguchiSugahara2015}を踏まえ、患者の行動変容を促すために
必要な水準として設定されたものである。

\subsection{2022年の制度改革:対象の大幅拡大}
\label{sec:sentei_policy_2022}
本研究が分析の焦点とする2022年10月の制度改革は、以下の3点の大きな特徴を持つ。第一に、義務化対象病院の拡大が挙げられる。従来の特定
機能病院・地域医療支援病院に加え、新たに制度化された紹介受診重点医療機関(一般病床200床以上)が選定療養費徴収義務化の対象となった。
 \cite{MHLW2020}によれば、紹介受診重点医療機関は外来機能報告制度に基づき手術、化学療法、放射線治療など医療資源を重点的に活用
 する外来医療を地域の中で基幹的に担う病院として都道府県が公表したものである。この新たな枠組みの導入により、これまで制度対象外で
 あった200床〜400床規模を含む多くの中規模病院がその機能実績に基づいて新たに選定療養費の徴収義務対象に組み込まれることとなった。
 また、紹介受診重点医療機関の公表時期は2023年3月〜8月頃と都道府県によって異なり、公表から6ヶ月間の経過措置期間が設けられている
 ため実際に徴収義務化が適用された時期には地域差が存在する点に留意が必要である。第二に、定額負担額の引き上げである。初診時の最低
 徴収金額は5,000円から7,000円へ、再診時は2,500円から3,000円へと増額された。これは消費増税等の影響を除けば、患者の実質負担を
 約1.4倍に引き上げる強力な価格介入である。第三に、保険給付範囲からの控除による病院に対する徴収強制力の強化である。これは制度の
 実効性の観点から非常に重要な変更点と言える。改定以前は選定療養費の徴収は義務であるとされつつも、徴収しない場合の罰則規定が弱く
 病院側の裁量が残されていた。しかし、2022年10月の改定以降は義務化対象病院が紹介状なしの患者を診療した(初診)場合、初診料等から一定
 額(200点=2,000円)が保険給付から控除される仕組みとなった。 具体的には、初診料(288点=2,880円)を算定する際、病院が患者から
 選定療養費を徴収しなければ、そこから200点=2,000円が差し引かれ、残りの880円しか保険請求できない。すなわち、病院は本来得られる
 はずだった診療報酬を失うこととなる。これにより、病院側にとって選定療養費の徴収は経営上の損失回避のために必要となった。なお、制度
 の公平性を担保するため救急搬送や公費負担医療の対象者、HIV等の特定の疾病についてはこれらの徴収義務の対象外とされている。

\section{理論的枠組みと仮説}
\label{sec:theory_hypothesis}
本節では上述した制度的背景に基づき、選定療養費の導入および拡大が患者行動に与える影響について経済学的な理論仮説を提示する。

\subsection{情報の非対称性と不確実性}
\label{sec:asymmetry_uncertainty}
医療サービス市場における消費者の意思決定を考える上で、\cite{Arrow1963}の情報の非対称性および不確実性は不可欠な視点である。医療サー
ビスの特性上、患者は自身の症状の深刻度やどの医療機関が適切な治療を提供できるかについて医師ほど完全な情報を持っていない。そのため、
患者は「大病院である」という規模や設備、評判を医療の質を示すシグナルとして利用し、たとえ軽症であっても安心を求めて大病院を選好する
傾向がある。

\subsection{時間価格と金銭価格のトレードオフ}
\label{sec:time_money_tradeoff}
\cite{Acton1973}が示したように、保険等によるカバーがあり、医療サービスの金銭価格が低い状況下では、移動時間や待ち時間のような時間価格
が需要を決定する主要因となる。通常、大病院は待ち時間が長く時間価格が高い。それでも患者が大病院を選好する傾向にあるのは、期待される
医療の質による効用が、時間コストを上回るからである。標準的な消費者理論によれば、医療サービスの需要は価格(自己負担額)の減少関数で
あると考えられるが、選定療養費の徴収義務化は紹介状を持たずに大病院を受診する場合の実質価格、すなわちシャドウ・プライスを上昇させる。
これによって代替財である診療所・クリニックへの受診が相対的に安価になるため、代替効果によって軽症患者の受診先は大病院から診療所・
クリニックへとシフトすることが予測される。

\subsection{仮説の提示}
\label{sec:hypothesis}
以上より、合理的な患者個人は、制度の施行後にはまず近隣の診療所を受診し、必要に応じて紹介状を取得してから大病院を受診するようになる。
また、軽微な症状であれば受診自体を控える可能性もある。いずれの場合においても、紹介状を持たない大病院の初診件数は減少するはずである。
しかし一方で、前述の情報の非対称性が、この価格メカニズムの機能を阻害する可能性も存在する。患者は医師に比べて自身の症状の深刻度を
正確に判断することができないため、たとえ7,000円の追加負担があったとしても、万が一重病であるリスクを大きく見積もり、設備や評判の
整った大病院を強く選好し続ける可能性がある。本研究では、これらの理論的背景を踏まえ、以下の仮説を検証する。\\
仮説1: 選定療養費の徴収義務化は、対象病院における初診外来件数を有意に減少させる。\\
仮説2: その減少効果は病院側へのペナルティ導入による厳格な運用開始に伴い制度適用直後から即時的に現れる。


%--------------------------------------------------

\chapter{先行研究}
\label{chap:literature}
本章では、本研究のテーマの基盤となるような先行研究を整理する。 本研究の問いは、経済学的には情報の非対称性、医療サービス需要の価格
弾力性、患者の受診行動と病院選択、そして選定療養費の評価といったトピックに関連する。以下、それぞれのトピックについて既存研究を概観
する。

\section{情報の非対称性と医療サービス需要の価格弾力性}
\label{sec:asymmetry_lit}
\subsection{理論的背景:情報の非対称性と不確実性}
\label{sec:asymmetry_lit_theory}
医療サービス需要を経済学的に分析する際の出発点は、Arrow \cite{Arrow1963} による古典的議論である。Arrowは、医療市場における情報の非対称性
と不確実性の存在により、医療サービスについては通常の市場メカニズムが機能しない可能性を指摘した。患者は自身の健康状態や必要な治療に
ついて医師ほどは十分な知識を持たないため、価格のみならず、医師への信頼や病院の評判を品質のシグナルとして行動する。情報の非対称性に
よって価格メカニズム、すなわち選定療養費による誘導が必ずしも合理的になされない可能性があり、本研究の分析においても重要な理論的背景
となる。また、Grossman \cite{Grossman1972} は医療需要を健康資本への投資としてモデル化し、医療サービスが消費財であると同時に投資財としての側面
を持つことを理論化した。

\subsection{海外における実験的アプローチ:RAND HIEとOregon HIE}
\label{sec:asymmetry_lit_rand_oregon}
医療需要が価格(自己負担)に対してどのように反応するかについては、米国を中心に多くの実証研究が蓄積されている。 Manning et al. \cite{Manning1987} 
ではRAND HIEによって、RCTにより自己負担率の上昇が医療支出を有意に抑制することを実証し、外来医療の価格弾力性は約-0.2であると推計
した。この結果は、医療需要は金銭的インセンティブによってある程度制御可能であることを示唆している。 より近年では、
Finkelstein et al. \cite{Finkelstein2012} がオレゴン州のMedicaid抽選を自然実験として利用した研究(Oregon HIE)を行い、保険適用が外来受診
や処方薬の利用を有意に増加させることを示した。しかし一方で、Brot-Goldberg et al. \cite{BrotGoldberg2017} は、高額な免責金額(Deductible)を
課された患者が不要な医療だけでなく必要な医療まで一律に減らしてしまう現象を報告しており、医療に関して患者が選別行動をとることの難し
さを示している。

\subsection{日本における実証的証拠}
\label{sec:asymmetry_lit_japan}
日本の公的データを用いた研究においても、医療需要が価格に対して有意に反応することが確認されている。 
Bhattacharya et al. \cite{Bhattacharya1996} は、1990年の患者調査を用いて外来受診行動を分析し、日本においても医療需要が価格に対して右下がりの
需要曲線(弾力性は-0.12〜-0.54程度。)を持つことを示した。 また、Shigeoka \cite{Shigeoka2014} は、70歳時点での窓口負担率の低下(3割から
1割等)を利用した回帰不連続デザインを用い、高齢者の外来受診行動が価格に敏感に反応することを示している(弾力性は-0.2程度。)。
また、Iizuka and Shigeoka \cite{IizukaShigeoka2022} は小児医療費助成の分析において、数百円程度の少額の自己負担であっても受診行動に有意な抑制効果
を持つことを示しており、日本の患者が価格シグナルに対して敏感であることを裏付けている。湯田 \cite{Yuda2023} はこれらの国内研究を包括的に
サーベイし、日本の医療需要の価格弾力性が総じて低いものの有意に負であることを確認している。Izumida \cite{Izumida2004} は1997年の自己負担率
の2割引き上げの影響をレセプトデータを用いて分析し、患者の自己負担率増は医療機関への受診確率を有意に引き下げるものの、一度受診した
後の受診形態の選択には有意な影響を与えないことを明らかにした。これは選定療養費のような最初の受診時の負担が、初診アクセスの抑制には
有効であるが、その後の受診形態には影響しにくいことを示唆しており、本研究の仮説構築の観点から重要である。

\section{患者の受診行動と病院選択}
\label{sec:hospital_choice_lit}
選定療養費制度は、患者を大病院から診療所・クリニックへと誘導する(病院の機能分化)ことを目的としている。したがって、患者がどの
ような要因で病院を選択しているかに関する先行研究について概観する。

\subsection{時間価格と質のトレードオフ}
\label{sec:time_quality_tradeoff}
患者の病院選択において、距離・時間などアクセスにかかるコストは非常に重要な要因である。Acton \cite{Acton1973} は、医療サービスの金銭価格が
低い場合には金銭価格に代わって時間価格が需要を決定する主要因になることを理論的・実証的に示した。これは、日本のフリーアクセス制度下
においては待ち時間の長い大病院が敬遠される、あるいは逆に近隣の診療所が選好されることを説明する。 一方で、患者は大病院で受けられる
医療サービスの質を選好する。Tay \cite{Tay2003} は離散選択モデルを用い、患者が距離的なコストと医療の質をトレードオフにかけて病院を選択し
ていることを示した。すなわち、患者はある程度の時間・金銭コストを支払ってでも質の高い医療を受けたいと考える傾向にあると言える。

\subsection{本邦における患者選好とバイパス行動}
\label{sec:hospital_choice}
日本における患者の受療行動については、Yamamoto \cite{Yamamoto2002} が国保レセプトデータを用いて分析し、患者が診療所・中小病院・大病院を症状
に応じて使い分けている実態を示した。しかし、患者が大病院を志向する背景には、単なる症状の重さだけでなく、大病院への信頼や設備への
期待があることがわかっている。Yagi \cite{Yagi2008} は、患者への定性調査(KJ法)に基づき、医療機関選択において立地や利便性のみならず、医療
機器の充実や評判といった情報が重視されていることを明らかにしている。 これらの研究から、患者は強い質への選好を持って大病院を選ぶ傾向
にあることが読み取れるため、選定療養費がどの程度行動を変容させるかの度合いが重要となる。

\section{選定療養費制度の評価と課題}
\label{sec:sentei_lit}
\subsection{表明選好法による制度導入の事前評価}
\label{sec:sentei_lit_sp}
紹介状のない外来受診に対する選定療養費の徴収義務化が患者の受診行動に与える影響については、制度導入前後においていくつかの予測研究が
行われている。 Sugahara \cite{Sugahara2015Rep} および Kawaguchi \& Sugahara \cite{KawaguchiSugahara2015} は、コンジョイント分析(表明選好法)を用いたアンケート
調査により、定額自己負担の導入が受診行動に与える影響をシミュレーションした。これらによれば、軽症患者の場合、5,000円から10,000円
程度の定額負担を設定することで、大病院から診療所への誘導が期待できるとされる。一方で、重症患者の受診行動は価格に対して非弾力的で
あることも示されている。これらは2016年の制度導入(医科初診時に最低5,000円徴収)の根拠となった。

\subsection{顕示選好データによる事後検証の現状}
\label{sec:sentei_lit_rp}
しかし、実際に制度が義務化された後の効果を大規模な顕示選好データを用いて検証した研究はほとんど見当たらない。例外として、
Iba et al. \cite{Iba2023} が挙げられる。これは茨城県のDPCデータを用いて2016年の制度導入の影響を分析したもので、
Controlled Interrupted Time-Series分析の結果、義務化対象となった病院では紹介率が約5\%ポイント上昇した一方、対象外の病院では
変化が見られなかったことを報告している。これは、制度導入によって病院の機能分化が一定程度進行したことを示す実証結果であり、
Kobayashi et al. \cite{Kobayashi2019} などの記述的な分析においても、制度導入後に大病院の初診患者数が減少傾向にあることが報告されている。

\subsection{先行研究の限界と本研究の貢献}
\label{sec:sentei_lit_limit}
紹介状のない外来受診に対する選定療養費の徴収義務化に関する先行研究はアンケートに基づく事前予測(Sugahara \cite{Sugahara2015Rep})や、特定の地域・
時点に限られた分析(Iba et al. \cite{Iba2023})に留まっている。特に、2022年10月の制度拡大は保険給付範囲からの控除によって病院に対する
徴収強制力が強化されたという点で非常に重要であるにもかかわらず、これについて全国規模のレセプトデータを用いて分析した研究は、筆者が
知る限り存在しない。 本研究は2022年の制度拡大という外生的な制度変更を利用し、全国の病院パネルデータを用いた差の差分析を行うこと
で、選定療養費の徴収義務化が患者の受診行動に与えた因果効果を検証するものである。これにより、Izumida \cite{Izumida2004} が示した受診確率の
抑制効果が、定額負担という形態においても成立するか、また Iba et al. \cite{Iba2023} で見られた効果が平均的には全国でも観察されるかを
明らかにする。

%--------------------------------------------------

\chapter{データ}
\label{chap:data}
本章では分析に用いるデータセットの構築手順、変数の定義、および記述統計について述べる。本研究では全国規模のレセプトデータを用いて制度変更の影響を
受けた病院(処置群)と受けなかった病院(対照群)を識別し、病院×月単位のパネルデータを構築する。

\section{データソース}
\label{sec:data_source}
本研究では、全国健康保険協会(以下、協会けんぽ)が管理・保有する「診療報酬明細書(レセプト)データ」を使用する。データの利用にあたっては、個人の
特定が不可能な形で匿名化された個票データの提供を受けた。

\subsection{データの特性と限界}
\label{sec:data_limitations}
協会けんぽは主に中小企業の従業員(被保険者)とその家族(被扶養者)を主な加入者とする日本最大規模の医療保険者であり、その加入者数は約4,000万人に
達する。しかし、NDB(レセプト情報・特定健診等情報データベース)のような日本全体の悉皆データと比較した場合本データには以下の限界が存在する。\\
第一に、サンプルに高齢者が含まれていないという点が挙げられる。協会けんぽの被保険者は就労者とその家族であり、75歳以上の後期高齢者は後期高齢者医療
制度へ移行するため原則としてデータに含まれない。また、65歳から74歳の前期高齢者についても定年退職に伴い国民健康保険へ移行するケースが多く、捕捉率
が低下する傾向にある。したがって、本研究の分析対象は、主に現役世代とその家族に限定される。\\
第二に、職業等に偏りがある。協会けんぽの加入者は中小企業の従業員が中心であり、大企業の従業員(組合健保)や公務員(共済組合)、自営業者(国保)は
含まれない。\\
第三に、追跡の打ち切りの問題が挙げられる。就職、転職、退職等によって被保険者資格を喪失した場合その時点でデータの追跡が途切れることとなる。

\subsection{本データを使用する妥当性}
\label{sec:data_validity}
上記のような限界はあるものの、本研究において協会けんぽデータを使用することは、以下の理由から妥当であると考えられる。 まず、約4,000万人というサン
プルサイズは統計的な検出力の観点から十分であると考えられる。 次に、政策評価の観点からの重要性である。選定療養費(定額負担)のような価格インセン
ティブに対しては、医療費の自己負担割合が1割または2割と低い高齢者層よりも、原則3割負担に直面している現役世代の方がより敏感に反応すると考えられる。
したがって協会けんぽデータは、対象制度の効果を検証する上で十分適切なデータと考える。

\section{サンプル構築}
\label{sec:sample}
本研究では、以下の手順に従って分析用データセットを構築した。

\subsection{集計単位}
\label{sec:aggregation_unit}
利用したデータは2015年4月から2024年3月までの期間における協会けんぽの全レセプトデータである。その中からまず「レセプト」テーブルにおいてレセプト種別が医科、かつ入外区分が
外来であるレセプトのみを抽出した。本研究は外来受診に焦点を当てているため歯科、調剤、および入院レセプトは分析対象から除外した。 次に、抽出された
レセプトデータを、医療機関コードおよび診療年月でグループ化して集計し、病院×月単位のパネルデータを作成した。

\subsection{除外基準}
\label{sec:exclusion_criteria}
分析にあたっては、以下の基準に基づきデータセットの絞り込みを行った。\\
第一に施設形態による絞り込みを行った。2022年10月から選定療養費徴収義務化の対象となる病院は、紹介受診重点医療機関を含む許可病床200床以上の病院
となった。協会けんぽのデータには病院と診療所を厳密に区別するフラグが存在しないため、本研究では許可病床200床以上と推定される医療機関のみを分析
対象として抽出し、それ以外の小規模病院および診療所はデータセットから除外した。\\
次に処置群と対照群の定義に基づき、サンプルの絞り込みを行った。 処置群は、2022年10月の制度改定以降2024年3月まで継続的に選定療養費を徴収した
記録のある病院群と定義した。 対照群は、許可病床200床以上と推定される病院のうち、処置群に入っていない病院群と定義した。 具体的には、処置群は、
2022年10月の制度改定以前の期間(2015年4月〜2022年9月)の月単位での選定療養費徴収割合が1割以下かつ2022年10月の制度改定以降2024年3月まで
すべての月で選定療養費を徴収している病院(Switchers)とした。 対照群は、分析期間(2015年4月〜2024年3月)を通じて選定療養費の徴収実績が
一度も確認されなかった病院(Never Treated)とした。 なお、これら以外の病院(期間の途中で不規則に徴収を行っている病院や、制度改正前から常に
徴収していた病院)は、分析結果にバイアスを与える可能性があるためサンプルから除外した。

\subsection{最終的な分析サンプル}
\label{sec:final_sample}
以上のプロセスを経て構築された最終的な分析サンプルは、以下の通りである。

\section{変数定義}
\label{sec:variables}
以下では、構築した分析用データセットに含まれる主要な変数の定義について述べる。本データセットは、観測単位を病院 $i$ 、診療年月 $t$ とする
パネルデータである。病院ID(hospital\_id)については、「レセプト」テーブルの都道府県コード、点数表コード、医療機関コードの組み合わせにより
作成した。これは協会けんぽのガイドラインに沿った作成方法である。

\subsection{被説明変数}
\label{sec:dependent_variables}
本研究では、選定療養費の導入および拡大が患者の受診行動に与える影響を検証するため、以下の4つの変数を被説明変数として設定した。外来件数に
ついては分析では自然対数変換を施した値を用いた。\\
第一の被説明変数は対数初診外来件数(log\_shoshin\_outpatients)である。これは、「レセプト診療行為」テーブルに初診料を表す診療行為コードが
記録されているレセプトの枚数を病院・年月ごとにグループ化して集計して shoshin\_outpatients とし、その自然対数をとったものである。選定
療養費の対象となるのは初診患者であるため、この変数を主要な被説明変数とした。\\
第二の被説明変数は対数再診件数 (log\_saishin\_outpatients) である。これは、「レセプト診療行為」テーブルに再診料、あるいは外来診療料を表す診療行為コードが
記録されているレセプトの枚数を病院・年月ごとにグループ化して集計して saishin\_outpatients とし、その自然対数をとったものである。これは初診よりも件数が多い再診への影響を確認するために使用する。
第三の被説明変数は対数総外来件数 (log\_total\_outpatients) である。これは、「レセプト」テーブルの「入外区分」列の値により外来レセプトの
総枚数 (total\_outpatients) をカウントし、対数変換したものである。初診だけでなく再診を含めた外来全体への影響を確認するために使用する。\\

\subsection{説明変数}
\label{sec:explanatory_variables}
差の差分析およびイベントスタディのために、つぎの変数を定義する。処置群ダミー (treated) は、選定療養費の徴収義務化対象病院を識別
するダミー変数である。具体的には2022年10月の制度改定以前の期間(2015年4月〜2022年9月)の月単位での選定療養費徴収割合が1割以下かつ2022年10月の制度改定以降2024年
3月まですべての月で選定療養費を徴収している病院を $1$、そうでない病院を $0$ と定義する。また、イベント
スタディのために経過時間変数を作成した。event\_timeは2022年10月を基準時点($0$)として経過月数を表す変数である。

\subsection{共変量と固定効果}
\label{sec:covariates_fixed_effects}
病院の規模や機能の代理変数として診療科数 (department\_count) を使用する。これは、各病院について「レセプト」テーブルの診療科
コード(診療科1〜3)に記録されたユニークな診療科の数をカウントしたものである。 また、地域特有のショックを制御するため病院所在地の
郵便番号の上2桁(postal\_2)を用いて地域を識別し、地域×月固定効果としてモデルに組み込む。

\section{記述統計}
\label{sec:descriptive}
本節では分析に用いるデータセットの基本統計量を示す。

\subsection{要約統計量}
\label{sec:summary_statistics}
分析に使用した全サンプルの主要変数の要約統計量を表\ref{tab:summary}に示す。 被説明変数である外来件数(実数)を見ると、月間初診外来件数
(shoshin\_outpatients)の平均は約266件、再診外来件数(saishin\_outpatients)の平均は約1,225件であった。 特筆すべきは標準偏差の
大きさであり、初診件数の最小値が10件、最大値が3,796件であることからも分かるように、分析対象とした「200床以上」の病院群の中にも、規模に
おいて大きなばらつき(異質性)が存在することが確認できる。 また、病院の機能の代理変数である診療科数(department\_count)についても、平均
約13科に対し、最大で102科を有する大規模病院が含まれている。 これらの病院ごとの規模や機能の違いは、固定効果モデルを用いることで制御する。

% --- Table 1 (generated from table1_summary.csv) ---
\begin{table}[htbp]
\centering
\caption{要約統計量}\label{tab:summary}
\begin{tabular}{lrrrrr}
\toprule
 & count & mean & sd & min & max \\
\midrule
shoshin\_outpatients & 154886 & 265.7033 & 206.4181 & 10 & 3796 \\
saishin\_outpatients & 154886 & 1224.928 & 1123.344 & 10 & 10450 \\
total\_outpatients & 154886 & 1490.632 & 1260.681 & 20 & 11261 \\
department\_count & 154886 & 13.1589 & 6.847548 & 1 & 102 \\
\bottomrule
\end{tabular}
\end{table}

\subsection{処置群と対照群の比較}
\label{sec:balance_check}
次に、制度導入前の期間(2015年4月〜2022年9月)における処置群と対照群の特性を比較するため、主要変数の平均値の差の検定(t検定)を行った
(表\ref{tab:balance})。表\ref{tab:balance}が示すように、すべての変数において処置群と対照群の間には統計的に有意な差(1\%水準)が
確認された。 具体的には、月間初診外来件数(実数)の平均値は対照群が約360件であるのに対し、処置群は約201件と有意に少ない。同様に、再診外来
件数や総外来件数についても対照群の方が規模が大きい傾向にある。また、病院の機能を示す診療科数についても、対照群の平均が約17.6科であるのに対し、
処置群は約9.8科であり、対照群には相対的に規模の大きな多機能な病院が多く含まれていることがわかる。しかし差の差分析(DID)においては、平行トレンド仮定が成立していればこれらの時間不変
のレベル差は病院固定効果として吸収され推定結果にバイアスを与えないが、仮定が成立しているかについては自明ではない。この点については、第6章のイベントスタディ分析において、処置前の期間の係数の挙動を含めて議論する。

\begin{table}[htbp]
\centering
\caption{処置群と対照群のバランスチェック(制度導入前:2015年4月〜2022年9月)}
\label{tab:balance}
\begin{tabular}{lcccc}
\toprule
& \multicolumn{1}{c}{(1)} & \multicolumn{1}{c}{(2)} & \multicolumn{2}{c}{(3)} \\
& Treatment & Control & \multicolumn{2}{c}{Difference} \\
& Mean & Mean & \multicolumn{2}{c}{(1)-(2)} \\
\midrule
初診外来件数 (月間) & 201.1 & 360.3 & -159.2$^{}$ & (1.10) \\
再診外来件数 (月間) & 683.9 & 1938.5 & -1254.6$^{}$ & (5.27) \\
総外来件数 (月間) & 885.0 & 2298.8 & -1413.8$^{}$ & (5.92) \\
診療科数 & 9.8 & 17.6 & -7.8$^{}$ & (0.03) \\
\midrule
観測数 ($N \times T$) & 73,001 & 55,098 & \multicolumn{2}{c}{128,099} \\
\bottomrule
\multicolumn{5}{l}{\footnotesize 注: 括弧内は標準誤差。$^{***}$ $p<0.01$, $^{**}$ $p<0.05$, $^{*}$ $p<0.1$。} \\
\multicolumn{5}{l}{\footnotesize 制度導入前の期間におけるプールされたデータを用いたt検定の結果である。}
\end{tabular}
\end{table}



%--------------------------------------------------

\chapter{分析方法}
\label{chap:analysis_method}
本章では構築したパネルデータを用いて選定療養費の徴収義務化が患者の受診行動に与える影響を推定するための実証戦略について述べる。

\section{識別戦略}
\label{sec:identification_strategy}
本研究の目的は選定療養費の徴収義務化が対象病院のアウトカムに与える平均処置効果(ATT)を推定することである。 本研究では2022年10月
の制度改定を外生的なショックと見なし、差の差分析(DID)を行う。DID推定量が因果効果として解釈できるためには平行トレンド仮定が満た
されている必要がある。すなわち、処置群を $D=1$、対照群を $D=0$、時点 $t$ における潜在的アウトカムを $Y_{it}(0)$(処置なし)
としたとき、制度導入後の任意の時点 $t$ において以下が成立することを要請する。
\begin{align*}
  E[Y_{it}(0) - Y_{i,t-1}(0) | D=1, X_{it}] = E[Y_{it}(0) - Y_{i,t-1}(0) | D=0, X_{it}]
\end{align*}

\section{推定モデル}
\label{sec:models}
\subsection{差の差分析 (Difference-in-Differences)}
\label{sec:did}
選定療養費の導入が初診外来件数等のアウトカムに与える平均的な効果を推定するため、以下の Two-Way Fixed Effectsモデルを推定する。
\begin{align*}
  \ln Y_{it} = \gamma (\text{Treat}_i \cdot \text{Post}_{it}) + \beta X_{it} + \alpha_i + \delta_t + \epsilon_{it}
\end{align*}
ここで、$Y_{it}$ は病院 $i$ の月 $t$ におけるアウトカムである。$\text{Treat}_i$ は処置群ダミー、$\text{Post}_t$ は制度
改定後(2022年10月以降)を示すダミー変数である。モデルには、病院固定効果 $\alpha_i$ および月固定効果 $\delta_t$ を含める。
さらに、時間を通じて変化する病院属性を制御するため、コントロール変数 $X_{it}$ を導入する。具体的には、病院の規模や機能を反映して
いる診療科数を含める。また、月次固定効果に代わり地域(郵便番号上2桁)$\times$ 月固定効果を導入した推定も行う。DID推定量は交差項の
係数 $\gamma$ であり、これが制度導入による平均処置効果(ATT)を表す。標準誤差については病院レベルでのクラスターロバスト標準誤差を使用する。

\subsection{イベントスタディ}
\label{sec:eventstudy}
処置効果の動的な推移を明らかにするため、以下のイベントスタディモデルを推定する。
\begin{align*}
  \ln Y_{it} = \beta_{pre} \cdot \mathbf{1}(t < E_i - 12) \cdot \text{Treat}_i + \sum_{k=-12, k \neq -1}^{17} \beta_k \cdot \mathbf{1}(t = E_i + k) \cdot \text{Treat}_i + \theta X_{it} + \alpha_i + \delta_t + \epsilon_{it}
\end{align*}
ここで、$E_i$ は制度改定が行われた年月(2022年10月)を表す。$\mathbf{1}(\cdot)$ は指示関数であり、括弧内の条件が満たされる
場合に1をとる。すなわち、$\mathbf{1}(t = E_i + k) \times \text{Treat}_i$ は、処置群の病院において、制度改定から
 $k$ ヶ月経過した時点($k$ が負の場合は改定前)であることを示すダミー変数である。 改定直前 $k = -1$ の月は分析の基準点として
 除外しており、イベントスタディ係数は係数は指示関数の係数$\beta_k$ であり、基準月と比較した際のアウトカムの差分を表す。これに
 より制度の動学的な効果を推定する。なお、2021年9月以前については一括でイベントスタディ係数$\beta_{pre}$を推定し、$k = -13$の月として扱う。

\section{識別仮定の検証と頑健性}
\label{sec:hypothesis_robustness}
\subsection{事前トレンドの検証}
\label{sec:test_pretrend}
平行トレンド仮定の妥当性を検証するため、前述したイベントスタディモデルの推定結果を利用する。 具体的には、制度導入前の期間
$k < -1$における係数 $\beta_k$ に注目し、$\beta_k$ が統計的にゼロと有意に異なっていない場合制度導入前の処置群と対照群の
トレンドに有意な差はなかったことになり、平行トレンドの仮定はデータと矛盾しない。

\subsection{プラセボテスト}
\label{sec:placebo_test}
また、偽の処置時点を用いたプラシーボテストを行う。 具体的には制度改定の影響を受けていない期間、例えば2019年4月を偽の処置時点と
して同様のDID推定を行う。この分析において偽の処置効果が統計的に有意でなければ、仮定成立が支持される。


%--------------------------------------------------

\chapter{分析結果}
\label{chap:results}
本章では前章で定義した実証モデルに基づく推定結果を提示する。

\section{動学的処置効果と事前トレンドの検証}
\label{sec:dynamic_effects}
DID分析が因果効果として解釈可能であるための前提条件は平行トレンド仮定の成立である。これを検証し、かつ制度導入後の効果の持続性や
即時性を確認するために実施したイベントスタディの結果を \textbf{Figure \ref{fig:event_study}} に示す。図中のドットは推定
された係数 $\hat{\beta}_k$ を、縦棒は95\%信頼区間を表している。基準時点は制度導入直前の月(2022年9月、$k=-1$)であり、
係数$\hat{\beta}_k$はこの時点との相対的な差分(対数値)として表現されている。

\begin{figure}[htbp]
  \centering\includegraphics[width=1.0\textwidth]{figure3_event_study.png}
  \caption{イベントスタディによる動学的処置効果の推定結果}
  \label{fig:event_study}
  \footnotesize{注: アウトカムは対数変換された外来件数。エラーバーは95\%信頼区間を示す。基準時点は2022年9月(k=-1)。}
\end{figure}

\subsection{事前トレンドの検証}
平行トレンド仮定の成立を検証するため、制度導入前の期間($k < -1$)における係数を確認する。対数初診外来件数(Panel A)および
対数総外来件数(Panel C)についてはいくつかの時点で係数はゼロ近傍の値を取るものの、全体的には有意にゼロと異なる値を取る月が多く、
基準時点以前においては一貫したトレンドが見られない。しかしながら対数総外来件数(Panel C)は対数初診外来件数(Panel A)に比べて
係数の絶対値が小さく、非常に小さい水準であることに留意が必要である。対数再診外来件数(Panel B)についてはよりゼロ近傍を取る値が
多いものの、やはり基準時点以前においては一貫したトレンドは見られない。こちらについても対数初診外来件数(Panel A)に比べて
係数の絶対値が小さく、非常に小さい水準であることに留意が必要である。

\subsection{動学的処置効果}
制度導入後の動学的な処置効果を確認する。まず対数初診外来件数(Panel A)について基準時点である2022年10月($k=0$)以降の
推移を見ると、導入直後の $k=1$(1ヶ月後)から $k=3$(3ヶ月後)にかけて係数は有意な負の値を示し、特に $k=2$ においては約 $-0.15$
(基準点比で約15\%減)まで急落している。その後、$k=4$ 以降は係数が正の値($+0.05 \sim +0.10$程度)に転じている。ここで減少トレンド
が途絶えており、$k \geq 4$ 以降は一貫したトレンドが見られない。次に対数再診外来件数(Panel B)については、$k=0$ 以降$k=3$(3ヶ月後)
にかけて多少落ち込む動きを見せているものの、その後は安定してゼロ近傍を推移している。これは紹介状のない再診時に課される選定療養費の額が
初診時のものより低いことから、患者が金銭的なペナルティに反応しなかったことを反映していると考えられる。対数総外来件数(Panel C)は
対数初診外来件数(Panel A)と同様に導入直後の $k=1$(1ヶ月後)から $k=3$(3ヶ月後)にかけて係数は有意な負の値を示しているものの
$k \geq 4$ 以降は一貫したトレンドが見られない。しかし係数の絶対値は対数初診外来件数(Panel A)に比べて非常に低い水準を示しており、
制度の効果は総外来件数に対しては薄かったことがわかる。これは外来患者全体に占める再診の割合が初診の割合に比べて高いことによるものと
考えられる。

\section{平均処置効果の推定結果}
\label{sec:did_results}
 \textbf{Table \ref{tab:did_results}} はTwo-Way Fixed Effect DIDモデルの推定結果を示している。表中の
 (1)列は対数初診外来件数、(2)列は対数再診外来件数、(3)列は対数総外来件数を被説明変数とした結果である。すべてのモデルにおいて、病院固定
 効果、月固定効果、および時間変動する共変量(診療科数)がコントロールされている。
 
\begin{table}[htbp]
\centering
\caption{DID推定による平均処置効果}\label{tab:did}
\begin{threeparttable}
\begin{tabular}{lccc}
\toprule
 & \multicolumn{1}{c}{Log First} & \multicolumn{1}{c}{Log Return} & \multicolumn{1}{c}{Log Total} \\
\midrule
did\_term & -0.1230*** & -0.0174** & -0.0401*** \\
 & (0.0103) & (0.0084) & (0.0079) \\
\addlinespace
department\_count & 0.0076* & 0.0080* & 0.0074* \\
 & (0.0042) & (0.0044) & (0.0042) \\
\addlinespace
\_cons & 5.1138*** & 6.5632*** & 6.8153*** \\
 & (0.0558) & (0.0584) & (0.0556) \\
\addlinespace
\midrule
Observations & 154886 & 154886 & 154886 \\
Adj. R-squared & 0.944 & 0.983 & 0.980 \\
\bottomrule
\end{tabular}
\begin{tablenotes}[flushleft]
\footnotesize
\item Note: All models include hospital and month fixed effects.
\item Standard errors clustered at the hospital level are in parentheses.
\end{tablenotes}
\end{threeparttable}
\end{table}

\subsection{初診外来件数に対する効果}
第(1)列について、平均処置効果(ATT)は $-0.1230$であり、1\%水準で統計的に有意であった。被説明変数は対数変換されているため、
これは選定療養費の徴収義務化が初診外来件数を平均的に12.3\%減少させたことを意味している。これは前項でのイベントスタディの結果とも
整合的であるものの、処置以前の期間について推定された係数が全体的に高い水準にあったことに留意が必要であるものの、少なくともこの結果からは
最低7,000円という定額負担額に対して患者がある程度は反応したと考えられる。

\subsection{再診および総外来件数に対する効果}
第(2)列の再診外来件数について、平均処置効果(ATT)は $-0.0174$ であり、5\%水準で統計的に有意であった。しかし、その減少幅は約1.7\%に過ぎず、
初診に対する効果(12.3\%減)と比較して約7分の1程度の大きさである。再診時にも選定療養費は設定されているものの最低5,000円と初診よりも低く、
継続的な治療を必要とする患者の需要は価格に対して非弾力的であることが示唆される。第(3)列について、平均処置効果(ATT)は $-0.0401$ であり、
1\%水準で統計的に有意であった。外来全体として患者数の減少率が4\%程度に留まったことは、制度導入によって大病院混雑の抜本的解決には至らなかったことを
示している。

\section{頑健性の確認}
\label{sec:robustness}
前項までの主分析結果の信頼性を確認するため、頑健性検証を行った。特にDID分析の識別仮定である平行トレンド仮定に関して偽の処置時点を設定した
プラセボテストを実施した。 \textbf{Table \ref{tab:robustness}} はその結果を示したものである。\\
ここでは、実際の制度改定(2022年10月)以前、かつコロナ禍以前の期間である2019年4月を偽の処置時点と設定し、2018年4月から2020年3月までの
データを用いて同様のDID推定を行った。もし処置群と対照群の間に構造的なトレンドの乖離が存在するならば、この期間においても有意な係数が推定
されることとなる。

\begin{table}[htbp]
  \centering
  \caption{頑健性検証:偽の処置時点を用いたプラセボテスト}
  \label{tab:robustness}
  \begin{tabular}{lccc}
    \toprule
    & (1) & (2) & (3) \\
    & Log First & Log Return & Log Total \\
    & (初診外来) & (再診外来) & (総外来) \\
    \midrule
    Pseudo-DID Term & 0.0155$^{**}$ & -0.0061 & 0.0022 \\
    (Fake Treat $\times$ Post) & (0.0068) & (0.0065) & (0.0056) \\
    [1ex]
    Department Count & 0.0048$^{**}$ & 0.0049$^{**}$ & 0.0043$^{*}$ \\
    & (0.0024) & (0.0024) & (0.0022) \\
    [1ex]
    Constant & 5.1854$^{***}$ & 6.5895$^{***}$ & 6.8494$^{***}$ \\
    & (0.0319) & (0.0317) & (0.0291) \\
    \midrule
    Observations & 84,277 & 84,277 & 84,277 \\
    Hospital FE & Yes & Yes & Yes \\
    Month FE & Yes & Yes & Yes \\
    \bottomrule
    \multicolumn{4}{l}{\footnotesize \textit{Notes:} 括弧内は病院レベルでクラスター化された標準誤差。} \\
    \multicolumn{4}{l}{\footnotesize $^{***}$ $p<0.01$, $^{**}$ $p<0.05$, $^{*}$ $p<0.1$。} \\
    \multicolumn{4}{l}{\footnotesize 分析期間は2018年4月〜2020年3月(コロナ禍以前)に限定。} \\
    \multicolumn{4}{l}{\footnotesize 2019年4月を偽の処置時点として設定。} \\
  \end{tabular}
\end{table}

分析の結果推定されたDID項の係数は $+0.0155$ であり、5\%水準で統計的に有意となった。この結果からは平行トレンド仮定が満たされていないこと
を示唆しており、コロナ禍以前の期間において処置群は対照群に比べて初診件数がわずかに増加するトレンドを持っていたと考えられる。\\
しかし主分析において、初診外来に関する係数の符号は負であった。これは制度導入による負の方向の影響が元々存在していた正のトレンドを打ち消し、さらに
マイナスに転じさせるほどであったことを意味する。また、プラセボテストにおける係数の絶対値($0.0155$)は主分析における係数($-0.1230$)の約8分の1
程度にすぎず、制度導入によるショックと比較して十分に小さい。
以上のことから本研究の結果は事前トレンドの存在を考慮してもなお頑健であると言える。

\section{処置効果の異質性}
\label{sec:heterogeneity}

最後に選定療養費の影響が病院の属性によってどのように異なるかを明らかにするため、病院が所在する地域の特性および病院規模による異質性分析を行った。
\textbf{Table \ref{tab:heterogeneity}}は2つの分析結果をまとめて示した表である。

\begin{table}[htbp]
  \centering
  \caption{処置効果の異質性分析(地域別および病院規模別)}
  \label{tab:heterogeneity}
  \begin{tabular}{lcccc}
    \toprule
    & \multicolumn{2}{c}{By Region} & \multicolumn{2}{c}{By Hospital Size} \\
    \cmidrule(lr){2-3} \cmidrule(lr){4-5}
    & (1) & (2) & (3) & (4) \\
    & Urban & Rural & Large & Small \\
    & (都市部) & (地方) & (大規模) & (小規模) \\
    \midrule
    \multicolumn{5}{l}{\textbf{Panel A: First Visits (初診外来)}} \\
    DID Term & -0.0895$^{***}$ & -0.1512$^{***}$ & -0.0424$^{**}$ & -0.1820$^{***}$ \\
    (Treat $\times$ Post) & (0.0161) & (0.0133) & (0.0178) & (0.0181) \\
    \textit{Observations} & 67,213 & 87,673 & 77,532 & 77,354 \\
    \midrule
    \multicolumn{5}{l}{\textbf{Panel B: Return Visits (再診外来)}} \\
    DID Term & -0.0104 & -0.0254$^{**}$ & 0.0085 & -0.0405$^{***}$ \\
    (Treat $\times$ Post) & (0.0135) & (0.0103) & (0.0152) & (0.0149) \\
    \textit{Observations} & 67,213 & 87,673 & 77,532 & 77,354 \\
    \midrule
    \multicolumn{5}{l}{\textbf{Panel C: Total Outpatients (総外来)}} \\
    DID Term & -0.0289$^{**}$ & -0.0510$^{***}$ & 0.0039 & -0.0828$^{***}$ \\
    (Treat $\times$ Post) & (0.0126) & (0.0098) & (0.0148) & (0.0127) \\
    \textit{Observations} & 67,213 & 87,673 & 77,532 & 77,354 \\
    \midrule
    Hospital FE & Yes & Yes & Yes & Yes \\
    Month FE & Yes & Yes & Yes & Yes \\
    \bottomrule
    \multicolumn{5}{l}{\footnotesize \textit{Notes:} 括弧内は病院レベルでクラスター化された標準誤差。} \\
    \multicolumn{5}{l}{\footnotesize $^{***}$ $p<0.01$, $^{**}$ $p<0.05$, $^{*}$ $p<0.1$。} \\
    \multicolumn{5}{l}{\footnotesize Urban/Ruralは所在地の区分、Large/Smallは病床数の中央値による分割。} \\
  \end{tabular}
\end{table}

\subsection{地域による異質性}
\textbf{Figure \ref{fig:heterogeneity_region}} は地域による異質性分析の結果を表す。

\begin{figure}[htbp]
  \centering\includegraphics[width=1.0\textwidth]{figure4a_heterogeneity_region.png}
  \caption{病院が所在する地域特性による異質性分析}
  \label{fig:heterogeneity_region}
  \footnotesize{注: アウトカムは対数変換された外来件数。エラーバーは95\%信頼区間を示す。}
\end{figure}

選定療養費の影響が地域特性によってどのように異なるかを検証するため、病院の所在地が「都市部(Urban)」か「地方(Rural)」かによってサンプルを分割し、DID推定を行った。
なお、都市部の定義は三大都市圏(東京、神奈川、千葉、埼玉、愛知、三重、岐阜、大阪、京都、兵庫、奈良)として、それ以外の都道府県を地方部とした。\\
分析の結果、初診外来件数(Panel A)において、都市部の病院の減少幅が約9.0\%(係数 $-0.0895^{***}$)であったのに対し、地方の病院では約15.1\%
(係数 $-0.1512^{***}$)と、地方においてより大きな受診抑制効果が確認された。一般に都市部は代替となる医療機関が豊富に存在するため、需要の価格弾力性
が高くなると予想されるが、本分析の結果は逆の傾向を示している。この結果の解釈として、以下の2つの可能性が考えられる。第一に、地方における「受診控え」の
可能性である。都市部と比較して地方では医療機関へのアクセスが悪く、紹介状を取得するためにまずクリニックを受診すること自体のハードルが高い。
そのため軽症患者が選定療養費を忌避して受診そのものを断念した可能性がある。第二に都市部における病院ブランドへの選好である。都市部の大病院
を受診する患者は大病院の医療の質を高く選好して受診しているケースが多く、価格弾力性が低かった(7,000円払ってでもその病院に行きたかった)可能性がある。

\subsection{病院規模による異質性}
\textbf{Figure \ref{fig:heterogeneity_size}} は病院規模による異質性分析の結果を表す。

\begin{figure}[htbp]
  \centering\includegraphics[width=1.0\textwidth]{figure4b_heterogeneity_size.png}
  \caption{病院の診療科数規模による異質性分析}
  \label{fig:heterogeneity_size}
  \footnotesize{注: アウトカムは対数変換された外来件数。エラーバーは95\%信頼区間を示す。}
\end{figure}

次に、病院規模によってサンプルを分割してDID推定を行った。ただし、レセプトデータからは各病院の病床数を取得することが不可能であるため、各病院の診療科数を規模の代理変数とした。
分析の結果、小規模群の初診件数が約18.2\%(係数 $-0.1820^{***}$)と大幅な減少を示した一方で大規模群の減少幅は約4.2\%(係数 $-0.0424^{**}$)と小幅に留まった。\\
特定機能病院や大規模な地域医療支援病院に近い大規模病院はサンプル内の病院の中でもより高度な医療を提供しており、患者にとって代替財が存在しにくい傾向にあり、選定療養費が課されても
受診を継続する傾向が強いと考えられる。それに対してより小規模な病院は大規模病院と比較すると診療所やクリニックに近いため、患者が受診行動の変化を起こしやすかったと考えられる。\\
この結果からは、制度導入が許可病床200床以上ではあるが相対的には規模が大きくない病院、すなわち中規模病院において機能分化の役割をより果たしたことを示唆している。

%--------------------------------------------------

\chapter{考察}
\label{chap:discussion}
本章では分析結果に基づき、制度が患者の受診行動に与えた影響について考察を行う。
第一に、制度導入直後の短期的な影響と、初診・再診による反応の違いから患者の受診行動メカニズムについて議論する。次に、病院特性(規模・混雑度)に
よる異質性分析の結果を用いてどのような病院において政策の効果が現れたのかについて議論する。最後に、これらの知見に基づいた政策的含意および
本研究の限界および今後の課題を提示する。

\section{患者の受診行動とメカニズム}
\label{sec:mechanism}
DID推定の結果、定額負担の増額は許可病床200床以上病院の初診患者に対して統計的に有意かつ平均的には小さくない受診抑制効果をもたらしたことが
明らかになった。具体的には、制度変更後に初診外来件数は対数モデルで平均約12.3\%(係数:$-0.123$)減少したことが確認された(Table \ref{tab:did_results} )。
この結果は患者が初診に対する選定療養費に対して高い価格弾力性を持つことを示唆している。ここから7,000円以上の金銭的コストは患者にとって
紹介状なしで大病院を受診するという選択を思いとどまらせ、近隣の診療所へ受診先を変更する、あるいは受診自体を控えるといった行動変容を促すのに十分な
ペナルティとして機能したと言える。イベントスタディの結果(Figure \ref{fig:event_study})より、制度導入直後の1〜3ヶ月間において初診件数の
減少トレンドが観測された。これは、メディア報道や病院による周知によって制度変更直後は患者が受診行動を考え直して診療所等を受診、または受診控えをした
ためと考えられる。しかし導入から4ヶ月目以降は減少トレンドは解消し、制度導入時点からはほとんど変わらない水準で推移した。
これは、制度導入時の衝撃によって本来受診が必要な患者までもが過剰に受診を控えてしまった可能性がある一方で、時間の経過とともに患者が新たな価格体系
に適応し、「7,000円を払ってでも受診する必要がある」と判断した場合には診療所・クリニックを経由せずに大病院を受診していた可能性を示唆
している。つまり、価格メカニズムによる抑制効果は導入初期に強く発揮される一方で患者の慣れなどの原因で長期的な効果は逓減すると考えられる。

\begin{figure}[H]
  \centering
  \includegraphics[width=1.0\textwidth]{figures/figure3_event_study.png}
  \caption{イベントスタディによる受診抑制効果の推移}
  \label{fig:event_study}
  \footnotesize{注: 係数は対数変換された外来件数に対する処置効果を示す。垂直破線は制度導入($k=-1$)を表す。エラーバーは95\%信頼区間。}
\end{figure}

また、再診外来件数・総外来件数への影響は極めて限定的であった。再診外来件数の減少幅は約1.7\%に留まり、またそれに伴って総外来件数への影響も
約4.0\%の減少に留まった。ここから、制度が病院の外来機能全体に与える影響は初診の減少分によってリードされていると推測される。これは再診の性質
やコストの観点から説明される。第一に、一度治療が開始され通院中となった再診患者は同じ病院・医師に対して信頼を置いている場合があり、金銭的コストが
増加したとしても治療の継続を重視する可能性がある。また、治療途中で病院を変更する場合には新たな検査や診断の手間がかかる(スイッチングコスト)ことや、
最低金額が5,000円と初診より低いため、患者がその程度の額にはさほど反応しなかった可能性もある。そのため再診患者の需要は価格に対して非弾力的に
なり、選定療養費の徴収義務化が受診抑制につながらなかったと解釈できる。

\section{病院の特性による反応の違い}
\label{sec:heterogeneity}
異質性分析の結果から、病院の特性によって政策効果がはっきりと異なっていることが示唆された。とりわけ病院の「機能(診療科数の多寡)」と「混雑度(総外来件数の多寡)」の2軸を用いた分析の結果からは、キャパシティが政策効果の大きさを決定していることが示された
(Figure \ref{fig:heterogeneity_matrix})。分析の結果、診療科数が少なく総外来件数も少ない「非多機能・非混雑
(Low Volume / Low Department)」群においては、DID推定の結果初診外来件数が平均的に約21.4\%(係数:$-0.214$)減少していたことがわかった。
これに対し、診療科数が多く外来患者数も多い「多機能・混雑(High Volume / High Department)」群においては、初診外来件数の平均的な減少幅は
約3.1\%(係数:$-0.031$)に留まった。これらの結果から、政策効果の大きさは単に病院の設備や医療の質だけでなく地域内でどの程度患者が多くて患者にとって
代替不可能かということに左右されていると考えられる。また、診療科数が少ない(Low Volume)病院群を利用する外来患者層は比較的軽症であり、他の診療所・
クリニックでも診療の代替が可能であるような医療サービス需要を持っている可能性が患者が多い(High Volume)病院群を利用する外来患者層に比べて高い。
そのため選定療養費徴収義務化という価格シグナルに直面した患者は比較的容易に診療所・クリニックへと受信先を変更しやすく、この点において
軽症患者を診療所・クリニックに誘導し病院の機能分化を達成するという元の政策の目標がある程度達成されているとも言える。

\begin{figure}[H]
  \centering
  \includegraphics[width=1.0\textwidth]{figures/figure4b_matrix_combined.png}
  \caption{病院の機能(診療科数)と混雑度(外来患者数)による異質性分析}
  \label{fig:heterogeneity_matrix}
  \footnotesize{注: サンプルを診療科数の中央値と、処置以前の期間における月間平均総外来件数の中央値で4分割して推定を行った。右端の「High Dep / High Vol」群において効果が最も小さく推定されている。}
\end{figure}

一方で、患者が高度な専門治療やブランド力を求めて受診するような多機能病院(High Volume)においては患者が比較的重篤な疾患を抱えているもしくは
その病院または医師による診療に特にこだわりがある、など強い選好を持っている場合が多いと考えられる。このような状況においては近隣地域に代替可能な
医療機関がないために需要の価格弾力性はきわめて低くなる。このような患者はたとえ7,000円以上の負担増であっても受診行動が変わらない傾向にあるため、
診療科数が多い病院群において政策効果が限定的であったという結果は価格メカニズムを用いた大病院診療の需要抑制策が、真に混雑緩和が必要な大病院に
対しては有効に機能しにくいという限界を示している。

\section{政策的含意}
\label{sec:implication}

以上の分析結果は外来医療の機能分化および日本の医療政策に対して、以下の三点のような重要な示唆を与える。第一に、紹介状のない外来受診に対する選定
療養費の徴収義務化は平均的には本来の目的である病院の機能分化に対して一定貢献していると評価することができる。特に許可病床200床以上の病院の中でも
中規模以下の病院(異質性分析における Low Volume / Low Department)群において平均的には初診外来件数が20\%以上も減少したことは非常に驚くべき
結果であった。これらの病院は選定療養費を継続的に徴収しているという意味でいわゆる”大病院”に分類されるものの、その規模や機能から一部地域のかかりつけ
医機能を果たしている可能性がある。2022年10月の制度改定により、こうした大病院と診療所・クリニックの中間にあるような病院から外来患者が離れ、地域の
診療所・クリニックへと移動して受診するようになったと推測される。これは政策の元の目的に合致しており、本来あるべき成果が出ていると解釈することが
できる。第二に、選定療養費徴収の効果は持続性の面で問題がある可能性が示された。イベントスタディ分析の結果、制度導入直後の1〜3ヶ月間においては
初診患者の大幅な減少が観察されたものの、4ヶ月目以降はその減少効果が限定的となり、下げ止まる傾向が確認された。この結果からは患者が数ヶ月かけて
制度変更に適応するとそれ以降は元の受診行動へと回帰してしまうリスクがあることが導かれる。価格メカニズムを用いて病院の機能分化を達成することを
目指す上では、経済情勢や患者の受診動向に合わせて選定療養費の金額を見直すことなどが重要であると言える。第三に、混雑度の高い病院からの患者の移動に
関しては、選定療養費という価格メカニズムだけでは限界がある可能性が示された。異質性分析の結果から
「多機能・混雑(High Volume / High Department)」群の病院においては最も初診患者の減少割合が小さく、このような病院では医師の負担軽減や
待ち時間短縮が求められているにもかかわらず受診抑制効果が小さかったことは政策的な課題として残る。地域内にはこのような病院を代替可能な医療機関が
存在しないことも多く、最低7,000円程度の追加負担があったとしても患者は受診先を変えたり受診を控えることは少ない。したがってこれらの病院で混雑緩和
を達成するためには、選定療養費という価格メカニズムに頼らない誘導策が必要になると考えられる。具体的にはかかりつけ医機能の法制化・制度化をさらに
進めより厳しいゲートキーパー機能を診療所・クリニックに持たせることが考えられる。たとえば紹介状を持たない患者の物理的な受け入れ制限、すなわち完全
予約制の徹底や初診受付の条件付きでの拒否などが挙げられる。これらの結果から、価格弾力性が低い領域では価格メカニズムではなく制度面の変更による解決
が不可欠であることが示唆される。

\section{本研究の限界と今後の課題}
\label{sec:limitation}

本研究では大規模なレセプトデータを用いて紹介状のない外来受診に対する選定療養費徴収義務化の効果を検証したが、いくつかの限界が存在する。第一に、
使用したデータセットの制約が挙げられる。本研究で用いたデータは協会けんぽのレセプトデータであるが協会けんぽの加入者の性質上すべての社会経済的
背景・年齢層の患者の行動を捉えているわけではない。したがって高齢者が制度導入後にどのように受診行動を変容させたかや、平均的に所得が高い大企業の
従業員とその家族がどのように受診行動を変えたかの識別ができない。第二に、レセプトデータから病院単位で集計し直したデータセットを分析に用いている
ため、患者の傷病、重症度、所得といった属性を考慮できていないという点である。処置群の病院について平均的に初診外来件数が減少しているということは
検証できているものの、その減少した患者が軽症であったのか重症であったのか、等を検証することができておらず個人単位のパネルデータを用いた分析が
必要である。また、受診の追跡が不可能である点も限界として挙げられる。制度導入によって大病院を受診しなくなった患者がその後診療所・クリニックを
受診したのか、あるいは医療機関自体を受診しなかったのかを識別することができておらず、医療機関の機能分化が最終的にどのような形で部分的にでも実現した
のかを追跡することができない。今後の課題として、NDB(レセプト情報・特定健診等情報データベース)のような悉皆データを用いて個人単位で構築したデータ
を用いた分析が求められる。これにより制度導入によってどのような属性を持った患者が受診行動を変えるかについてや、大病院を受診しなくなった後に診療所
・クリニックを受診した割合等も評価することができると考えられる。


%--------------------------------------------------

\chapter{結論}
\label{chap:conclusion}


%==================================================
% 付録
%==================================================
\appendix

\chapter{表・図の補足}
\label{app:figtab}

\section{イベントスタディ係数表、プレトレンドF-test}
\label{app:eventstudy_tables}


\chapter{ロバストネスチェック}
\label{app:robustness}


\chapter{使用データ・コード一覧}
\label{app:data_code_list}


%==================================================
% 参考文献
%==================================================

% \bibliographystyle{jplain} % 和文向けに jplain などに変更可
\bibliographystyle{plain} % 和文向けに jplain などに変更可
\bibliography{thesis}

\end{document}