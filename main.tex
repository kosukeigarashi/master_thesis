%==================================================
% 修士論文_五十嵐康佑
%==================================================
% \documentclass[uplatex,a4paper,11pt]{jsreport}
\documentclass[uplatex,a4paper,11pt]{jsreport}

% ---- Packages ----
% pdfLaTeX 用の設定は使わない(uplatex では不要)
% \usepackage[utf8]{inputenc}
% \usepackage[T1]{fontenc}
% \usepackage{lmodern}

\usepackage{geometry}
\geometry{left=25mm,right=25mm,top=25mm,bottom=25mm}

% ★ LuaLaTeX 用のパッケージは削除する
% \usepackage{luatexja}
% \usepackage[haranoaji]{luatexja-preset} % 日本語フォント設定

\usepackage{setspace}
%\onehalfspacing

\usepackage{amsmath,amssymb,amsthm}
\usepackage{bm}
\usepackage[dvipdfmx]{graphicx}  % uplatex + dvipdfmx 用
\usepackage{booktabs}
\usepackage{here}
\usepackage[dvipdfmx]{hyperref}  % 同上
\usepackage{url}     % URL 表示用

% 1. 章タイトルのデザイン変更(改行させず1行にする・余白を詰める)
\usepackage{titlesec}

% 章タイトルのフォーマット定義
% [hang] : ラベル(第1章)とタイトル(序論)を1行に並べる設定
\titleformat{\chapter}[hang]
  {\normalfont\huge\bfseries} % フォント(太字・巨大)
  {第\thechapter 章}         % ラベルの形式
  {1em}                       % ラベルとタイトルの間の空白
  {}

% 章タイトルの周囲の余白調整
% \titlespacing*{コマンド}{左余白}{上余白}{下余白}
\titlespacing*{\chapter}{0pt}{-20pt}{20pt} 
% ※ 上余白をマイナスにすると、ページのかなり上の方から始まります。
%    デフォルトより詰めたい場合は -10pt や 0pt などで調整してください。

% 2. 本文の行間の調整
% 現在 \onehalfspacing(約1.5倍)が入っていますが、これが「広い」原因です。
% 削除するか、以下のように \setstretch で数値を指定して調整してください。

% \onehalfspacing  % ← これを削除またはコメントアウト
\setstretch{1.1}   % 1.0が標準。0.9〜1.1くらいでお好みに調整してください。

% ---------- 定理環境など(必要なら) ----------
\newtheorem{theorem}{定理}[chapter]
\newtheorem{lemma}{補題}[chapter]
\newtheorem{proposition}{命題}[chapter]
\theoremstyle{definition}
\newtheorem{definition}{定義}[chapter]

% ---------- タイトルなど ----------
\title{紹介状を持たない患者に対する選定療養費徴収義務化が外来患者の受診行動に与えた影響}
\author{五十嵐 康佑}
\date{\today}  % 提出日が決まっていれば固定の日付を書く

%==================================================
\begin{document}
%==================================================

% --- 表紙 ---
\begin{titlepage}
  \begin{center}
    \vspace*{2cm}
    {\Large 修士論文}\\[1cm]
    {\huge \textbf{紹介状を持たない患者に対する選定療養費徴収義務化が外来患者の受診行動に与えた影響}}\\[3cm]

    {\Large 五十嵐 康佑}\\[0.5cm]
    % {\large 指導教員:XXXX 教授}\\[3cm]

    {\large 東京大学大学院経済学研究科経済専攻経済学コース}\\[0.5cm]
    % {\large 2025年度}\\[1cm]
  \end{center}
\end{titlepage}

% --- 前付け(目次・概要など) ---
\pagenumbering{roman} % ローマ数字

\tableofcontents      % 目次
% \listoffigures        % 図目次(不要ならコメントアウト)
% \listoftables         % 表目次(不要ならコメントアウト)

\chapter*{概要}       % 日本語の要約
\addcontentsline{toc}{chapter}{概要} % 目次に載せる


% \chapter*{Abstract}   % 英語要約
% \addcontentsline{toc}{chapter}{Abstract}
% Here you write the English abstract of your master thesis.

\clearpage
\pagenumbering{arabic} % ここから本文はアラビア数字

%==================================================
% 本文
%==================================================

\chapter{序論}
\label{chap:intro}

\section{背景:情報の非対称性と医療資源配分}
\label{sec:information_asymmetry}
日本の医療制度は、国民皆保険制度とフリーアクセス(自由開業・自由標榜・自由受診)を二大支柱として発展し、世界最高水準の平均寿命と
保健衛生水準を達成してきた。特にフリーアクセス制度下では、患者は居住地や症状の軽重に関わらず、自身の選好に基づいて自由に医療機関
を選択することが可能である。この仕組みは、医療へのアクセス障壁を下げることで早期発見・早期治療に寄与してきた一方で、医療
資源配分の非効率性という深刻な副作用をもたらしている。非効率な医療資源配分の例として挙げられるのが、大病院への患者集中問題である。
本来、高度急性期医療や専門医療を担うべき特定機能病院や地域医療支援病院(以下、大病院)に対し、軽微な傷病や慢性疾患の安定期にある
患者が、「念のため」「安心だから」という理由で選好して受診する傾向が常態化している。このような受診行動は、大病院における長時間待機
や医療従事者の疲弊を招くだけでなく、真に高度医療を必要とする重症患者へのリソース投入を阻害する要因となる。経済学の古典的論文である
 Arrow (1963) が指摘するように、医療サービス市場には顕著な「情報の非対称性」と「不確実性」が存在する。患者は自身の症状が軽症か
 重症かを正確に判断する知識を持たず、またどの医療機関が適切な治療を提供できるかについても不完全な情報しか持たない。そのため、患者
 は「大病院である」という規模や設備、評判を、医療の質を保証するシグナルとして利用し、たとえ軽症であっても、効用最大化行動の結果と
 して大病院を選好する。この患者行動は、個人の視点では合理的であっても、社会全体で見れば、高度医療資源の浪費という「負の外部性」を
 引き起こしている。この問題に対処するため、諸外国(イギリスや北欧諸国等)では、家庭医(General Practitioner: GP)登録制度に
 よる厳格なゲートキーピング機能を導入し、専門医への受診にはGPの紹介状を必須とするケースが多い (大久保, 2021)。しかし、かかり
 つけ医の事前登録制度を持たない日本において、同様の強権的な受診制限を導入することは、フリーアクセスの理念と衝突するため困難である。
 そこで日本独自の政策手段として導入されたのが、価格メカニズムを通じた誘導、すなわち「紹介状なしで大病院を受診する患者に対する選定
 療養費の徴収義務化」である。2016年から始まったこの制度は、紹介状を持たずに特定の大病院を受診する患者に対し、保険診療の一部負担金
 とは別に、全額自己負担となる特別料金(定額負担)の支払いを義務付けるものである。これは、患者が直面する金銭的コストを引き上げるこ
 とで、不必要な大病院受診を抑制し、中小病院・診療所への逆紹介を促すことを目的としており、強制的なゲートキーパー制度の代わりに、
 価格インセンティブを用いて患者の自発的な行動変容と医療機関の機能分化を達成しようとする試みである。しかし、この制度が政策意図通り
 に機能しているかは自明ではない。前述の通り、医療における情報の非対称性が存在する場合、患者は価格よりも大病院における診療の質の
 高さを優先する可能性がある。実際、Sugahara (2015) らの事前研究によれば、5,000円から10,000円程度の負担増がなければ軽症患者
 の行動変容は起きにくいと予測されている。日本のような「医療への信頼」が高い社会において、定額の追加費用がどの程度患者の行動変容
 を引き起こすかは、実証的に検証されるべき重要な問いである。
 
 \section{本研究の目的とリサーチクエスチョン}
\label{sec:research_question}
本研究の目的は、紹介状を持たない外来患者に対する選定療養費徴収義務化が、実際にどの程度患者の受診行動を変容させたかを定量的に明らか
にすることである。具体的には、2022年10月に実施された制度改正を自然実験と見なして利用する。この改正によって、特定機能病院・一般
病床200床以上の地域医療支援病院に加えて一般病床200床以上の紹介受診重点医療機関が選定療養費の徴収義務対象となる医療機関となり、
同時に最低徴収金額も増額された。この外生的な制度変更、これまで対象外であった多くの中規模病院が新たに処置の対象となった状況を作り
出しており、因果推論を行う上で理想的な環境を提供している。本研究では、この外生的な制度変更を利用して「選定療養費の徴収義務化および
その拡大は、対象病院における初診外来件数を有意に減少させたか? また、その効果は時間の推移によってどの程度異なるか」というリサーチ
クエスチョンに対する回答を出すことを目的とする。

\section{本研究のアプローチ}
\label{sec:approach}
本研究では、全国健康保険協会(協会けんぽ)の2015年度〜2023年度のレセプトデータを使用した病院×月単位のパネルデータを構築し、差の
差分析およびイベントスタディを用いて政策効果を推定する。制度の対象となる病院(処置群)とそうでない病院(対照群)の分類についてレセプ
トデータ上では各病院が特定機能病院・地域医療支援病院・紹介受診重点医療機関に該当するかどうかは判別不可能であるが、本研究では診療
行為記録内の選定療養費算定の有無を用いることで、各病院がどの時点から制度の対象となったかを特定する。これにより、2022年10月の
ショックを受けた病院群(処置群)と、影響を受けなかった病院群(対照群)を構成し、時間を通じて変化しない病院固有の効果などをコント
ロールした上で、政策効果を推定することを目指す。また、分析のアウトカムとしては、制度の影響を最も直接的に受けると考えられる初診
外来件数に焦点を当てる。Izumida (2004) は自己負担率の引き上げが受診確率を抑制する一方で、その後の治療プロセスには影響しない
ことを示している。これにならい、選定療養費が初診における受診抑制に対して機能したかを見ることで、政策の効果を検証する。

\section{本研究の貢献と意義}
\label{sec:approach}
本研究の貢献は、主に以下の3点である。\\
第一に、選定療養費徴収義務化の効果に関するレセプトデータを利用した全国規模での検証である。本制度に関する既存の実証研究は極めて
少ない。Iba et al. (2023) は2016年の制度導入が紹介率を向上させたことを示したが、その分析対象は茨城県のデータに限られていた。
また、Sugahara (2015) らの研究はアンケート調査に基づくシミュレーションであり、実際の行動変容を捉えたものではない。本研究は、
全国規模のレセプトデータを用いることで2022年の大規模な対象拡大の効果を検証する初の試みの一つである。\\
第二に、医療需要の価格弾力性に関する新たな知見の提示である。RAND HIE (Manning et al., 1987) や Shigeoka (2014) などの
先行研究の多くは、自己負担率の変化が医療需要に与える影響を分析してきた。これに対し、本研究が対象とする選定療養費は、受診一回あたり
に課される定額負担であり、かつ「紹介状がない場合」という条件付きのコストである。このような非線形かつ条件付きの価格設定が、患者の
受診行動にどのような影響を与えるかを示した研究は蓄積が少ない。\\
第三に、今後の医療政策への示唆である。かかりつけ医機能の法制化が議論される中、価格メカニズムによる誘導がどの程度有効かを示すこと
は極めて重要である。もし、現在の価格設定(医科初診の場合・最低金額7,000円)で十分な抑制効果が見られないとすれば、さらなる価格の
引き上げや、より強力なゲートキーピング制度の導入が必要であることが示唆される。逆に、過度な受診抑制が発生していれば、アクセス阻害の
懸念が生じる。本研究の結果は、次期診療報酬改定や医療提供体制の見直しに向けたエビデンスの基盤となりうると考える。

\section{主要な結果の要約}
\label{sec:main_results}
分析の結果、以下の知見が得られた。 第一に、イベントスタディの結果、2022年10月の制度拡大直後においては、対象病院の初診外来件数に
統計的に有意な変化は見られなかった。しかし、制度導入から6ヶ月が経過した時点から、初診件数の有意な減少トレンドが確認された。これは、
紹介受診重点医療機関については新たに認定されてから制度適用までに6か月の経過措置が設けられていることが主な原因と考えられる。//
第二に、DID推定の結果、選定療養費の義務化は平均して初診外来件数を約[ Y ]\%減少させる効果を持っていたことが明らかになった。この結果
は、紹介状を持たない患者の一部が、追加費用を忌避して地域の診療所等へ受診先を変更した(あるいは受診を控えた)ことを示唆しており、制
度が一定の効果を発揮したと評価できる。//
一方で、総外来件数への影響は限定的であった。この結果からは、Izumida (2004) が示したように、負担増は「入り口」のアクセスを抑制
するものの、慢性疾患などで通院を続ける再診患者を含めた病院全体の患者フローを大きく変えるには至っていないということができる。

\section{本論文の構成}
\label{sec:structure}
本論文の構成は以下の通りである。第2章では、日本の医療制度におけるフリーアクセスの課題と、選定療養費制度の変遷について、特にかかり
つけ医機能との関連から詳述する。第3章では、医療需要の価格弾力性および患者の受診選択行動に関する先行研究を概観し、本研究の位置付け
を明確にする。第4章では、使用するレセプトデータの詳細と記述統計を示す。第5章では、DIDおよびイベントスタディを用いた実証モデルの
特定化について説明する。第6章で推定結果を示し、第7章でその解釈と政策的含意、および研究の限界について考察を行う。最後に第8章で結論
を述べる。

%--------------------------------------------------

\chapter{制度背景}
\label{chap:institution_background}
本章では、本研究の分析対象である紹介状なし外来患者に対する選定療養費制度の背景と変遷について記述する。まず、日本の医療制度の特徴で
あるフリーアクセスとその課題について整理し、主要国との比較を交えながら、なぜ価格メカニズムによる介入が必要とされたのかを論じる。
次に、選定療養費制度の導入から2016年の徴収義務化導入、そして2022年の大規模改定に至るまでの経緯を、厚生労働省の資料等に基づき概観
し、本研究が利用する自然実験としての制度的枠組みを明らかにする。最後に、本制度が患者行動に与える影響に関する経済学的な理論仮説を
提示する。

\section{制度的背景:フリーアクセスとゲートキーピング}
\label{sec:freeaccess}
\subsection{フリーアクセスの功罪と資源配分}
\label{sec:fa_merit_demerit}
日本の医療制度は、国民皆保険制度とフリーアクセスを二大支柱として発展してきた。フリーアクセスの下では、患者は自身の傷病の軽重や居住
地に関わらず、全国どの医療機関であっても自由に選択し受診することが可能である。また、原則として医療機関側が正当な理由なく診療を拒否
することは医師法によって禁じられている。この高いアクセス性は、国民の健康維持や疾患の早期発見に大きく寄与し、世界最高水準の平均寿命
と保健衛生水準に貢献してきた。しかし一方で、フリーアクセスは医療資源配分における深刻な非効率性を引き起こすリスクを持つ。非効率性の
例として、医療機関の「機能分化」と患者の「受診行動」のミスマッチが挙げられる。本来、大学病院や特定機能病院・地域医療支援病院などの
大病院は、高度急性期医療や専門的治療を提供するために、高額な医療機器と専門スタッフを集約させた施設である。これら希少なリソースは
重症患者や難治性疾患の治療に優先的に配分されるべきである。しかしフリーアクセスの下では、本来は一次医療(プライマリ・ケア)で対応
可能な軽症患者が大病院を選好して受診する現象が生じやすい。これは、患者が大病院の持つ安心感・ブランドを選好する一方で、自身の受診が
引き起こす混雑やリソースの占有といった社会的費用(負の外部性)を考慮しないために発生する「共有地の悲劇」のような状況と言える。この
結果大病院における長時間待機や、勤務医の過重労働による疲弊が常態化し、医療提供体制の持続可能性が危ぶまれている (土倉, 2015)。

\subsection{ゲートキーパー機能の国際比較}
\label{sec:international_gp}
この問題に関連して、日本ではかかりつけ医のゲートキーパーとしての機能が欠如している。大久保 (2021) が指摘するように、イギリスの
NHS制度やフランス、北欧諸国に見られるように、多くの先進国では家庭医(General Practitioner: GP)登録制度が採用されている。
たとえばイギリスでは救急の場合を除き、患者はまず登録したGPを受診する義務があり、専門医や大病院へのアクセスはGPが必要と判断して発行
する紹介状を持つ患者に厳格に制限されている。これにより、医療の機能分化が制度的に担保されている。それとは対照的に、日本には制度化
されたかかりつけ医の事前登録システムが存在しない。これは患者にとっての利便性を高める反面、専門知識を持たない患者が自己判断で過剰に
高度な医療機関を選択するリスクを持つ。

\subsection{選定療養費制度の経済学的意義}
\label{sec:economical_mean}
このように、フリーアクセスの維持と医療資源の効率的配分はトレードオフの関係にある。大病院への患者集中を解消して病院の機能分化を推進
するためには軽症患者の受診行動を変容させる必要があるものの、フリーアクセスが制限される予定が当面ない日本においては紹介状を持たない
患者の受診を一律に拒否するような強権的な規制を導入することは、政治的・社会的に困難である。このジレンマに対する解決策として導入され
たのが、選定療養費制度を用いた価格メカニズムによる誘導である。選定療養費とは保険診療と保険外診療の併用(混合診療)を例外的に認める
制度の一つである。紹介状のない外来受診に対する選定療養費制度は、紹介状を持たずに大病院のような高度医療機関を受診する行為に対して
定額の追加負担を課すことで患者に金銭的ディスインセンティブを与えるものである。経済学的に解釈すれば、選定療養費は大病院の混雑が引き
起こす外部性を内部化するためのピグー税のようなものと見なすことができる。日本では現在、強制的なゲートキーピング制度を導入できない
制約の中で価格シグナルを通じて患者を診療所・クリニックへと誘導し、間接的に医療提供体制の機能分化を達成しようとしている。

\section{政策介入:選定療養費制度と2022年改革}
\label{sec:sentei_policy}
本節では、紹介状を持たない外来受診に対する選定療養費制度の変遷について概観する。本制度は、当初は医療機関の裁量による任意の徴収で
あったが、段階的な診療報酬改定を経て対象病院の拡大と徴収の義務化および最低徴収金額の引き上げが行われてきた。

\subsection{制度の導入と初期の拡大(2016年以前)}
\label{sec:sentei_policy_early}
2016年度以前においては、一般病床200床以上の病院は地方厚生局への届出を行うことで紹介状を持たない初診患者から任意の金額の選定療養費
を徴収することが認められていた。しかし、厚生労働省 (2015) によればこの段階では徴収はあくまで医療機関の自由裁量であり、地域住民へ
の配慮や患者減少への懸念から実際に徴収を行う病院は限定的であった。また、徴収を行っている場合でもその金額は平均して2,000円台にとど
まり、受診抑制効果は限定的であった。

\subsection{徴収の義務化:2016年改革}
\label{sec:sentei_policy_2016}
2016年4月の診療報酬改定時には、特定機能病院および一般病床500床以上の地域医療支援病院を対象として紹介状なしの外来受診に対する選定
療養費の徴収が義務化された。対象となった医療機関について、「特定機能病院」とは、高度の医療を提供するとともに、高度の医療技術の開発
・評価、および医療に関する研修を行う能力を有する病院として厚生労働大臣の承認を受けた施設(主に大学病院等)である。一方、「地域医療
支援病院」とは、地域医療の第一線を担うかかりつけ医(診療所・中小病院)を支援する能力を有する病院として都道府県知事の承認を受けた
施設であり、救急医療や共同利用機能の中核を担う。義務化に伴い、最低徴収金額は初診時5,000円(歯科は3,000円)、再診時2,500円
(歯科は1,500円)と設定された。これは、当時のコンジョイント分析等の結果(河口・菅原 2015)を踏まえ、患者の行動変容を促すために
必要な水準として設定されたものである。

\subsection{2022年の制度改革:対象の大幅拡大}
\label{sec:sentei_policy_2022}
本研究が分析の焦点とする2022年10月の制度改革は、以下の3点の大きな特徴を持つ。第一に、義務化対象病院の拡大が挙げられる。従来の特定
機能病院・地域医療支援病院に加え、新たに制度化された紹介受診重点医療機関(一般病床200床以上)が選定療養費徴収義務化の対象となった。
 厚生労働省 (2020) によれば、紹介受診重点医療機関は外来機能報告制度に基づき手術、化学療法、放射線治療など医療資源を重点的に活用
 する外来医療を地域の中で基幹的に担う病院として都道府県が公表したものである。この新たな枠組みの導入により、これまで制度対象外で
 あった200床〜400床規模を含む多くの中規模病院がその機能実績に基づいて新たに選定療養費の徴収義務対象に組み込まれることとなった。
 また、紹介受診重点医療機関の公表時期は2023年3月〜8月頃と都道府県によって異なり、公表から6ヶ月間の経過措置期間が設けられている
 ため実際に徴収義務化が適用された時期には地域差が存在する点に留意が必要である。第二に、定額負担額の引き上げである。初診時の最低
 徴収金額は5,000円から7,000円へ、再診時は2,500円から3,000円へと増額された。これは消費増税等の影響を除けば、患者の実質負担を
 約1.4倍に引き上げる強力な価格介入である。第三に、保険給付範囲からの控除による病院に対する徴収強制力の強化である。これは制度の
 実効性の観点から非常に重要な変更点と言える。改定以前は選定療養費の徴収は義務であるとされつつも、徴収しない場合の罰則規定が弱く
 病院側の裁量が残されていた。しかし、2022年10月の改定以降は義務化対象病院が紹介状なしの患者を診療した(初診)場合、初診料等から一定
 額(200点=2,000円)が保険給付から控除される仕組みとなった。 具体的には、初診料(288点=2,880円)を算定する際、病院が患者から
 選定療養費を徴収しなければ、そこから200点=2,000円が差し引かれ、残りの880円しか保険請求できない。すなわち、病院は本来得られる
 はずだった診療報酬を失うこととなる。これにより、病院側にとって選定療養費の徴収は経営上の損失回避のために必要となった。なお、制度
 の公平性を担保するため救急搬送や公費負担医療の対象者、HIV等の特定の疾病についてはこれらの徴収義務の対象外とされている。

\section{理論的枠組みと仮説}
\label{sec:theory_hypothesis}
本節では、上述した制度的背景に基づき、選定療養費の導入および拡大が患者行動に与える影響について、経済学的な理論仮説を提示する。

\subsection{情報の非対称性と不確実性}
\label{sec:asymmetry_uncertainty}
医療サービス市場における消費者の意思決定を考える上で、Arrow (1963) の情報の非対称性および不確実性は不可欠な視点である。医療サー
ビスの特性上、患者は自身の症状の深刻度やどの医療機関が適切な治療を提供できるかについて医師ほど完全な情報を持っていない。そのため、
患者は「大病院である」という規模や設備、評判を医療の質を示すシグナルとして利用し、たとえ軽症であっても安心を求めて大病院を選好する
傾向がある。

\subsection{時間価格と金銭価格のトレードオフ}
\label{sec:time_money_tradeoff}
Acton (1973) が示したように、保険等によるカバーがあり、医療サービスの金銭価格が低い状況下では、移動時間や待ち時間のような時間価格
が需要を決定する主要因となる。通常、大病院は待ち時間が長く時間価格が高い。それでも患者が大病院を選好する傾向にあるのは、期待される
医療の質による効用が、時間コストを上回るからである。標準的な消費者理論によれば、医療サービスの需要は価格(自己負担額)の減少関数で
あると考えられるが、選定療養費の徴収義務化は紹介状を持たずに大病院を受診する場合の実質価格、すなわちシャドウ・プライスを上昇させる。
これによって代替財である診療所・クリニックへの受診が相対的に安価になるため、代替効果によって軽症患者の受診先は大病院から診療所・
クリニックへとシフトすることが予測される。

\subsection{仮説の提示}
\label{sec:hypothesis}
以上より、合理的な患者個人は、制度の施行後にはまず近隣の診療所を受診し、必要に応じて紹介状を取得してから大病院を受診するようになる。
また、軽微な症状であれば受診自体を控える可能性もある。いずれの場合においても、紹介状を持たない大病院の初診件数は減少するはずである。
しかし一方で、前述の情報の非対称性が、この価格メカニズムの機能を阻害する可能性も存在する。患者は医師に比べて自身の症状の深刻度を
正確に判断することができないため、たとえ7,000円の追加負担があったとしても、万が一重病であるリスクを大きく見積もり、設備や評判の
整った大病院を強く選好し続ける可能性がある。本研究では、これらの理論的背景を踏まえ、以下の仮説を検証する。\\
仮説1: 選定療養費の徴収義務化は、対象病院における初診外来件数を有意に減少させる。\\
仮説2: その減少効果は、制度適用や患者の学習期間を待つ必要があるため即時的ではなくある程度のタイムラグの後に現れる。


%--------------------------------------------------

\chapter{先行研究}
\label{chap:literature}
本章では、本研究のテーマの基盤となるような先行研究を整理する。 本研究の問いは、経済学的には情報の非対称性、医療サービス需要の価格
弾力性、患者の受診行動と病院選択、そして選定療養費の評価といったトピックに関連する。以下、それぞれのトピックについて既存研究を概観
する。

\section{情報の非対称性と医療サービス需要の価格弾力性}
\label{sec:asymmetry_lit}
\subsection{理論的背景:情報の非対称性と不確実性}
\label{sec:asymmetry_lit_theory}
医療サービス需要を経済学的に分析する際の出発点は、Arrow \cite{Arrow1963} による古典的議論である。Arrowは、医療市場における情報の非対称性
と不確実性の存在により、医療サービスについては通常の市場メカニズムが機能しない可能性を指摘した。患者は自身の健康状態や必要な治療に
ついて医師ほどは十分な知識を持たないため、価格のみならず、医師への信頼や病院の評判を品質のシグナルとして行動する。情報の非対称性に
よって価格メカニズム、すなわち選定療養費による誘導が必ずしも合理的になされない可能性があり、本研究の分析においても重要な理論的背景
となる。また、Grossman \cite{Grossman1972} は医療需要を健康資本への投資としてモデル化し、医療サービスが消費財であると同時に投資財としての側面
を持つことを理論化した。

\subsection{海外における実験的アプローチ:RAND HIEとOregon HIE}
\label{sec:asymmetry_lit_rand_oregon}
医療需要が価格(自己負担)に対してどのように反応するかについては、米国を中心に多くの実証研究が蓄積されている。 Manning et al. \cite{Manning1987} 
ではRAND HIEによって、RCTにより自己負担率の上昇が医療支出を有意に抑制することを実証し、外来医療の価格弾力性は約-0.2であると推計
した。この結果は、医療需要は金銭的インセンティブによってある程度制御可能であることを示唆している。 より近年では、
Finkelstein et al. \cite{Finkelstein2012} がオレゴン州のMedicaid抽選を自然実験として利用した研究(Oregon HIE)を行い、保険適用が外来受診
や処方薬の利用を有意に増加させることを示した。しかし一方で、Brot-Goldberg et al. \cite{BrotGoldberg2017} は、高額な免責金額(Deductible)を
課された患者が不要な医療だけでなく必要な医療まで一律に減らしてしまう現象を報告しており、医療に関して患者が選別行動をとることの難し
さを示している。

\subsection{日本における実証的証拠}
\label{sec:asymmetry_lit_japan}
日本の公的データを用いた研究においても、医療需要が価格に対して有意に反応することが確認されている。 
Bhattacharya et al. \cite{Bhattacharya1996} は、1990年の患者調査を用いて外来受診行動を分析し、日本においても医療需要が価格に対して右下がりの
需要曲線(弾力性は-0.12〜-0.54程度。)を持つことを示した。 また、Shigeoka \cite{Shigeoka2014} は、70歳時点での窓口負担率の低下(3割から
1割等)を利用した回帰不連続デザインを用い、高齢者の外来受診行動が価格に敏感に反応することを示している(弾力性は-0.2程度。)。
また、Iizuka and Shigeoka \cite{IizukaShigeoka2022} は小児医療費助成の分析において、数百円程度の少額の自己負担であっても受診行動に有意な抑制効果
を持つことを示しており、日本の患者が価格シグナルに対して敏感であることを裏付けている。湯田 \cite{Yuda2023} はこれらの国内研究を包括的に
サーベイし、日本の医療需要の価格弾力性が総じて低いものの有意に負であることを確認している。Izumida \cite{Izumida2004} は1997年の自己負担率
の2割引き上げの影響をレセプトデータを用いて分析し、患者の自己負担率増は医療機関への受診確率を有意に引き下げるものの、一度受診した
後の受診形態の選択には有意な影響を与えないことを明らかにした。これは選定療養費のような最初の受診時の負担が、初診アクセスの抑制には
有効であるが、その後の受診形態には影響しにくいことを示唆しており、本研究の仮説構築の観点から重要である。

\section{患者の受診行動と病院選択}
\label{sec:hospital_choice_lit}
選定療養費制度は、患者を大病院から診療所・クリニックへと誘導する(病院の機能分化)ことを目的としている。したがって、患者がどの
ような要因で病院を選択しているかに関する先行研究について概観する。

\subsection{時間価格と質のトレードオフ}
\label{sec:time_quality_tradeoff}
患者の病院選択において、距離・時間などアクセスにかかるコストは非常に重要な要因である。Acton \cite{Acton1973} は、医療サービスの金銭価格が
低い場合には金銭価格に代わって時間価格が需要を決定する主要因になることを理論的・実証的に示した。これは、日本のフリーアクセス制度下
においては待ち時間の長い大病院が敬遠される、あるいは逆に近隣の診療所が選好されることを説明する。 一方で、患者は大病院で受けられる
医療サービスの質を選好する。Tay \cite{Tay2003} は離散選択モデルを用い、患者が距離的なコストと医療の質をトレードオフにかけて病院を選択し
ていることを示した。すなわち、患者はある程度の時間・金銭コストを支払ってでも質の高い医療を受けたいと考える傾向にあると言える。

\subsection{本邦における患者選好とバイパス行動}
\label{sec:hospital_choice}
日本における患者の受療行動については、Yamamoto \cite{Yamamoto2002} が国保レセプトデータを用いて分析し、患者が診療所・中小病院・大病院を症状
に応じて使い分けている実態を示した。しかし、患者が大病院を志向する背景には、単なる症状の重さだけでなく、大病院への信頼や設備への
期待があることがわかっている。Yagi \cite{Yagi2008} は、患者への定性調査(KJ法)に基づき、医療機関選択において立地や利便性のみならず、医療
機器の充実や評判といった情報が重視されていることを明らかにしている。 これらの研究から、患者は強い質への選好を持って大病院を選ぶ傾向
にあることが読み取れるため、選定療養費がどの程度行動を変容させるかの度合いが重要となる。

\section{選定療養費制度の評価と課題}
\label{sec:sentei_lit}
\subsection{表明選好法による制度導入の事前評価}
\label{sec:sentei_lit_sp}
紹介状のない外来受診に対する選定療養費の徴収義務化が患者の受診行動に与える影響については、制度導入前後においていくつかの予測研究が
行われている。 Sugahara \cite{Sugahara2015Rep} および Kawaguchi \& Sugahara \cite{KawaguchiSugahara2015} は、コンジョイント分析(表明選好法)を用いたアンケート
調査により、定額自己負担の導入が受診行動に与える影響をシミュレーションした。これらによれば、軽症患者の場合、5,000円から10,000円
程度の定額負担を設定することで、大病院から診療所への誘導が期待できるとされる。一方で、重症患者の受診行動は価格に対して非弾力的で
あることも示されている。これらは2016年の制度導入(医科初診時に最低5,000円徴収)の根拠となった。

\subsection{顕示選好データによる事後検証の現状}
\label{sec:sentei_lit_rp}
しかし、実際に制度が義務化された後の効果を大規模な顕示選好データを用いて検証した研究はほとんど見当たらない。例外として、
Iba et al. \cite{Iba2023} が挙げられる。これは茨城県のDPCデータを用いて2016年の制度導入の影響を分析したもので、
Controlled Interrupted Time-Series分析の結果、義務化対象となった病院では紹介率が約5\%ポイント上昇した一方、対象外の病院では
変化が見られなかったことを報告している。これは、制度導入によって病院の機能分化が一定程度進行したことを示す実証結果であり、
Kobayashi et al. \cite{Kobayashi2019} などの記述的な分析においても、制度導入後に大病院の初診患者数が減少傾向にあることが報告されている。

\subsection{先行研究の限界と本研究の貢献}
\label{sec:sentei_lit_limit}
紹介状のない外来受診に対する選定療養費の徴収義務化に関する先行研究はアンケートに基づく事前予測(Sugahara \cite{Sugahara2015Rep})や、特定の地域・
時点に限られた分析(Iba et al. \cite{Iba2023})に留まっている。特に、2022年10月の制度拡大は保険給付範囲からの控除によって病院に対する
徴収強制力が強化されたという点で非常に重要であるにもかかわらず、これについて全国規模のレセプトデータを用いて分析した研究は、筆者が
知る限り存在しない。 本研究は2022年の制度拡大という外生的な制度変更を利用し、全国の病院パネルデータを用いた差の差分析を行うこと
で、選定療養費の徴収義務化が患者の受診行動に与えた因果効果を検証するものである。これにより、Izumida \cite{Izumida2004} が示した受診確率の
抑制効果が、定額負担という形態においても成立するか、また Iba et al. \cite{Iba2023} で見られた効果が平均的には全国でも観察されるかを
明らかにする。

%--------------------------------------------------

\chapter{データ}
\label{chap:data}
本章では分析に用いるデータセットの構築手順、変数の定義、および記述統計について述べる。本研究では全国規模のレセプトデータを用いて制度変更の影響を
受けた病院(処置群)と受けなかった病院(対照群)を識別し、病院×月単位のパネルデータを構築する。

\section{データソース}
\label{sec:data_source}
本研究では、全国健康保険協会(以下、協会けんぽ)が管理・保有する「診療報酬明細書(レセプト)データ」を使用する。データの利用にあたっては、個人の
特定が不可能な形で匿名化された個票データの提供を受けた。

\subsection{データの特性と限界}
\label{sec:data_limitations}
協会けんぽは主に中小企業の従業員(被保険者)とその家族(被扶養者)を主な加入者とする日本最大規模の医療保険者であり、その加入者数は約4,000万人に
達する。しかし、NDB(レセプト情報・特定健診等情報データベース)のような日本全体の悉皆データと比較した場合本データには以下の限界が存在する。\\
第一に、サンプルに高齢者が含まれていないという点が挙げられる。協会けんぽの被保険者は就労者とその家族であり、75歳以上の後期高齢者は後期高齢者医療
制度へ移行するため原則としてデータに含まれない。また、65歳から74歳の前期高齢者についても定年退職に伴い国民健康保険へ移行するケースが多く、捕捉率
が低下する傾向にある。したがって、本研究の分析対象は、主に現役世代とその家族に限定される。\\
第二に、職業等に偏りがある。協会けんぽの加入者は中小企業の従業員が中心であり、大企業の従業員(組合健保)や公務員(共済組合)、自営業者(国保)は
含まれない。\\
第三に、追跡の打ち切りの問題が挙げられる。就職、転職、退職等によって被保険者資格を喪失した場合その時点でデータの追跡が途切れることとなる。

\subsection{本データを使用する妥当性}
\label{sec:data_validity}
上記のような限界はあるものの、本研究において協会けんぽデータを使用することは、以下の理由から妥当であると考えられる。 まず、約4,000万人というサン
プルサイズは統計的な検出力の観点から十分であると考えられる。 次に、政策評価の観点からの重要性である。選定療養費(定額負担)のような価格インセン
ティブに対しては、医療費の自己負担割合が1割または2割と低い高齢者層よりも、原則3割負担に直面している現役世代の方がより敏感に反応すると考えられる。
したがって協会けんぽデータは、対象制度の効果を検証する上で十分適切なデータと考える。

\section{サンプル構築}
\label{sec:sample}
本研究では、以下の手順に従って分析用データセットを構築した。

\subsection{集計単位}
\label{sec:aggregation_unit}
利用したデータは2015年4月から2024年3月までの期間における協会けんぽの全レセプトデータである。その中からまずレセプト種別が医科、かつ入外区分が
外来であるレセプトのみを抽出した。本研究は外来受診に焦点を当てているため歯科、調剤、および入院レセプトは分析対象から除外した。 次に、抽出された
レセプトデータを、医療機関コードおよび診療年月でグループ化して集計し、病院×月単位のパネルデータを作成した。

\subsection{除外基準}
\label{sec:exclusion_criteria}
分析にあたっては、以下の基準に基づきデータセットの絞り込みを行った。\\
第一に施設形態による絞り込みを行った。2022年10月から選定療養費徴収義務化の対象となる病院は、紹介受診重点医療機関を含む許可病床200床以上の病院
となった。協会けんぽのデータには病院と診療所を厳密に区別するフラグが存在しないため、本研究では許可病床200床以上と推定される医療機関のみを分析
対象として抽出し、それ以外の小規模病院および診療所はデータセットから除外した。\\
次に処置群と対照群の定義に基づき、サンプルの絞り込みを行った。 処置群は、2022年10月の制度改定以降2024年3月まで継続的に選定療養費を徴収した
記録のある病院群と定義した。 対照群は、許可病床200床以上と推定される病院のうち、処置群に入っていない病院群と定義した。

\subsection{最終的な分析サンプル}
\label{sec:final_sample}
以上のプロセスを経て構築された最終的な分析サンプルは、以下の通りである。

\section{変数定義}
\label{sec:variables}
以下では、構築した分析用データセットに含まれる主要な変数の定義について述べる。本データセットは、観測単位を病院 $i$ 、診療年月 $t$ とする
パネルデータである。病院ID(hospital\_id)については、「レセプト」テーブルの都道府県コード、点数表コード、医療機関コードの組み合わせにより
作成した。これは協会けんぽのガイドラインに沿った作成方法である。

\subsection{被説明変数}
\label{sec:dependent_variables}
本研究では、選定療養費の導入および拡大が患者の受診行動に与える影響を検証するため、以下の4つの変数を被説明変数として設定した。外来件数に
ついては分析では自然対数変換を施した値を用いた。\\
第一の被説明変数は対数初診外来件数(log\_shoshin\_outpatients)である。これは、「レセプト診療行為」テーブルに初診料を表す診療行為コードが
記録されているレセプトの枚数を病院・年月ごとにグループ化して集計して shoshin\_outpatients とし、その自然対数をとったものである。選定
療養費の対象となるのは初診患者であるため、この変数を主要な被説明変数とした。\\
第二の被説明変数は対数総外来件数 (log\_total\_outpatients) である。これは、「レセプト」テーブルの「入外区分」列の値により外来レセプトの
総枚数 (total\_outpatients) をカウントし、対数変換したものである。初診だけでなく再診を含めた外来全体への影響を確認するために使用する。\\
さらに機能分化の進展を評価する第三の被説明変数として、紹介率 (referral\_ratio) も作成した。これは、「レセプト診療行為」テーブルに
「初診料(文書による紹介がない患者)」を表す診療行為コードが記録されているレセプトの枚数を病院・年月ごとにグループ化して集計して「紹介状なし
初診外来件数 (no\_referral\_outpatients)」とする。次に、初診外来件数 (shoshin\_outpatients) からこの紹介状なし初診外来件数を差し
引くことで「紹介状あり初診外来患者数 (referral\_outpatients)」を推定し、これを初診外来件数で除すことで紹介率を算出した。\\
また、第四の被説明変数として紹介状あり初診外来患者数の対数値 (log\_referral\_outpatients) も用いる。

\subsection{説明変数}
\label{sec:explanatory_variables}
差の差分析およびイベントスタディのために、つぎの変数を定義する。処置群ダミー (treated) は、選定療養費の徴収義務化対象病院を識別
するダミー変数である。具体的には2022年10月から2024年3月までの全期間において、毎月継続して「レセプト診療行為」テーブルに「初診料
(文書による紹介がない患者)」を表す診療行為コードが記録されている病院を $1$、そうでない病院を $0$ と定義する。また、イベント
スタディのために経過時間変数を作成した。event\_timeは2022年10月を基準時点($0$)として経過月数を表す変数である。

\subsection{共変量と固定効果}
\label{sec:covariates_fixed_effects}
病院の規模や機能の代理変数として診療科数 (department\_count) を使用する。これは、各病院について「レセプト」テーブルの診療科
コード(診療科1〜3)に記録されたユニークな診療科の数をカウントしたものである。 また、地域特有のショックを制御するため病院所在地の
郵便番号の上2桁(postal\_2)を用いて地域を識別し、地域×月固定効果としてモデルに組み込む。

\section{記述統計}
\label{sec:descriptive}
本節では分析に用いるデータセットの基本統計量を示す。

\subsection{要約統計量}
\label{sec:summary_statistics}

%--------------------------------------------------

\chapter{実証戦略}
\label{chap:empirical_strategy}

\section{DID}
\label{sec:did}


\section{イベントスタディ}
\label{sec:eventstudy}


\section{識別の仮定}
\label{sec:identification}


%--------------------------------------------------

\chapter{分析結果}
\label{chap:results}

\section{イベントスタディ結果}
\label{sec:results_eventstudy}


\section{DID結果}
\label{sec:results_did}


\section{異質性分析}
\label{sec:heterogeneity}


\section{ロバストネスチェック}
\label{sec:robustness}


%--------------------------------------------------

\chapter{考察}
\label{chap:discussion}

\section{メカニズムの解釈}
\label{sec:mechanism}


\section{政策的含意}
\label{sec:policy_implication}


\section{限界}
\label{sec:limitations}


%--------------------------------------------------

\chapter{結論}
\label{chap:conclusion}


%==================================================
% 付録
%==================================================
\appendix

\chapter{表・図の補足}
\label{app:figtab}

\section{イベントスタディ係数表、プレトレンドF-test}
\label{app:eventstudy_tables}


\chapter{ロバストネスチェック}
\label{app:robustness}


\chapter{使用データ・コード一覧}
\label{app:data_code_list}


%==================================================
% 参考文献
%==================================================

\bibliographystyle{jplain} % 和文向けに jplain などに変更可
\bibliography{thesis}

\end{document}