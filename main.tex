%==================================================
% 修士論文 基本テンプレート (thesis.tex の例)
%==================================================
\documentclass[a4paper,11pt]{report} % 必要なら jsreport, bxjsreport などに変更

% ---------- パッケージ類 ----------
\usepackage[utf8]{inputenc}   % 文字コード(LuaLaTeX/XeLaTeXなら通常不要)
\usepackage[T1]{fontenc}
\usepackage{lmodern}          % 欧文フォント
\usepackage{geometry}         % 余白設定
\geometry{left=25mm,right=25mm,top=25mm,bottom=25mm}

\usepackage{setspace}         % 行間
\onehalfspacing               % 1.5倍行間

\usepackage{amsmath,amssymb,amsthm} % 数式関連
\usepackage{bm}               % 太字ベクトルなど
\usepackage{graphicx}         % 図
\usepackage{booktabs}         % きれいな表
\usepackage{caption}
\usepackage{subcaption}       % 図の(a)(b)など
\usepackage{here}             % [H]指定で図表を「ここ」に出す
\usepackage{hyperref}         % 目次や参照をハイパーリンクに

% ---------- 和文向けに使う場合の例 ----------
% pLaTeX / upLaTeX を使うなら inputenc, fontenc, lmodern は不要なことが多いです。
% ドライバや環境に応じて調整してください。
% jsreport を使う場合:
% \documentclass[uplatex,a4paper,11pt]{jsreport}
% などに差し替える。

% ---------- 定理環境など(必要なら) ----------
\newtheorem{theorem}{定理}[chapter]
\newtheorem{lemma}{補題}[chapter]
\newtheorem{proposition}{命題}[chapter]
\theoremstyle{definition}
\newtheorem{definition}{定義}[chapter]

% ---------- タイトルなど ----------
\title{修士論文タイトル(仮)}
\author{氏名}
\date{\today}  % 提出日が決まっていれば固定の日付を書く

%==================================================
\begin{document}
%==================================================

% --- 表紙 ---
\begin{titlepage}
  \begin{center}
    \vspace*{2cm}
    {\Large 修士論文}\\[1cm]
    {\huge \textbf{修士論文タイトル(仮)}}\\[3cm]

    {\Large 氏名}\\[0.5cm]
    {\large 指導教員:XXXX 教授}\\[3cm]

    {\large ○○大学大学院 △△研究科}\\[0.5cm]
    {\large 2025年度}\\[1cm]
  \end{center}
\end{titlepage}

% --- 前付け(目次・概要など) ---
\pagenumbering{roman} % ローマ数字

\tableofcontents      % 目次
\listoffigures        % 図目次(不要ならコメントアウト)
\listoftables         % 表目次(不要ならコメントアウト)

\chapter*{概要}       % 日本語の要約
\addcontentsline{toc}{chapter}{概要} % 目次に載せる
ここに日本語の概要を書く。研究の背景、目的、方法、主要な結果、結論を簡潔にまとめる。

\chapter*{Abstract}   % 英語要約
\addcontentsline{toc}{chapter}{Abstract}
Here you write the English abstract of your master thesis.

\clearpage
\pagenumbering{arabic} % ここから本文はアラビア数字

%==================================================
% 本文
%==================================================

\chapter{序論}
\label{chap:intro}
本章では、研究の背景、問題意識、目的、貢献などを述べる。

\section{研究背景}
テキストテキストテキスト。

\section{研究目的}
テキストテキストテキスト。

\section{本論文の構成}
本論文の構成を説明する。

%--------------------------------------------------

\chapter{関連研究}
\label{chap:literature}
関連する先行研究のサーベイを行う。

%--------------------------------------------------

\chapter{理論モデル / 分析枠組み}
\label{chap:model}
ここに理論モデルや分析の枠組みを書く。

\section{モデルの設定}
数式環境の例:
\begin{equation}
  y_{it} = \alpha + \beta x_{it} + \varepsilon_{it}.
\end{equation}

%--------------------------------------------------

\chapter{データと実証方法}
\label{chap:data_method}
データの説明、推定方法、識別戦略などを書く。

\section{データ}
図・表の例:
\begin{figure}[H]
  \centering
  \includegraphics[width=0.7\textwidth]{figure_example} % figure_example.pdf など
  \caption{サンプル図のキャプション}
  \label{fig:sample}
\end{figure}

\begin{table}[H]
  \centering
  \caption{サンプル表のキャプション}
  \label{tab:sample}
  \begin{tabular}{lrr}
    \toprule
    項目 & 値1 & 値2 \\
    \midrule
    A    & 1.0 & 2.0 \\
    B    & 3.0 & 4.0 \\
    \bottomrule
  \end{tabular}
\end{table}

\section{推定手法}
DID、イベントスタディなどの推定式を書く。

%--------------------------------------------------

\chapter{実証結果}
\label{chap:results}
推定結果の提示と解釈を行う。

\section{ベースライン結果}
テキストテキストテキスト。

\section{ロバストネスチェック}
感度分析やロバストネスチェックを記述。

%--------------------------------------------------

\chapter{結論}
\label{chap:conclusion}
本研究の結論、政策的含意、今後の課題などをまとめる。

%==================================================
% 付録
%==================================================
\appendix

\chapter{追加の表・図}
追加の結果、感度分析など。

\chapter{証明}
理論部分の証明など。

%==================================================
% 参考文献
%==================================================

\bibliographystyle{plain} % 和文向けに jplain などに変更可
\bibliography{thesis}     % thesis.bib を別途用意する

\end{document}