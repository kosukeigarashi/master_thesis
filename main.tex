%==================================================
% 修士論文_五十嵐康佑
%==================================================
\documentclass[uplatex,a4paper,11pt]{jsreport}

% ---- Packages ----
% ---- 欧文 pdfLaTeX 向けの設定は消す or コメントアウト ----
% \usepackage[utf8]{inputenc}
% \usepackage[T1]{fontenc}
% \usepackage{lmodern}

\usepackage{geometry}
\geometry{left=25mm,right=25mm,top=25mm,bottom=25mm}

\usepackage{setspace}
\onehalfspacing

\usepackage{amsmath,amssymb,amsthm}
\usepackage{bm}
\usepackage[dvipdfmx]{graphicx}  % ここポイント
\usepackage{booktabs}
\usepackage{here}
\usepackage[dvipdfmx]{hyperref}  % ここも dvipdfmx を指定

% ---------- 和文向けに使う場合の例 ----------
% pLaTeX / upLaTeX を使うなら inputenc, fontenc, lmodern は不要なことが多いです。
% ドライバや環境に応じて調整してください。
% jsreport を使う場合:
% \documentclass[uplatex,a4paper,11pt]{jsreport}
% などに差し替える。
% report を使う場合(英文向け):
% \documentclass[a4paper,11pt]{report} 
% などに差し替える。

% ---------- 定理環境など(必要なら) ----------
\newtheorem{theorem}{定理}[chapter]
\newtheorem{lemma}{補題}[chapter]
\newtheorem{proposition}{命題}[chapter]
\theoremstyle{definition}
\newtheorem{definition}{定義}[chapter]

% ---------- タイトルなど ----------
\title{紹介状を持たない患者に対する選定療養費徴収義務化が外来患者の受診行動に与えた影響}
\author{五十嵐 康佑}
\date{\today}  % 提出日が決まっていれば固定の日付を書く

%==================================================
\begin{document}
%==================================================

% --- 表紙 ---
\begin{titlepage}
  \begin{center}
    \vspace*{2cm}
    {\Large 修士論文}\\[1cm]
    {\huge \textbf{紹介状を持たない患者に対する選定療養費徴収義務化が外来患者の受診行動に与えた影響}}\\[3cm]

    {\Large 五十嵐 康佑}\\[0.5cm]
    % {\large 指導教員:XXXX 教授}\\[3cm]

    {\large 東京大学大学院経済学研究科経済専攻経済学コース}\\[0.5cm]
    % {\large 2025年度}\\[1cm]
  \end{center}
\end{titlepage}

% --- 前付け(目次・概要など) ---
\pagenumbering{roman} % ローマ数字

\tableofcontents      % 目次
\listoffigures        % 図目次(不要ならコメントアウト)
\listoftables         % 表目次(不要ならコメントアウト)

\chapter*{概要}       % 日本語の要約
\addcontentsline{toc}{chapter}{概要} % 目次に載せる
ここに日本語の概要を書く。研究の背景、目的、方法、主要な結果、結論を簡潔にまとめる。

% \chapter*{Abstract}   % 英語要約
% \addcontentsline{toc}{chapter}{Abstract}
% Here you write the English abstract of your master thesis.

\clearpage
\pagenumbering{arabic} % ここから本文はアラビア数字

%==================================================
% 本文
%==================================================

\chapter{序論}
\label{chap:intro}

\section{動機}
\label{sec:motivation}
ここに研究の動機を書く。

\section{問題}
\label{sec:problem}
ここに問題意識・政策的/学術的課題を書く。

\section{本研究のアプローチ}
\label{sec:approach}
研究デザイン、データ、識別戦略の概要を書く。

\section{主要な結果の要約}
\label{sec:mainfindings}
主要な実証結果を簡潔にまとめる。

\section{貢献}
\label{sec:contribution}
先行研究に対する位置づけと貢献を整理する。

%--------------------------------------------------

\chapter{制度背景}
\label{chap:institution_background}

\section{日本の医療制度とフリーアクセス}
\label{sec:freeaccess}
日本の医療提供体制とフリーアクセスの基本的特徴を書く。

\section{選定療養費制度の変遷}
\label{sec:sentei_history}
制度制定から改定の経緯の整理を書く。

\section{理論的枠組みと仮説}
\label{sec:theory_hypothesis}
制度が受診行動に与えうる影響の理論整理と仮説を提示する。

%--------------------------------------------------

\chapter{先行研究(Literature Review)}
\label{chap:literature}

\section{医療需要の価格弾力性に関する研究}
\label{sec:elasticity_lit}
関連する理論・実証の主要知見を整理する。

\section{日本の選定療養費に関する研究}
\label{sec:sentei_lit}
国内の制度研究・実証研究をまとめ、本研究との差分を明確化する。

%--------------------------------------------------

\chapter{データ}
\label{chap:data}

\section{データソース}
\label{sec:data_source}
データ出所、期間、観測単位、カバレッジを書く。

\section{サンプル構築}
\label{sec:sample}
除外基準、集計単位、パネルの作り方を書く。

\section{変数定義}
\label{sec:variables}
主要アウトカム、処置変数、統制変数の定義を書く。

\section{記述統計}
\label{sec:descriptive}
主要変数の記述統計・分布を提示する。

%--------------------------------------------------

\chapter{実証戦略}
\label{chap:empirical_strategy}

\section{DID}
\label{sec:did}
DID の基本仕様を提示する。
\begin{equation}
  y_{it} = \alpha + \beta \,\text{Post}_t \times \text{Treat}_i
  + \gamma_i + \delta_t + X_{it}'\theta + \varepsilon_{it}.
\end{equation}

\section{イベントスタディ}
\label{sec:eventstudy}
イベントスタディ仕様と推定の要点を書く。

\section{識別の仮定}
\label{sec:identification}
パラレルトレンド等の識別仮定と検証方法を書く。

%--------------------------------------------------

\chapter{分析結果}
\label{chap:results}

\section{イベントスタディ結果}
\label{sec:results_eventstudy}
イベントスタディの図示と解釈を書く。

\section{DID結果}
\label{sec:results_did}
ベースラインDIDの推定結果と解釈を書く。

\section{異質性分析(Heterogeneity)}
\label{sec:heterogeneity}
病院属性・地域・患者属性などでの効果の違いを示す。

\section{ロバストネスチェック}
\label{sec:robustness}
仕様変更、サンプル制限、代替変数等の結果をまとめる。

%--------------------------------------------------

\chapter{考察}
\label{chap:discussion}

\section{メカニズムの解釈}
\label{sec:mechanism}
結果が生じた行動・制度的メカニズムを議論する。

\section{政策的含意}
\label{sec:policy_implication}
制度設計・運用への含意を整理する。

\section{限界}
\label{sec:limitations}
データ・識別・外的妥当性などの限界を書く。

%--------------------------------------------------

\chapter{結論}
\label{chap:conclusion}
本研究の要点を簡潔にまとめる。

%==================================================
% 付録
%==================================================
\appendix

\chapter{表・図の補足}
\label{app:figtab}

\section{イベントスタディ係数表、プレトレンドF-test}
\label{app:eventstudy_tables}
本文では省略したイベントスタディの係数表やプレトレンド検定結果を掲載する。

\chapter{ロバストネスチェック}
\label{app:robustness}
追加のロバストネス結果をまとめる。

\chapter{使用データ・コード一覧}
\label{app:data_code_list}
使用データの一覧、前処理・推定コードの概要(または管理方法)を書く。

%==================================================
% 参考文献
%==================================================

\bibliographystyle{plain} % 和文向けに jplain などに変更可
\bibliography{thesis}

\end{document}