% !TEX program = xelatex
% ===========
% XMU/SINTEF Beamer template (minimal 20–25 min)
% ===========
% Assumptions:
% - beamerthemesintef.sty and sintefcolor.sty are in the same directory as this .tex
% - Images are placed under ./XMU-assets/  (see paths below)
%   Required (at least):
%     XMU-assets/XMU-logo-negative.png
%     XMU-assets/XMU-negative.png
%     XMU-assets/XMU.png

\documentclass{beamer}
\usetheme{sintef}

% --- Japanese / fonts (XeLaTeX) ---
\usepackage{fontspec}
\usepackage{xeCJK}
% If you get font errors, comment out the next two lines and set fonts available on your machine.
% \setCJKmainfont{IPAexGothic}
% \setCJKsansfont{IPAexGothic}
\setCJKmainfont{Hiragino Kaku Gothic ProN}
\setCJKsansfont{Hiragino Kaku Gothic ProN}

\usefonttheme[onlymath]{serif}

% --- Math / tables / figures ---
\usepackage{amsmath, amssymb, bm}
\usepackage{booktabs, threeparttable}
\usepackage{graphicx}
\usepackage{tikz}
\usetikzlibrary{positioning}
\usepackage{pgfplots}
\pgfplotsset{compat=1.18}
\usepackage{multirow}
\usepackage{array}
\usepackage{arydshln} % 点線の罫線(\cdashline)用
\usepackage[table]{xcolor}


% --- Title background (theme provides \titlebackground) ---
\titlebackground*{XMU-assets/XMU-logo-negative.png}

% --- Meta (edit as needed) ---
\title{紹介状を持たない患者に対する\\選定療養費徴収義務化が外来患者の受診行動に与えた影響}
\author{五十嵐 康佑}
\date{2026年1月20日}

\begin{document}

% 1. Title
\maketitle

% 1. TOC
\begin{frame}{目次}
  \tableofcontents
\end{frame}

% 2. Summary
\section{サマリー}
\begin{frame}{サマリー}
\small
\textbf{研究目的とデータ・手法}
\begin{itemize}
  \item 目的:2022年10月導入の「紹介状なし大病院外来に対する選定療養費・徴収義務化」が、大病院への患者集中をどの程度是正したかを検証
  \item データ:協会けんぽレセプトデータ(2015〜2023年度)を用いて病院×月単位のパネルデータを作成
  \item 手法:DID+イベントスタディ
\end{itemize}
\textbf{分析結果(DID)}
\begin{itemize}
  \item 初診外来件数:選定療養費徴収義務化により\textbf{約12.5\%有意に減少}
  \item 再診外来件数:選定療養費徴収義務化により\textbf{約2.8\%有意に減少}
  \item 総外来件数:選定療養費徴収義務化により\textbf{約4.3\%有意に減少}
\end{itemize}
\end{frame}

\begin{frame}{サマリー}
\small
\textbf{分析結果(イベントスタディ・異質性分析)}
\begin{itemize}
  \item イベントスタディ:初診・総外来で制度改正直後に急減し、その後は横ばい傾向
  \item 異質性分析:総外来件数が多い病院ほど政策効果が小さい
\end{itemize}
\textbf{結論}
\begin{itemize}
  \item 本制度により医療機関の機能分化は一定進んだが、価格メカニズムのみでは限界がある
  \item かかりつけ医機能の法制化など制度的介入の必要性が示唆される
\end{itemize}
\end{frame}

% 3. Motivation
\section{モチベーション}
\begin{frame}{日本の医療制度:フリーアクセスと問題}
\begin{itemize}
  \item フリーアクセス:患者が医療機関を自由に選択可能
  \item 問題:軽症でも大病院を選好しやすく、医療資源の非効率な使用を招きうる
\end{itemize}
\vspace{1ex}
\begin{figure}[h] % 図の環境を開始
        \centering     % 中央揃え
        % width=0.8\textwidth でスライド横幅の80%の大きさに指定
        \includegraphics[width=0.6\textwidth]{freeacess_problem.jpg}
        \caption{フリーアクセスと医療資源の非効率な使用} % キャプション(不要なら削除)
\end{figure}
\end{frame}

\begin{frame}{選定療養費制度(2016年~)}
\textbf{選定療養とは}:\\
保険外併用療養のうち、将来的な保険導入を前提としないもので、患者の選択により特別の料金を支払うことで保険外の診療と保険診療を併用するもの
\begin{itemize}
  \item 2016年以前:一般病床200床以上の病院において任意で選定療養費を徴収可能
  \item 2016年:一定の大病院で選定療養費の徴収義務化スタート
  \item 2022年10月:制度対象病院の拡大と選定療養費額の引き上げ、徴収強制力の強化
\end{itemize}
\end{frame}

\begin{frame}{選定療養費制度(2016年~)}
現行制度の対象病院および選定療養費額については以下の通り。
\begin{itemize}
  \item 対象病院:特定機能病院、一般病床200床以上の地域医療支援病院、一般病床200床以上の紹介受診重点医療機関
  \item 選定療養費額
\end{itemize}
% preamble(必要なら追加)
\definecolor{myblue}{RGB}{0,92,160}        % 線・文字の青
\definecolor{myblueLight}{RGB}{230,243,255} % 薄い背景

\begin{table}[t]
\centering
\renewcommand{\arraystretch}{1.55}
\setlength{\tabcolsep}{6pt}
\arrayrulecolor{myblue} % 罫線を青に
\begin{tabular}{|>{\centering\arraybackslash}m{1.5cm}
                |>{\centering\arraybackslash}m{1.5cm}
                |>{\centering\arraybackslash}m{3.4cm}|}
\hline
\rowcolor{myblueLight}
\multirow{2}{*}{\large\color{myblue}\bfseries 初診}
  & \large\color{myblue}\bfseries 医科
  & \large\bfseries 7,000円以上 \\ \cdashline{2-3}
\rowcolor{myblueLight}
  & \large\color{myblue}\bfseries 歯科
  & \large\bfseries 5,000円以上 \\ \hline

\rowcolor{myblueLight}
\multirow{2}{*}{\large\color{myblue}\bfseries 再診}
  & \large\color{myblue}\bfseries 医科
  & \large\bfseries 3,000円以上 \\ \cdashline{2-3}
\rowcolor{myblueLight}
  & \large\color{myblue}\bfseries 歯科
  & \large\bfseries 1,900円以上 \\ \hline
\end{tabular}
\arrayrulecolor{black} % 以降の表に影響させない
\end{table}
\end{frame}

{
\setbeamertemplate{footline}{}
\begin{frame}{選定療養費制度(2016年~)}
現行制度における保険給付範囲からの控除については以下の通り。
\begin{figure}[h] % 図の環境を開始
        \centering     % 中央揃え
        % width=0.8\textwidth でスライド横幅の80%の大きさに指定
        \includegraphics[width=0.7\textwidth]{deduction.png}
        \par\vspace{1ex} % 少し間隔を空ける
        \scriptsize{出所:厚生労働省 令和4年度診療報酬改定の概要 外来I (\url{https://www.mhlw.go.jp/content/12400000/000920428.pdf})}
\end{figure}
→選定療養費を徴収しないと保険給付額の減少により病院側が損。
\end{frame}
}

\begin{frame}{制度施行状況}
2016年に徴収義務化がスタートしたものの、選定療養費の徴収が本格的にスタートしたのは2022年10月改定から
\vspace{1ex}
\begin{figure}[h] % 図の環境を開始
        \centering     % 中央揃え
        % width=0.8\textwidth でスライド横幅の80%の大きさに指定
        \includegraphics[width=0.6\textwidth]{figures/figure1_manipulation_check_total.png}
        \caption{選定療養費徴収件数推移} % キャプション(不要なら削除)
\end{figure}
\end{frame}

\begin{frame}{情報の非対称性と政策効果}
\begin{itemize}
  \item 医療サービス市場における情報の非対称性から、患者は自身の病状に関する判断を正確にできない
  \item 大病院志向が強ければ、7,000円以上の選定療養費があったとしても大病院を選び続ける可能性がある
  \item よって、選定療養費制度の目的である医療機関の機能分化を達成しているかは自明ではない
\end{itemize}
\end{frame}

\begin{frame}{リサーチクエスチョン}
\begin{itemize}
  \item RQ1(政策の有効性):選定療養費の徴収義務化は対象病院において
  \begin{itemize}
    \item 初診外来件数・再診外来件数を有意に減少させたか?
    \item 総外来件数にどのような影響を与えたか?
  \end{itemize}
  \item RQ2(効果の異質性):政策効果は病院の属性によってどのように異なるか?
\end{itemize}
\end{frame}

% 4. Literature
\section{先行研究}
\begin{frame}{先行研究}
    \small % 文字サイズを少し小さくして収まりを良くする
    \begin{itemize}
        \item \textbf{1. 医療サービス需要の価格弾力性}
            \begin{itemize}
                \item \textbf{海外:} RAND HIE, Oregon HIE 等
                \item \textbf{日本:} 湯田 (2023) 等
            \end{itemize}
        \vspace{0.5em} % 行間調整
        
        \item \textbf{2. 医療機関選択}
            \begin{itemize}
                \item \textbf{Acton (1973):} 保険により金銭価格が低くなれば、時間的コストが需要決定の主要因となる。
                \item \textbf{Tay (2003):} 患者は質を強く選好し、多少遠くても質を重視して医療機関を選択する傾向がある。
            \end{itemize}
        \vspace{0.5em}
        
        \item \textbf{3. 選定療養費制度の評価}
            \begin{itemize}
                \item \textbf{菅原 (2013) 等:} アンケート調査に基づき、定額自己負担の導入が受診行動に与える影響をシミュレーション。
                \item \textbf{Iba et al. (2025):} 茨城県のデータ(国保・後期)を用い、地域医療支援病院において紹介率が有意に上昇したことを報告。
            \end{itemize}
    \end{itemize}
\end{frame}

% 5. Data
\section{データ}

\begin{frame}{データ:協会けんぽレセプト}
\begin{itemize}
  \item 2015~2023年度のレセプトデータを利用
  \item 病院$\times$月のパネルデータを構築
  \item 分析対象:許可病床200床以上の病院に限定
  \begin{itemize}
    \item 協会けんぽデータの制約上、一般病床200床以上の識別が困難
  \end{itemize}
\end{itemize}

\vspace{1ex}
\begin{figure}
\centering
\begin{tikzpicture}
  \node[draw, rounded corners, align=left, minimum width=0.92\linewidth, minimum height=1.2cm] {
  単位:病院$\times$月\quad 期間:2015/04--2024/03(例)\\
  主要変数:初診・再診・総外来件数($\ln$)、診療科数、Treat推定フラグ};
\end{tikzpicture}
\caption{分析データの概要(模式図)}
\end{figure}
\end{frame}

\begin{frame}{アウトカム・コントロール}
\begin{itemize}
  \item アウトカム(対数変換):
  \[
    \ln(\text{初診外来件数}),\ \ln(\text{再診外来件数}),\ \ln(\text{総外来件数})
  \]
  \item コントロール:診療科数
\end{itemize}

\vspace{1ex}
\begin{table}
\centering
\small
\begin{tabular}{lcc}
\toprule
変数 & 定義 & 備考 \\
\midrule
初診外来件数 & 月次初診件数 & $\ln(\cdot)$を使用 \\
再診外来件数 & 月次再診件数 & $\ln(\cdot)$を使用 \\
総外来件数 & 初診+再診 & $\ln(\cdot)$を使用 \\
診療科数 & 診療科の数 & コントロール \\
\bottomrule
\end{tabular}
\caption{主要変数(仮表)}
\end{table}
\end{frame}

% 6. Methods
\section{手法}

\begin{frame}{実証戦略:介入群の同定}
\begin{itemize}
  \item レセプトから「制度対象病院」を直接観測できない
  \item 2022年10月以降に選定療養費(紹介状なし)徴収件数が急増することを利用
  \item 診療行為コード(選定療養費)の記録に基づき、
  \begin{itemize}
    \item 改定後に継続的に徴収している病院:介入群
    \item そうでない病院:対照群
  \end{itemize}
\end{itemize}

\vspace{1ex}
\begin{figure}
\centering
\begin{tikzpicture}
\begin{axis}[
  width=0.92\linewidth, height=0.34\linewidth,
  xlabel={年月(例)}, ylabel={徴収件数(例)},
  xtick={1,2,3,4,5,6},
  xticklabels={2022/04,2022/07,2022/10,2023/01,2023/04,2023/07},
  x tick label style={font=\scriptsize},
  y tick label style={font=\scriptsize},
  grid=major
]
\addplot+[mark=*] coordinates {(1,50) (2,55) (3,60) (4,240) (5,260) (6,255)};
\end{axis}
\end{tikzpicture}
\caption{選定療養費徴収件数の推移(模式図:2022/10で急増)}
\end{figure}
\end{frame}

\begin{frame}{TWFE DID}
\[
\ln Y_{it} = \gamma (\text{Treat}_i \cdot \text{Post}_{it})
+ \beta X_{it} + \alpha_i + \delta_t + \epsilon_{it}
\]
\begin{itemize}
  \item $Y_{it}$:初診・再診・総外来(対数)
  \item $X_{it}$:診療科数
  \item $\alpha_i$:病院固定効果,$\delta_t$:月固定効果
\end{itemize}
\end{frame}

\begin{frame}{イベントスタディ}
\small
\[
\ln Y_{it} =
\beta_{\text{pre}} \cdot \mathbf{1}(t < E_i - 12)\cdot \text{Treat}_i
+ \sum_{k=-12,\,k\neq -1}^{17}
\beta_k \cdot \mathbf{1}(t = E_i + k)\cdot \text{Treat}_i
+ \theta X_{it} + \alpha_i + \delta_t + \epsilon_{it}
\]

\vspace{0.5ex}
\begin{figure}
\centering
\begin{tikzpicture}
\begin{axis}[
  width=0.92\linewidth, height=0.34\linewidth,
  xlabel={イベント時間 $k$}, ylabel={係数(例)},
  grid=major, xmin=-12, xmax=17
]
\addplot+[mark=*] coordinates {
(-12,0.06) (-9,0.05) (-6,0.04) (-3,0.03) (0,-0.08) (3,-0.11) (6,-0.10) (9,-0.10) (12,-0.10) (15,-0.10) (17,-0.10)
};
\addplot+[domain=-12:17, samples=2] ({x},{0});
\end{axis}
\end{tikzpicture}
\caption{イベントスタディ(模式図)}
\end{figure}
\end{frame}

% 7. Results
\section{結果}

\begin{frame}{DID:平均処置効果(ATT)}
\begin{itemize}
  \item 初診:$\approx -12.5\%$(有意)
  \item 再診:$\approx -2.8\%$
  \item 総外来:$\approx -4.0\%$(約$-4.3\%$)
\end{itemize}

\vspace{1ex}
\begin{table}
\centering
\small
\begin{threeparttable}
\begin{tabular}{lccc}
\toprule
 & 初診($\ln$) & 再診($\ln$) & 総外来($\ln$) \\
\midrule
$\text{Treat}\cdot\text{Post}$ & $-0.125^{***}$ & $-0.028^{***}$ & $-0.043^{***}$ \\
 & $(0.010)$ & $(0.008)$ & $(0.008)$ \\
診療科数 & $0.016^{**}$ & $0.019^{***}$ & $0.018^{***}$ \\
 & $(0.007)$ & $(0.007)$ & $(0.007)$ \\
\bottomrule
\end{tabular}
\begin{tablenotes}\footnotesize
\item 注:数値は仮。標準誤差は括弧内。$^{***}p<0.01$, $^{**}p<0.05$.
\end{tablenotes}
\end{threeparttable}
\caption{TWFE DID推定(仮表)}
\end{table}
\end{frame}

\begin{frame}{イベントスタディ:改定後の動学効果}
\begin{itemize}
  \item 改定後1~3ヶ月:減少トレンド
  \item それ以降:横ばい
\end{itemize}

\vspace{1ex}
\begin{figure}
\centering
\begin{tikzpicture}
\begin{axis}[
  width=0.92\linewidth, height=0.34\linewidth,
  xlabel={イベント時間(月)}, ylabel={係数(例)},
  grid=major, xmin=-6, xmax=12
]
\addplot+[mark=*] coordinates {(-6,0.03) (-3,0.02) (-1,0.00) (0,-0.06) (1,-0.09) (2,-0.11) (3,-0.10) (6,-0.10) (9,-0.10) (12,-0.10)};
\addplot+[domain=-6:12, samples=2] ({x},{0});
\end{axis}
\end{tikzpicture}
\caption{改定後に急減し横ばい(模式図)}
\end{figure}
\end{frame}

\begin{frame}{異質性分析:診療科数$\times$総外来件数}
\begin{itemize}
  \item 診療科数と総外来件数の2軸の異質性
  \item 総外来件数が少ない病院ほど処置効果が大きいことが示唆
\end{itemize}

\vspace{1ex}
\begin{table}
\centering
\small
\begin{tabular}{lccc}
\toprule
 & 総外来:低 & 中 & 高 \\
\midrule
診療科数:少 & $-0.16$ & $-0.13$ & $-0.10$ \\
診療科数:中 & $-0.14$ & $-0.11$ & $-0.08$ \\
診療科数:多 & $-0.12$ & $-0.09$ & $-0.05$ \\
\bottomrule
\end{tabular}
\caption{異質性(初診の係数:仮)}
\end{table}
\end{frame}

% 8. Interpretation
\section{解釈}

\begin{frame}{解釈(初診)}
\begin{itemize}
  \item 制度変更直後:周知により受診控え・診療所への移動が発生
  \item 数ヶ月後:患者が新たな価格水準に適応し、効果が横ばい化した可能性
\end{itemize}
\end{frame}

\begin{frame}{解釈(再診・総外来/異質性/制度的介入)}
\begin{itemize}
  \item 再診・総外来への影響が小さい可能性
  \begin{itemize}
    \item 再診は初診より選定療養費が低い
    \item 治療継続中の医療機関変更はスイッチングコストが高い
    \item 総外来は再診比率が高く、再診の小さい変化に引っ張られる
  \end{itemize}
  \item 異質性の解釈
  \begin{itemize}
    \item 総外来が少ない(中規模)病院:代替機関が見つかりやすい可能性
    \item 総外来が多い病院:代替がないと患者が考え移動が起きにくい可能性
  \end{itemize}
  \item 価格メカニズムのみでは限界 $\Rightarrow$ かかりつけ医機能の法制化等の制度的介入が必要
\end{itemize}
\end{frame}

% 9. Limitations & Future
\section{限界・今後}

\begin{frame}{限界:協会けんぽデータ}
\begin{itemize}
  \item 高齢者の情報が含まれていない
  \item 大企業等の従業員とその家族の情報が含まれていない
\end{itemize}

\vspace{1ex}
\begin{figure}
\centering
\begin{tikzpicture}
  \node[draw, rounded corners, align=left, minimum width=0.92\linewidth, minimum height=1.1cm] {
  対象母集団の偏り(年齢・企業規模)に留意が必要};
\end{tikzpicture}
\caption{データ制約(模式図)}
\end{figure}
\end{frame}

\begin{frame}{限界:平行トレンド/今後の修正分析}
\begin{itemize}
  \item 平行トレンド仮定
  \begin{itemize}
    \item イベントスタディから仮定が満たされていない可能性
    \item 介入直後の負へのシフトから、受診抑制効果の可能性は否定されない
  \end{itemize}
  \item 今後(修正分析案)
  \begin{itemize}
    \item 診療科数と年月ダミーをコントロールに追加
    \item IPW-DID / PSM-DID
    \item Synthetic DID
  \end{itemize}
\end{itemize}
\end{frame}

% (Optional) Short conclusion slide (minimal, but useful for timing)
\begin{frame}{結論}
\begin{itemize}
  \item 徴収義務化は初診外来を有意に減少(約$-12.5\%$)
  \item 再診・総外来への効果は限定的(スイッチングコスト等)
  \item 価格メカニズムのみでは限界 $\Rightarrow$ 制度的介入の検討が必要
\end{itemize}
\end{frame}

\backmatter
\end{document}