% !TEX program = xelatex
% ===========
% XMU/SINTEF Beamer template (minimal 20–25 min)
% ===========
% Assumptions:
% - beamerthemesintef.sty and sintefcolor.sty are in the same directory as this .tex
% - Images are placed under ./XMU-assets/  (see paths below)
%   Required (at least):
%     XMU-assets/XMU-logo-negative.png
%     XMU-assets/XMU-negative.png
%     XMU-assets/XMU.png

\documentclass{beamer}
\usetheme{sintef}

% --- Japanese / fonts (XeLaTeX) ---
\usepackage{lmodern}
\usepackage[T1]{fontenc}
\usepackage{fontspec}
\usepackage{xeCJK}
% If you get font errors, comment out the next two lines and set fonts available on your machine.
% \setCJKmainfont{IPAexGothic}
% \setCJKsansfont{IPAexGothic}
\setCJKmainfont{Hiragino Kaku Gothic ProN}
\setCJKsansfont{Hiragino Kaku Gothic ProN}

\usefonttheme[onlymath]{serif}

% --- Math / tables / figures ---
\usepackage{amsmath, amssymb, bm}
\usepackage{booktabs, threeparttable}
\usepackage{tabularx}
\usepackage{graphicx}
\usepackage{tikz}
\usetikzlibrary{positioning}
\usepackage{pgfplots}
\pgfplotsset{compat=1.18}
\usepackage{multirow}
\usepackage{array}
\usepackage{arydshln} % 点線の罫線(\cdashline)用
\usepackage[table]{xcolor}


% --- Title background (theme provides \titlebackground) ---
\titlebackground*{XMU-assets/XMU-logo-negative.png}

% --- Meta (edit as needed) ---
\title{紹介状を持たない患者に対する\\選定療養費徴収義務化が外来患者の受診行動に与えた影響}
\author{五十嵐 康佑}
\date{2026年1月20日}

\begin{document}

% 1. Title
\maketitle

% 1. TOC
\begin{frame}{目次}
  \tableofcontents
\end{frame}

% 2. Summary
\section{サマリー}
\begin{frame}{サマリー}
\small
\textbf{研究目的とデータ・手法}
\begin{itemize}
  \item 目的:2022年10月導入の「紹介状なし大病院外来に対する選定療養費・徴収義務化」が、大病院への患者集中をどの程度是正したかを検証
  \item データ:協会けんぽレセプトデータ(2015〜2023年度)を用いて病院$\times$月単位のパネルデータを作成
  \item 手法:DID, イベントスタディ
\end{itemize}
\textbf{分析結果(DID)}
\begin{itemize}
  \item 初診外来件数:選定療養費徴収義務化により\textbf{約12.5\%有意に減少}
  \item 再診外来件数:選定療養費徴収義務化により\textbf{約2.8\%有意に減少}
  \item 総外来件数:選定療養費徴収義務化により\textbf{約4.3\%有意に減少}
\end{itemize}
\end{frame}

\begin{frame}{サマリー}
\small
\textbf{分析結果(イベントスタディ・異質性分析)}
\begin{itemize}
  \item イベントスタディ:初診・総外来で制度改正直後に急減し、その後は横ばい傾向
  \item 異質性分析:総外来件数が多い病院ほど政策効果が小さい
\end{itemize}
\textbf{結論}
\begin{itemize}
  \item 本制度により医療機関の機能分化は一定進んだが、価格メカニズムのみでは限界がある
  \item かかりつけ医機能の法制化など制度的介入の必要性が示唆される
\end{itemize}
\end{frame}

% 3. Motivation
\section{モチベーション}
\begin{frame}{日本の医療制度:フリーアクセスと問題}
\begin{itemize}
  \item フリーアクセス:患者が医療機関を自由に選択可能
  \item 問題:軽症でも大病院を選好しやすく、医療資源の非効率な使用を招きうる
\end{itemize}
\vspace{1ex}
\begin{figure}[h] % 図の環境を開始
        \centering     % 中央揃え
        % width=0.8\textwidth でスライド横幅の80%の大きさに指定
        \includegraphics[width=0.6\textwidth]{freeacess_problem.jpg}
        \caption{フリーアクセスと医療資源の非効率な使用} % キャプション(不要なら削除)
\end{figure}
\end{frame}

\begin{frame}{選定療養費制度(2016年~)}
\textbf{選定療養とは}:\\
保険外併用療養のうち、将来的な保険導入を前提としないもので、患者の選択により特別の料金を支払うことで保険外の診療と保険診療を併用するもの
\begin{itemize}
  \item 2016年以前:一般病床200床以上の病院において任意で選定療養費を徴収可能
  \item 2016年:一定の大病院で選定療養費の徴収義務化スタート
  \item 2022年10月:制度対象病院の拡大と選定療養費額の引き上げ、徴収強制力の強化
\end{itemize}
\end{frame}

\begin{frame}{選定療養費制度(2016年~)}
現行制度の対象病院および選定療養費額については以下の通り。
\begin{itemize}
  \item 対象病院:特定機能病院、一般病床200床以上の地域医療支援病院、一般病床200床以上の紹介受診重点医療機関
  \item 選定療養費額
\end{itemize}
% preamble(必要なら追加)
\definecolor{myblue}{RGB}{0,92,160}        % 線・文字の青
\definecolor{myblueLight}{RGB}{230,243,255} % 薄い背景

\begin{table}[t]
\centering
\renewcommand{\arraystretch}{1.55}
\setlength{\tabcolsep}{6pt}
\arrayrulecolor{myblue} % 罫線を青に
\begin{tabular}{|>{\centering\arraybackslash}m{1.5cm}
                |>{\centering\arraybackslash}m{1.5cm}
                |>{\centering\arraybackslash}m{3.4cm}|}
\hline
\rowcolor{myblueLight}
\multirow{2}{*}{\large\color{myblue}\bfseries 初診}
  & \large\color{myblue}\bfseries 医科
  & \large\bfseries {\color{red} 7,000円以上} \\ \cdashline{2-3}
\rowcolor{myblueLight}
  & \large\color{myblue}\bfseries 歯科
  & \large\bfseries 5,000円以上 \\ \hline

\rowcolor{myblueLight}
\multirow{2}{*}{\large\color{myblue}\bfseries 再診}
  & \large\color{myblue}\bfseries 医科
  & \large\bfseries {\color{blue} 3,000円以上} \\ \cdashline{2-3}
\rowcolor{myblueLight}
  & \large\color{myblue}\bfseries 歯科
  & \large\bfseries 1,900円以上 \\ \hline
\end{tabular}
\arrayrulecolor{black} % 以降の表に影響させない
\end{table}
\end{frame}

{
\setbeamertemplate{footline}{}
\begin{frame}{選定療養費制度(2016年~)}
現行制度における保険給付範囲からの控除については以下の通り。
\begin{figure}[h] % 図の環境を開始
        \centering     % 中央揃え
        % width=0.8\textwidth でスライド横幅の80%の大きさに指定
        \includegraphics[width=0.7\textwidth]{deduction.png}
        \par\vspace{1ex} % 少し間隔を空ける
        \scriptsize{出所:厚生労働省 令和4年度診療報酬改定の概要 外来I (\url{https://www.mhlw.go.jp/content/12400000/000920428.pdf})}
\end{figure}
→選定療養費を徴収しないと保険給付額の減少により病院側が損
\end{frame}
}

\begin{frame}{制度施行状況}
2016年に徴収義務化がスタートしたものの、選定療養費の徴収が本格的にスタートしたのは2022年10月改定から
\vspace{1ex}
\begin{figure}[h] % 図の環境を開始
        \centering     % 中央揃え
        % width=0.8\textwidth でスライド横幅の80%の大きさに指定
        \includegraphics[width=0.6\textwidth]{figures/figure1_manipulation_check_total.png}
        \caption{選定療養費徴収件数推移} % キャプション(不要なら削除)
\end{figure}
\end{frame}

\begin{frame}{情報の非対称性と政策効果}
\begin{itemize}
  \item 医療サービス市場における情報の非対称性から、患者は自身の病状に関する判断を正確にできない
  \item 大病院志向が強ければ、7,000円以上の選定療養費があったとしても大病院を選び続ける可能性がある
  \item よって、選定療養費制度の目的である医療機関の機能分化を達成しているかは自明ではない
\end{itemize}
\end{frame}

\begin{frame}{リサーチクエスチョン}
\begin{itemize}
  \item RQ1(政策の有効性):選定療養費の徴収義務化は対象病院において
  \begin{itemize}
    \item 初診外来件数・再診外来件数を有意に減少させたか?
    \item 総外来件数にどのような影響を与えたか?
  \end{itemize}
  \item RQ2(効果の異質性):政策効果は病院の属性によってどのように異なるか?
\end{itemize}
\end{frame}

% 4. Literature
\section{先行研究}
\begin{frame}{先行研究}
    \small % 文字サイズを少し小さくして収まりを良くする
    \begin{itemize}
        \item \textbf{1. 医療サービス需要の価格弾力性}
            \begin{itemize}
                \item \textbf{海外:} RAND HIE, Oregon HIE 等
                \item \textbf{日本:} 湯田 (2023) 等
            \end{itemize}
        \vspace{0.5em} % 行間調整
        
        \item \textbf{2. 医療機関選択}
            \begin{itemize}
                \item \textbf{Acton (1973):} 保険により金銭価格が低くなれば、時間的コストが需要決定の主要因となる。
                \item \textbf{Tay (2003):} 患者は質を強く選好し、多少遠くても質を重視して医療機関を選択する傾向がある。
            \end{itemize}
        \vspace{0.5em}
        
        \item \textbf{3. 選定療養費制度の評価}
            \begin{itemize}
                \item \textbf{菅原 (2013) 等:} アンケート調査に基づき、定額自己負担の導入が受診行動に与える影響をシミュレーション。
                \item \textbf{Iba et al. (2025):} 茨城県のデータ(国保・後期)を用い、地域医療支援病院において紹介率が有意に上昇したことを報告。
            \end{itemize}
    \end{itemize}
\end{frame}

% 5. Data
\section{データ}

\begin{frame}{データ}
\begin{itemize}
  \item 2015~2023年度の協会けんぽレセプトデータを利用
  \item 病院$\times$月のパネルデータを構築
  \item 分析対象:許可病床200床以上の病院に限定
  \begin{itemize}
    \item 協会けんぽデータの制約上、一般病床200床以上の識別が困難
  \end{itemize}
\end{itemize}
\begin{itemize}
  \item アウトカム(対数変換):
  \[
    \ln(\text{初診外来件数}),\ \ln(\text{再診外来件数}),\ \ln(\text{総外来件数})
  \]
  \item コントロール:診療科数
  \item 最終的な分析サンプル
  \begin{itemize}
    \item 1,546病院
    \item 総観測数:154,698
  \end{itemize}
  \item 変数詳細:Appendix
\end{itemize}
\end{frame}

% 6. Methods
\section{手法}

\begin{frame}{実証戦略}
本研究では2022年10月の制度改定を自然実験とみなして分析を行う。
\begin{itemize}
  \item 2022年10月から選定療養費の徴収件数が飛躍的に増大している。(下図)
  \item 保険給付範囲からの控除により徴収強制力が強化
\end{itemize}
\vspace{1ex}
\begin{figure}[h] % 図の環境を開始
        \centering     % 中央揃え
        % width=0.8\textwidth でスライド横幅の80%の大きさに指定
        \includegraphics[width=0.5\textwidth]{figures/figure1_manipulation_check_total.png}
\end{figure}
\end{frame}

\begin{frame}{実証戦略}
\textbf{観測単位・期間・サンプル}
\begin{itemize}
  \item 医科・外来のレセプトに限定
  \item 病院$\times$月単位のパネルデータ(2015/4〜2024/3)を作成
  \item サンプルに含まれる病院は許可病床200床以上と推定される病院のみ
  \begin{itemize}
    \item 協会けんぽデータの制約上、一般病床200床以上の識別が困難
  \end{itemize}
\end{itemize}
\textbf{処置群・対照群の定義}
\begin{itemize}
  \item {\color{red} 処置群}:処置前期間に選定療養費を徴収している月数の割合が1割未満\footnote{一般病床200床以上であれば任意で徴収可能であるため。}、処置後はすべての月で選定療養費を徴収している病院群
  \item {\color{blue} 対照群}:データ期間を通じて一度も選定療養費の徴収実績がない病院群
\end{itemize}
\end{frame}

\begin{frame}{TWFE DID}
\[
\ln Y_{it} = \gamma (\text{Treat}_i \cdot \text{Post}_{it})
+ \beta X_{it} + \alpha_i + \delta_t + \epsilon_{it}
\]
\begin{itemize}
  \item $\text{Treat}_i \cdot \text{Post}_{it}$:DID項
  \item $Y_{it}$:アウトカム(対数)
  \begin{itemize}
    \item 初診外来件数
    \item 再診外来件数
    \item 総外来件数
  \end{itemize}
  \item $X_{it}$:
  \begin{itemize}
    \item 診療科数
  \end{itemize}
  \item $\alpha_i$:病院固定効果
  \item $\delta_t$:月固定効果
\end{itemize}
{\tiny 標準誤差については病院レベルでのクラスターロバスト標準誤差を使用する。}
\end{frame}

\begin{frame}{イベントスタディ}
\small
\[
\ln Y_{it} =
\beta_{\text{pre}} \cdot \mathbf{1}(t < E_i - 12)\cdot \text{Treat}_i
+ \sum_{k=-12,\,k\neq -1}^{17}
\beta_k \cdot \mathbf{1}(t = E_i + k)\cdot \text{Treat}_i
+ \theta X_{it} + \alpha_i + \delta_t + \epsilon_{it}
\]
\begin{itemize}
  \item $Y_{it}$, $X_{it}$, $\alpha_i$, $\delta_t$: 上ページと同じ
  \item $\mathbf{1}(t = E_i + k)\cdot \text{Treat}_i$:処置群の病院において, 制度改定から$k$ヶ月経
過した時点($k$が負の場合は改定前)であることを示すダミー変数
\end{itemize}
制度改定直前($ = -1$)の月は分析の基準点として除外
\end{frame}

% 7. Results
\section{結果}

\begin{frame}{DID:平均処置効果(ATT)}
\vspace{1ex}
\begin{table}[t]
\centering
\scriptsize
\setlength{\tabcolsep}{4pt}
\renewcommand{\arraystretch}{1.15}
\caption{DID推定結果}\label{tab:did_slide}
\begin{threeparttable}
\begin{tabular}{lccc}
\toprule
 & 初診件数(対数) & 再診件数(対数) & 総外来件数(対数) \\
\midrule
$\text{Treat}_i \times \text{Post}_{it}$ & -0.1250*** & -0.0283*** & -0.0430*** \\
診療科数 & 0.0163** & 0.0185*** & 0.0176*** \\
\midrule
Observations & 154,698 & 154,698 & 154,698 \\
Adj. $R^2$ & 0.944 & 0.982 & 0.980 \\
\bottomrule
\end{tabular}
\begin{tablenotes}[flushleft]
\tiny
\item 注:病院固定効果・月固定効果を含む。標準誤差は病院レベルでクラスタ化(表では省略)。
\end{tablenotes}
\end{threeparttable}
\end{table}

\begin{itemize}
  \item 初診:約$-12.5\%$
  \item 再診:約$-2.8\%$
  \item 総外来:約$-4.3\%$
\end{itemize}
→いずれも1\%水準で統計的に有意な結果
\end{frame}

\begin{frame}{イベントスタディ:動学的処置効果}
\vspace{2ex}
  \hspace*{2em}1. 初診外来件数
\vspace{1ex}
\begin{figure}[h] % 図の環境を開始
        \centering     % 中央揃え
        % width=0.8\textwidth でスライド横幅の80%の大きさに指定
        \includegraphics[width=0.7\textwidth]{figures/figure3-a_event_study_first.png}
\end{figure}
\begin{itemize}
  \item 改定後1~3ヶ月:減少トレンド
  \item それ以降:特定のトレンドなし
\end{itemize}
\end{frame}

\begin{frame}{イベントスタディ:動学的処置効果}
\vspace{2ex}
  \hspace*{2em}2. 再診外来件数
\vspace{1ex}
\begin{figure}[h] % 図の環境を開始
        \centering     % 中央揃え
        % width=0.8\textwidth でスライド横幅の80%の大きさに指定
        \includegraphics[width=0.7\textwidth]{figures/figure3-b_event_study_return.png}
\end{figure}
\begin{itemize}
  \item 改定後1~3ヶ月:減少トレンド
  \item それ以降:0近傍で横ばい
\end{itemize}
\end{frame}

\begin{frame}{イベントスタディ:動学的処置効果}
\vspace{2ex}
  \hspace*{2em}3. 総外来件数
\vspace{1ex}
\begin{figure}[h] % 図の環境を開始
        \centering     % 中央揃え
        % width=0.8\textwidth でスライド横幅の80%の大きさに指定
        \includegraphics[width=0.7\textwidth]{figures/figure3-c_event_study_total.png}
\end{figure}
\begin{itemize}
  \item 改定後1~3ヶ月:減少トレンド
  \item それ以降:特定のトレンドなし
\end{itemize}
\end{frame}

\begin{frame}{異質性分析:診療科数$\times$総外来件数}
\begin{itemize}
  \item 診療科数と総外来件数の2軸による異質性分析
\end{itemize}
\vspace{1ex}
\begin{figure}[h] % 図の環境を開始
        \centering     % 中央揃え
        % width=0.8\textwidth でスライド横幅の80%の大きさに指定
        \includegraphics[width=0.5\textwidth]{figures/figure4c_matrix_First.png}
\end{figure}
→総外来件数が少ない病院ほど政策効果が大きいことが示唆される。
\end{frame}

\begin{frame}{異質性分析:診療科数$\times$総外来件数}
\begin{itemize}
  \item すべてのアウトカムに対しての異質性分析の結果
\end{itemize}
\vspace{1ex}
\begin{figure}[h] % 図の環境を開始
        \centering     % 中央揃え
        % width=0.8\textwidth でスライド横幅の80%の大きさに指定
        \includegraphics[width=0.55\textwidth]{figures/figure4c_matrix_combined.png}
\end{figure}
\end{frame}

% 8. Interpretation
\section{解釈}

\begin{frame}{解釈}
\begin{itemize}
  \item 初診外来件数への効果
  \begin{itemize}
    \item 平均的には約12.5\%減少したものの減少傾向は制度改定から3ヶ月まで
    \item 制度変更直後には医療機関からの周知などの効果で診療所・クリニックへの移動や受診控えが発生
    \item 4ヶ月以降は患者が新たな価格水準に適応し、効果が薄くなった可能性
  \end{itemize}
   \vspace{1em}
  \item 再診外来件数・総外来件数への効果
  \begin{itemize}
    \item 再診は平均的に約-2.8\%減少、総外来は約-4.3\%減少
    \item 再診に課される選定療養費は初診よりも低い
    \item 再診患者にとって医療機関変更はスイッチングコストが高い?
    \item 総外来件数は再診比率が高く、再診に引っ張られた結果となった
  \end{itemize}
\end{itemize}
\end{frame}

\begin{frame}{解釈}
\begin{itemize}
  \item 異質性分析の結果から
  \begin{itemize}
    \item 総外来件数が少ない(中規模)病院:地域内で代替となる医療機関が見つかりやすい
    \item 総外来件数が多い病院:地域内で代替となる医療機関が見つかりにくく患者の強い選好があり、政策効果が薄いと考えられる
  \end{itemize}
\end{itemize}
\end{frame}

% 9. Conclusion
\section{結論}

\begin{frame}{結論}
\begin{itemize}
  \item 分析結果
  \begin{itemize}
    \item 徴収義務化は初診外来を有意に減少(約$-12.5\%$)
    \item 再診・総外来への効果は限定的(スイッチングコスト等)
    \item 価格メカニズムのみでは限界 $\Rightarrow$ 制度的介入の検討が必要
  \end{itemize}
  \vspace{2em}
  \item 政策的含意
  \begin{itemize}
    \item 分析結果から価格メカニズムによる誘導には限界がある
    \item かかりつけ医機能の法制化も含めた制度的介入が必要ではないか
  \end{itemize}
\end{itemize}
\end{frame}

% 10. Limitations & Future
\section{限界と今後の課題}

\begin{frame}{限界:協会けんぽデータ}
\begin{itemize}
  \item 加入者の偏り:以下の情報が含まれていない
  \begin{itemize}
    \item 65歳以上の高齢者
    \item 大企業等の従業員とその家族
  \end{itemize}
  →対象母集団の偏りに留意が必要
  \vspace{1em}
  \item 所得や傷病等の患者の属性を分析に用いていない\\
  →選定療養費導入によって行動を変えた患者の属性は不明
\end{itemize}
\vspace{2em}
$\therefore$ NDBを用いたさらなる検証が待たれる
\end{frame}

\begin{frame}{限界:平行トレンド/修正分析案}
\begin{itemize}
  \item 平行トレンド仮定が厳密には成り立っていない
  \begin{itemize}
    \item イベントスタディから処置前期間のトレンドに変動がある
    \item 処置前の多くの期間で係数が正であるのに対し、処置直後は負にシフトしていることから受診抑制効果があったことは否定されない
    \item 係数が過小に推定されているおそれがある
  \end{itemize}
  \vspace{2em}
  \item 今後の課題(修正分析案):Synthetic DIDなど
\end{itemize}
\end{frame}

\appendix
\section{付録}

% ---------- Appendix (1/2) ----------
\begin{frame}{付録:データセット変数一覧(1/2)}
\scriptsize
\setlength{\tabcolsep}{4pt}
\renewcommand{\arraystretch}{1.2} % 少し行間を広げて読みやすくしました

% 列構成を変更: 3列目を削除し、1列目を広げました
\begin{tabularx}{\textwidth}{@{}>{\ttfamily}p{5.0cm} X @{}}
\toprule
\multicolumn{1}{l}{\textbf{変数名}} & \textbf{内容} \\ % ヘッダーを太字に
\midrule
hospital\_id & 病院ID \\
ym\_tm & 年月 \\
year & 年 \\
prefecture & 都道府県コード \\
niji\_area & 二次医療圏コード \\
postal\_5 & 郵便番号(5桁) \\
postal\_2 & 郵便番号上2桁(固定効果用) \\
\midrule
shoshin\_outpatients & 初診外来件数 \\
saishin\_outpatients & 再診外来件数 \\
total\_outpatients & 総外来件数 \\
no\_referral\_outpatients & 紹介状なし初診外来件数 \\
log\_shoshin\_outpatients & shoshin\_outpatientsの対数値 \\
log\_saishin\_outpatients & saishin\_outpatientsの対数値 \\
log\_total\_outpatients & total\_outpatientsの対数値 \\
\bottomrule
\end{tabularx}
\end{frame}

% ---------- Appendix (2/2) ----------
\begin{frame}{付録:データセット変数一覧(2/2)}
\scriptsize
\setlength{\tabcolsep}{4pt}
\renewcommand{\arraystretch}{1.2}

\begin{tabularx}{\textwidth}{@{}>{\ttfamily}p{5.0cm} X @{}}
\toprule
\multicolumn{1}{l}{\textbf{変数名}} & \textbf{内容} \\
\midrule
post & 処置後ダミー:2022年10月以降=1 \\
treated & 処置群ダミー:2022年10月から継続的に徴収=1 \\
event\_time & イベントスタディ用時間変数:2022年10月を0 \\
department\_count & 診療科数 \\
\midrule
pre\_rate & Pre期間の選定療養費徴収月数割合 \\
post\_rate & Post期間の選定療養費徴収月数割合 \\
\bottomrule
\end{tabularx}
\end{frame}

\backmatter
\end{document}