\documentclass[aspectratio=169]{beamer}
% 注意: XeLaTeXを使う場合、[dvipdfmx]は決して入れてはいけません

% ---------------------------------------------------
% パッケージ設定
% ---------------------------------------------------
\usetheme{sintef}

% --- 日本語環境設定 (XeLaTeX用) ---
\usepackage{xeCJK}

% TeX Live 2025に標準搭載されているフォントを指定します
% ※もしエラーが出る場合は "Hiragino Kaku Gothic ProN" (Mac標準) に変えてください
\setCJKmainfont{Hiragino Kaku Gothic ProN}
\setCJKsansfont{Hiragino Kaku Gothic ProN}
\setCJKmonofont{Hiragino Kaku Gothic ProN}

% --- 数式・フォント設定 ---
\usepackage{amsfonts,amsmath,oldgerm}
\usepackage{booktabs} 
\usepackage{graphicx} % [dvipdfmx]は不要
\usepackage{here}

% 数式用フォント設定
\usefonttheme[onlymath]{serif}

% ---------------------------------------------------
% タイトルページ設定
% ---------------------------------------------------
% ロゴ画像の指定
\titlebackground*{XMU-assets/XMU-logo-negative.png} 

\title{紹介状を持たない患者に対する\\選定療養費徴収義務化が\\外来患者の受診行動に与えた影響}
\subtitle{The Impact of the Mandatory Charge of Additional Fees for\\Non-Referral Patients on Outpatients' Behavior}

\author{五十嵐 康佑}
\date{令和8年1月6日}

% ---------------------------------------------------
% ドキュメント開始
% ---------------------------------------------------
\begin{document}

% タイトルスライド
\maketitle

% 目次スライド
\begin{frame}{目次}
    \tableofcontents
\end{frame}

% ===================================================
% 1. サマリー
% ===================================================
\section{サマリー}

\begin{frame}{サマリー (1/2): 研究の概要}
    \begin{itemize}
        \item \textbf{研究目的}
            \begin{itemize}
                \item 大病院への患者集中を是正するための「選定療養費徴収義務化」(2022年10月改定) の政策効果を検証。
                \item 追加負担が患者の受診行動に与える因果効果を推定。
            \end{itemize}
        \item \textbf{データと手法}
            \begin{itemize}
                \item 協会けんぽレセプトデータ (2015--2023年度)
                \item DID (差の差分析) および イベントスタディ
            \end{itemize}
        \item \textbf{主な結果}
            \begin{itemize}
                \item \textbf{初診外来:} 約12.5\%の有意な減少 $\rightarrow$ 軽症患者の行動変容、価格メカニズムが機能。
                \item \textbf{再診・総外来:} 減少は限定的 (再診 -2.8\%, 総外来 -4.3\%) $\rightarrow$ 高いスイッチングコストを示唆。
            \end{itemize}
    \end{itemize}
\end{frame}

\begin{frame}{サマリー (2/2): 動態と政策的含意}
    \begin{itemize}
        \item \textbf{時系列的な変化 (イベントスタディ)}
            \begin{itemize}
                \item 制度改正直後に急激な減少、その後は横ばい傾向。
            \end{itemize}
        \item \textbf{異質性分析}
            \begin{itemize}
                \item 診療科数が多く総外来件数が多い病院ほど、受診抑制効果が小さい。
                \item 代替医療機関が見つけにくい環境では需要の価格弾力性が低下。
            \end{itemize}
        \item \textbf{結論・含意}
            \begin{itemize}
                \item 医療機関の機能分化は一定程度進んだが、価格メカニズムのみによる誘導には限界あり。
                \item 「かかりつけ医機能の法制化」など、制度的介入の必要性が示唆される。
            \end{itemize}
    \end{itemize}
\end{frame}

% ===================================================
% 2. モチベーション
% ===================================================
\section{モチベーション}

\begin{frame}{背景: 日本の医療制度と課題}
    \begin{columns}
        \begin{column}{0.5\textwidth}
            \textbf{フリーアクセス制度}
            \begin{itemize}
                \item 患者が医療機関を自由に選択可能
                \item 日本の高い保健衛生水準を支える基盤
            \end{itemize}
        \end{column}
        \begin{column}{0.5\textwidth}
            \textbf{生じている問題}
            \begin{itemize}
                \item \textbf{大病院への患者集中}:
                    \begin{itemize}
                        \item 軽症患者の流入
                        \item 医療資源の非効率な使用
                        \item 勤務医の過重労働、長い待ち時間
                    \end{itemize}
            \end{itemize}
        \end{column}
    \end{columns}
    \vspace{1em}
    \textbf{対策:} 2016年より「紹介状なしの大病院受診」に対する選定療養費徴収を義務化。
\end{frame}

\begin{frame}{情報の非対称性とリサーチクエスチョン}
    \begin{itemize}
        \item \textbf{理論的課題: 情報の非対称性}
            \begin{itemize}
                \item 患者は自身の重症度を正確に判断できない。
                \item 「価格ペナルティ(選定療養費)」があっても、安心感を求めて大病院を選び続ける可能性。
                \item 制度が意図通りに機能分化を達成するかは自明ではない。
            \end{itemize}
        \item \textbf{Research Questions}
            \begin{enumerate}
                \item 選定療養費の徴収義務化は、対象病院における\textbf{初診・再診外来件数}を有意に減少させたか?
                \item 制度の効果は\textbf{病院の属性}によってどのように異なるか?
            \end{enumerate}
    \end{itemize}
\end{frame}

% ===================================================
% 3. 先行研究
% ===================================================
\section{先行研究}

\begin{frame}{先行研究と本研究の位置づけ}
    \begin{itemize}
        \item \textbf{医療需要の価格弾力性}
            \begin{itemize}
                \item RAND HIE, Oregon HIE (海外), Shigeoka(2014), Iizuka and Shigeoka(2022) (日本)
                \item 一般に医療需要は価格に感応的だが、救急医療などでは非弾力的。
            \end{itemize}
        \item \textbf{病院選択行動}
            \begin{itemize}
                \item Acton(1973), Tay(2003): 患者は質のためならコスト(時間・金銭)を支払う傾向。
            \end{itemize}
        \item \textbf{選定療養費に関する研究}
            \begin{itemize}
                \item 菅原(2013): コンジョイント分析によるシミュレーション。
                \item Iba et al.(2025): 茨城県のデータで紹介率上昇を確認。
            \end{itemize}
        \item \textbf{本研究の貢献}: 全国規模のレセプトデータを用い、2022年改定を自然実験として因果効果を推定。
    \end{itemize}
\end{frame}

% ===================================================
% 4. データ
% ===================================================
\section{データ}

\begin{frame}{データセットと変数}
    \begin{itemize}
        \item \textbf{データソース}: 協会けんぽ レセプトデータ
        \item \textbf{期間}: 2015年度 〜 2023年度
        \item \textbf{分析単位}: 病院 $\times$ 月 のパネルデータ
        \item \textbf{サンプル制限}: 許可病床200床以上の病院
            \begin{itemize}
                \item \footnotesize{※協会けんぽデータの制約上、一般病床200床以上を厳密に識別できないため許可病床で代理}
            \end{itemize}
    \end{itemize}
    \vspace{1em}
    \begin{table}[]
        \centering
        \small
        \begin{tabular}{ll}
            \toprule
            \textbf{種類} & \textbf{変数名} \\
            \midrule
            アウトカム & $\ln(\text{初診外来件数})$, $\ln(\text{再診外来件数})$, $\ln(\text{総外来件数})$ \\
            コントロール & 診療科数 \\
            \bottomrule
        \end{tabular}
    \end{table}
\end{frame}

% ===================================================
% 5. 手法
% ===================================================
\section{手法}

\begin{frame}{実証戦略: 処置群の特定}
    \begin{itemize}
        \item \textbf{課題}: レセプトデータ上では、各病院が制度の義務化対象か直接は不明。
        \item \textbf{識別方法}:
            \begin{itemize}
                \item 「紹介状なし初診」に対する選定療養費算定コードの記録を利用。
                \item 2022年10月の制度改定後、\textbf{継続的に徴収を行っている病院}を介入群(Treatment)と定義。
                \item それ以外を対照群(Control)とする。
            \end{itemize}
    \end{itemize}
    % ここに「選定療養費徴収件数が爆発的に増加している図」を入れると効果的です
    \begin{figure}[h]
        \centering
        % \includegraphics[height=3cm]{figures/descriptive_trend.png} 
        \framebox[0.7\textwidth][c]{\rule{0pt}{2.5cm}(図:2022年10月以降の選定療養費算定件数の急増トレンド)}
    \end{figure}
\end{frame}

\begin{frame}{推定モデル}
    \textbf{1. Two-Way Fixed Effects (TWFE) DID}
    \begin{equation*}
        \ln Y_{it} = \gamma (\text{Treat}_i \cdot \text{Post}_{it}) + \beta X_{it} + \alpha_i + \delta_t + \epsilon_{it}
    \end{equation*}
    \begin{itemize}
        \item $\text{Treat}_i$: 処置群ダミー, $\text{Post}_{it}$: 2022年10月以降ダミー
        \item $\alpha_i$: 病院固定効果, $\delta_t$: 月次固定効果
    \end{itemize}

    \vspace{1em}
    \textbf{2. Event Study Model}
    \begin{equation*}
        \ln Y_{it} = \sum_{k=-12, k \neq -1}^{17} \beta_k \cdot \mathbf{1}(t = E_i + k) \cdot \text{Treat}_i + \theta X_{it} + \alpha_i + \delta_t + \epsilon_{it}
    \end{equation*}
    \begin{itemize}
        \item 平行トレンド仮定の検証および動的な効果の推定。
    \end{itemize}
\end{frame}

% ===================================================
% 6. 結果
% ===================================================
\section{結果}

\begin{frame}{分析結果: DID推定 (平均処置効果)}
    \begin{table}[h]
        \centering
        \caption{選定療養費徴収義務化の因果効果 (ATT)}
        \begin{tabular}{lccc}
            \toprule
             & (1) 初診外来 & (2) 再診外来 & (3) 総外来 \\
            \midrule
            \textbf{ATT} & \textbf{-0.125***} & \textbf{-0.028**} & \textbf{-0.043***} \\
             & (0.012) & (0.011) & (0.010) \\
            \midrule
            Obs & 15,469 & 15,469 & 15,469 \\
            Hospital FE & Yes & Yes & Yes \\
            Month FE & Yes & Yes & Yes \\
            \bottomrule
            \multicolumn{4}{r}{\footnotesize *** p$<$0.01, ** p$<$0.05, * p$<$0.1 (数値は例)}
        \end{tabular}
    \end{table}
    \begin{itemize}
        \item 初診外来は約12.5\%の減少。再診・総外来への影響は限定的。
    \end{itemize}
\end{frame}

\begin{frame}{分析結果: イベントスタディ (初診外来)}
    % 初診外来のイベントスタディ図
    \begin{figure}[h]
        \centering
        % \includegraphics[width=0.7\textwidth]{figures/event_study_shoshin.png}
        \framebox[0.8\textwidth][c]{\rule{0pt}{5cm}(図:初診外来のイベントスタディ結果)}
        \caption{初診外来件数への動的な効果}
    \end{figure}
    \begin{itemize}
        \item 改定直後($t=0, 1$)に急減し、その後は低水準で横ばい。
    \end{itemize}
\end{frame}

\begin{frame}{分析結果: 異質性分析}
    \begin{itemize}
        \item \textbf{分析軸}: 診療科数 $\times$ 総外来件数(病院規模のプロキシ)
        \item \textbf{結果の概要}:
            \begin{itemize}
                \item 総外来件数が\textbf{少ない}病院ほど、処置効果(減少幅)が大きい。
                \item 総外来件数が\textbf{多い}(大規模な)病院では、効果が小さい。
            \end{itemize}
    \end{itemize}
    \vspace{0.5em}
    \begin{figure}[h]
        \centering
        % \includegraphics[width=0.6\textwidth]{figures/heterogeneity.png}
        \framebox[0.6\textwidth][c]{\rule{0pt}{3cm}(図:異質性分析の結果グラフ)}
    \end{figure}
\end{frame}

% ===================================================
% 7. 解釈と考察
% ===================================================
\section{解釈と考察}

\begin{frame}{結果の解釈}
    \begin{itemize}
        \item \textbf{初診への強い効果 (-12.5\%)}
            \begin{itemize}
                \item 制度周知により、軽症患者が診療所・クリニックへ移動または受診控え。
                \item 直後の減少+横ばい $\rightarrow$ 新たな価格水準への迅速な適応。
            \end{itemize}
        \item \textbf{再診・総外来への限定的な効果}
            \begin{itemize}
                \item 再診は選定療養費の設定額が初診より低い。
                \item 治療継続中の患者にとって、病院変更の\textbf{スイッチングコスト}が高い。
                \item 総外来件数は再診の割合が大きいため、再診の結果に引張られた。
            \end{itemize}
    \end{itemize}
\end{frame}

\begin{frame}{異質性の要因と政策的含意}
    \begin{itemize}
        \item \textbf{異質性のメカニズム}
            \begin{itemize}
                \item \textbf{中規模病院 (効果大)}: 地域内に代替となる診療所が見つかりやすい。
                \item \textbf{大規模病院 (効果小)}: 「ここしか治療できない」と患者が考える、あるいは地域独占的であり、代替機関がないため需要が非弾力的。
            \end{itemize}
        \item \textbf{政策的含意}
            \begin{itemize}
                \item 価格メカニズムによる誘導には限界がある(特に大規模病院や再診)。
                \item 金銭的インセンティブだけでなく、\textbf{「かかりつけ医機能の法制化」}など、患者フローを構造的に変える制度的介入が必要。
            \end{itemize}
    \end{itemize}
\end{frame}

% ===================================================
% 8. 限界・今後
% ===================================================
\section{限界・今後}

\begin{frame}{本研究の限界と今後の課題}
    \begin{itemize}
        \item \textbf{データの限界 (協会けんぽ)}
            \begin{itemize}
                \item 高齢者、大企業従業員の情報が含まれていない(一般化の課題)。
            \end{itemize}
        \item \textbf{平行トレンド仮定}
            \begin{itemize}
                \item イベントスタディにおいて、プレトレンドで係数が正の傾向 $\rightarrow$ 処置後の負へのシフトは確認できるが、厳密な仮定の充足には課題。
            \end{itemize}
        \item \textbf{今後の修正分析案}
            \begin{itemize}
                \item モデルの改善: 診療科数 $\times$ 年月ダミーのコントロール。
                \item ロバストネスチェック: IPW-DID, PSM-DID, Synthetic DID の適用による平行トレンド問題への対処。
            \end{itemize}
    \end{itemize}
\end{frame}

% ===================================================
% 終了
% ===================================================
\backmatter

\end{document}