\documentclass[dvipdfmx,a4paper,11pt]{jsreport}

% ---- Packages ----
\usepackage[dvipdfmx]{geometry}
\geometry{left=25mm,right=25mm,top=25mm,bottom=25mm}

\usepackage{setspace}
\usepackage{amsmath,amssymb,amsthm}
\usepackage{bm}
\usepackage[dvipdfmx]{graphicx}
\usepackage{booktabs}
\usepackage{natbib}
\usepackage{threeparttable}
\usepackage{here}
\usepackage[dvipdfmx]{hyperref}
\usepackage{url}
\usepackage{etoolbox}

\usepackage{titlesec}
\titleformat{\chapter}[hang]
  {\normalfont\huge\bfseries}
  {第\thechapter 節}
  {1em}
  {}
\titlespacing*{\chapter}{0pt}{-20pt}{20pt} 

\setstretch{1.1}

%==================================================
\begin{document}
%==================================================

% ページ番号を消す場合はこれを入れる
\thispagestyle{empty} 

% --- タイトル部分(titlepage環境は使わない) ---
\begin{center}
  \vspace*{0.1cm} % 上部の余白調整
  {\huge 論  文  要  旨}\\[1cm] % スペースを少し詰めました
  {\huge \textbf{紹介状を持たない患者に対する選定療養費徴収義務化が外来患者の受診行動に与えた影響}}\\[0.5cm]
  {\huge \textbf{(The Impact of the Mandatory Charge of Additional Fees for Non-Referral Patients on Outpatients' Behavior)}}
\end{center}

\vspace{0.5cm} % タイトルと氏名の間のスペース

\begin{flushright}
  {\Large 経済学研究科}\\[0.2cm]
  {\Large 経済専攻経済学コース}\\[0.2cm]
  {\Large 学籍番号:29-246004}\\[0.2cm]
  {\Large 氏名:五十嵐 康佑}\\[0.2cm]
\end{flushright}

\vspace{1cm} % 本文との間のスペース

% --- 本文 ---
\noindent % 段落の字下げをしない(必要に応じて)
要旨本文:\\
\textbf{1. 研究の背景} 日本の医療制度は国民皆保険とフリーアクセスにより高い保健衛生水準を達成してきた一方、低い自己負担と情報の非対称性の下で、
軽症でも高度医療資源を備えた大病院を選好する「大病院志向」が常態化している。その結果、医療資源の配分は非効率化し、重症患者へのリソース
投入が阻害されるだけでなく、勤務医の過重労働や待ち時間の増大など社会的費用も生じる。こうした課題への対応として、政府は紹介状なしで特定の
大病院を受診する外来患者に定額の特別料金(全額自己負担)を課す「選定療養費制度」を導入し、2016年に特定機能病院等で徴収を義務化した。
しかし最低5,000円という当初水準ではブランド選好を抑えるには弱く、効果は限定的と指摘されてきた。これを踏まえ2022年10月に大規模な厳格化
が行われ、①義務化対象の拡大(紹介受診重点医療機関の追加)、②最低徴収額の引上げ(初診5,000円→7,000円)、③未徴収時の保険給付控除に
よる履行強化、という三点で価格・制度両面から介入が強化された。\\
\textbf{2. 研究目的と意義} 本研究は、2022年10月改定を自然実験として、強化された選定療養費制度が患者の受診行動に与えた因果効果を定量的に評価
することを目的とする。先行研究はアンケートに基づくシミュレーションや地域・病院を限定した記述分析が中心で、全国規模での因果推論に基づく
検証は十分でなかった。そこで協会けんぽのレセプトデータを用い、日本全体における制度の有効性、効果の持続性、病院属性による異質性を包括的に
明らかにする。特に、5,000円から7,000円への価格シグナル強化が患者フローをどの程度変えるかを実証することは、今後の医療政策設計に重要な
示唆を与える。\\
\textbf{3. データと分析手法} 2015〜2023年度の月次レセプトから病院パネルを構築し、2022年10月改定で新たに義務化対象または増額対象となった
特定機能病院・地域医療支援病院・紹介受診重点医療機関を処置群、それ以外の一般病院を対照群とした。識別には介入時点を2022年10月とする
差の差分析(DID)を用い、アウトカムは初診・再診・総外来件数の対数。病院固定効果と月固定効果で時間不変の病院特性と全国的季節変動を制御し、
純粋な処置効果を推定した。加えてイベントスタディにより平行トレンド仮定の妥当性と動学的効果を検証した。\\
\textbf{4. 分析結果} 第一に、DIDでは徴収義務化により処置群の初診外来件数が平均約12.3\%有意に減少し、追加負担への反応として非紹介患者が受診先を
変えたことが示唆された。第二に、再診外来への影響は平均約1.7\%の減少に留まり、既存の通院関係を維持する便益や転院に伴う診療情報の再提供・
関係構築といったスイッチングコストが金銭負担を上回る可能性が示された。第三に、イベントスタディでは改定直後に断絶的な初診減少が見られたが、
その後の減少幅は拡大せず横ばいで推移し、周知効果の一巡後に患者が新価格体系へ適応したことがうかがえる。第四に、異質性として、診療科数が
少なく比較的空いている中規模病院では初診が約21.4\%減少した一方、診療科数が多く混雑度の高い大規模病院では約3.1\%にとどまり、代替先の
乏しさや高いブランド価値の下で価格弾力性が低下することが示唆された。\\
\textbf{5. 結論と政策的含意} 以上より、制度厳格化は軽症患者の初診行動を変え、大病院集中の緩和に一定の効果を持った。7,000円への増額は新規受診に
対して抑制策として機能した一方、再診への効果が小さいこと、最も混雑緩和が求められる大規模・高混雑病院で抑制が弱いことは、価格メカニズム
単独の限界を示す。情報の非対称性の下で患者が「質」を強く重視する状況では、金銭負担の増加だけで最適配分を実現しにくい。今後は、選定療養費
の水準議論に加え、かかりつけ医機能の制度化・信頼性向上、適切な医療情報提供など非金銭的施策を組み合わせるポリシー・ミックスが不可欠である。
一次医療を安心して選べる環境整備と価格介入を両輪として機能分化を進め、持続可能な医療提供体制の構築につなげるべきである。

\end{document}