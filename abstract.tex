\documentclass[dvipdfmx,a4paper,11pt]{jsreport}

% ---- Packages ----
\usepackage[dvipdfmx]{geometry}
\geometry{left=25mm,right=25mm,top=25mm,bottom=25mm}

\usepackage{setspace}
\usepackage{amsmath,amssymb,amsthm}
\usepackage{bm}
\usepackage[dvipdfmx]{graphicx}
\usepackage{booktabs}
\usepackage{natbib}
\usepackage{threeparttable}
\usepackage{here}
\usepackage[dvipdfmx]{hyperref}
\usepackage{url}
\usepackage{etoolbox}

\usepackage{titlesec}
\titleformat{\chapter}[hang]
  {\normalfont\huge\bfseries}
  {第\thechapter 節}
  {1em}
  {}
\titlespacing*{\chapter}{0pt}{-20pt}{20pt} 

\setstretch{1.1}

%==================================================
\begin{document}
%==================================================

% ページ番号を消す場合はこれを入れる
%\thispagestyle{empty} 

% --- タイトル部分(titlepage環境は使わない) ---
\begin{center}
  \vspace*{0.1cm} % 上部の余白調整
  {\huge 論  文  要  旨}\\[1cm] % スペースを少し詰めました
  {\huge \textbf{紹介状を持たない患者に対する選定療養費徴収義務化が外来患者の受診行動に与えた影響}}\\[0.5cm]
  {\huge \textbf{(The Impact of the Mandatory Charge of Additional Fees for Non-Referral Patients on Outpatients' Behavior)}}
\end{center}

\vspace{0.5cm} % タイトルと氏名の間のスペース

\begin{flushright}
  {\Large 経済学研究科}\\[0.2cm]
  {\Large 経済専攻経済学コース}\\[0.2cm]
  {\Large 学籍番号:29-246004}\\[0.2cm]
  {\Large 氏名:五十嵐 康佑}\\[0.2cm]
\end{flushright}

\vspace{1cm} % 本文との間のスペース

% --- 本文 ---
\noindent % 段落の字下げをしない(必要に応じて)
要旨本文:\\
\textbf{1. 研究の背景} 日本の医療制度は国民皆保険とフリーアクセスを二大支柱として発展し、高い保健衛生水準を達成してきた。その一方でフリー
アクセス下では患者の大病院志向により医療資源配分が非効率化し、重症患者へのリソース投入が阻害される・勤務医の過重労働問題・患者の待機時間増大などの
問題が生じている。こうした課題への対応として、政府は紹介状なしで特定の大病院を受診する外来患者に自己負担金とは別に定額の特別料金を課す
「選定療養費制度」を導入し、2016年に特定機能病院、一般病床500床以上の地域医療支援病院を対象として徴収を義務化した。しかし制度の仕様上病院に対する
徴収の強制力が弱く、本来の目的である医療機関の機能分化への効果は限定的であった。数回の改定を踏まえて2022年10月の改訂では大規模な厳格化が行われ、
①徴収義務化対象の拡大(紹介受診重点医療機関の追加)、②選定療養費の最低金額の引き上げ(医科初診で5,000円→7,000円)、③徴収対象の紹介状を持たない
患者に対して一定程度の点数を保健給付範囲から控除、という3点で拘束力を強めた。\\
\textbf{2. 研究目的と意義} 本研究は2022年10月の制度改定を自然実験として、選定療養費の徴収義務化が患者の受診行動に与えた影響を定量的に明らか
にすることを目的とする。先行研究はアンケートに基づくコンジョイント分析や地域を限定した分析が中心で、全国規模のデータを用いた因果推論に基づく検証は
十分でなかった。そこで全国健康保険協会(協会けんぽ)のレセプトデータを用いて当該制度の有効性、政策効果の持続性、病院属性による異質性を明らかにする。\\
\textbf{3. データと分析手法} 協会けんぽのレセプトデータのうち2015〜2023年度のものを利用し、病院×月単位のパネルデータを構築した。サンプルには
許可病床200床以上の病院のみを含めた。許可病床200床以上の病院の特定については協会けんぽのガイドラインに従った。制度改定時の2022年10月以前の徴収が
ほぼなく、制度改定時から継続的に選定療養費を徴収している病院を処置群とし、それ以外の病院を対照群とした。選定療養費の徴収パターンが不規則な病院は
データセットから除外した。分析にあたってはアウトカムを初診外来件数(対数)、再診外来件数(対数)、総外来件数(対数)とし、差の差分析(DID)およびイベント
スタディ分析を用いた。モデルには病院固定効果と月固定効果を含め、病院の規模を示す変数として診療科数もコントロールしている。\\
\textbf{4. 分析結果} 第一に、差の差分析(DID)の結果選定療養費の徴収義務化前後で処置群の初診外来件数が平均的に約12.3\%有意に減少したことが
示され、追加負担に反応した紹介状を持たない患者が一定程度受診先を大病院から変更したことが示唆された。第二に、再診外来件数および総外来件数の減少幅は
統計的に有意ではあるものの約1.7\%, 約4\%に留まり、再診についてはすでに治療プロセスにいる中で受診先変更に伴う診療情報の提供やリサーチなどの
スイッチングコストが金銭負担を上回る可能性が示唆された。第三に、イベントスタディ分析の結果初診外来件数・総外来件数については制度改定直後から大幅な
減少トレンドが3ヶ月間ほど見られたがその後の減少幅は拡大せず横ばいで推移し、制度改定後数ヶ月間で患者が新しい価格体系に適応して受診行動を元に戻した
ことが示唆された。第四に、診療科数および総外来件数の2軸による異質性分析の結果「診療科数が少ない・総外来件数が少ない」病院群においては初診外来件数が
約21.4\%減少した一方で「診療科数が多い・総外来件数が多い」病院群においては約3.1\%に留まり、地域内で代替先が乏しいような大病院においては政策効果が
乏しいことが示唆された。\\
\textbf{5. 結論と政策的含意} 以上の分析結果から、2022年10月の選定療養費制度の改定は平均的には選定療養費を徴収している大病院において初診外来件数
を減少させ、医療機関の機能分化に一定の効果を持った。その一方で再診外来件数・総外来件数への影響が小さいことおよび診療科数が多い/総外来件数が多い病院
において受診抑制効果が小さいことは、価格メカニズムによる患者の受診行動の制御に限界があることを示している。医療サービスにおける情報の非対称性があり、
患者が大病院診療による質を強く重視している状況においては選定療養費という定額負担だけではさまざまな課題を解決しにくく、今後は同制度だけではなくかかり
つけ医機能の法制化も含めた介入が求められると考えられる。

\end{document}