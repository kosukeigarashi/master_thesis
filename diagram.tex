% !TEX program = uplatex
\documentclass[dvipdfmx,11pt,a4paper]{bxjsarticle}

\usepackage[dvipdfmx]{graphicx}
\usepackage[dvipdfmx]{hyperref}
\usepackage{booktabs}
\usepackage{threeparttable}
\usepackage{amsmath,amssymb}

% geometryはbxjsarticleが先に読み込むため、再ロードしない
\geometry{margin=25mm}

\title{付録:図表リスト}
\author{}
\date{}

\begin{document}
\maketitle

% \tableofcontents
% \clearpage

\section{図表一覧}

\subsection{図一覧}
\begin{enumerate}
  \item 図1. 選定療養費徴収件数の爆発的増加(Manipulation check)
  \item 図2. 平行トレンド仮定の成立(Trends plot)
  \item 図3. イベントスタディ結果(Event study)
  \item 図4. 異質性分析の結果1:地域別(Urban vs Rural)
  \item 図5. 異質性分析の結果2:病院規模別(Large vs Small)
\end{enumerate}

\subsection{表一覧}
\begin{enumerate}
  \item 表1. 要約統計量
  \item 表2. DID推定結果
\end{enumerate}

\clearpage

\section{図}

\subsection{図1:選定療養費徴収件数の爆発的増加}
\begin{figure}[htbp]
  \centering
  \includegraphics[width=0.90\linewidth]{figure1_manipulation_check.png}
  \caption{選定療養費徴収件数の爆発的増加}
  \label{fig:app_manipulation}
\end{figure}
\clearpage

\subsection{図2:平行トレンド仮定の成立}
\begin{figure}[htbp]
  \centering
  % 大きすぎる場合は0.72などに調整
  \includegraphics[width=0.72\linewidth]{figure2_trends_combined.png}
  \caption{平行トレンド仮定の成立}
  \label{fig:app_trends}
\end{figure}
\clearpage

\subsection{図3:イベントスタディ結果}
\begin{figure}[htbp]
  \centering
  \includegraphics[width=0.72\linewidth]{figure3_event_study.png}
  \caption{イベントスタディ結果}
  \label{fig:app_eventstudy}
\end{figure}
\clearpage

\subsection{図4:異質性分析の結果1(地域別)}
\begin{figure}[htbp]
  \centering
  \includegraphics[width=0.90\linewidth]{figure4a_heterogeneity_region.png}
  \caption{異質性分析の結果1(地域別)}
  \label{fig:app_hetero_region}
\end{figure}
\clearpage

\subsection{図5:異質性分析の結果2(病院規模別)}
\begin{figure}[htbp]
  \centering
  \includegraphics[width=0.90\linewidth]{figure4b_heterogeneity_size.png}
  \caption{異質性分析の結果2(病院規模別)}
  \label{fig:app_hetero_size}
\end{figure}
\clearpage

\section{表}

% =========================
% 表1:要約統計量(table1_summary.csv から生成済み)
% =========================
\subsection{表1:要約統計量}
\begin{table}[htbp]
\centering
\caption{要約統計量}\label{tab:app_summary}
\begin{tabular}{lrrrrr}
\toprule
 & count & mean & sd & min & max \\
\midrule
shoshin\_outpatients & 154886 & 265.7033 & 206.4181 & 10 & 3796 \\
saishin\_outpatients & 154886 & 1224.928 & 1123.344 & 10 & 10450 \\
total\_outpatients & 154886 & 1490.632 & 1260.681 & 20 & 11261 \\
department\_count & 154886 & 13.1589 & 6.847548 & 1 & 102 \\
\bottomrule
\end{tabular}
\end{table}
\clearpage

% =========================
% 表2:DID推定結果(table2_did_results.csv から生成済み)
% =========================
\subsection{表2:DID推定結果}
\begin{table}[htbp]
\centering
\caption{DID推定結果}\label{tab:app_did}
\begin{threeparttable}
\begin{tabular}{lccc}
\toprule
 & \multicolumn{1}{c}{Log First} & \multicolumn{1}{c}{Log Return} & \multicolumn{1}{c}{Log Total} \\
\midrule
did\_term & -0.1230*** & -0.0174** & -0.0401*** \\
 & (0.0103) & (0.0084) & (0.0079) \\
\addlinespace
department\_count & 0.0076* & 0.0080* & 0.0074* \\
 & (0.0042) & (0.0044) & (0.0042) \\
\addlinespace
\_cons & 5.1138*** & 6.5632*** & 6.8153*** \\
 & (0.0558) & (0.0584) & (0.0556) \\
\addlinespace
\midrule
Observations & 154886 & 154886 & 154886 \\
Adj. R-squared & 0.944 & 0.983 & 0.980 \\
\bottomrule
\end{tabular}
\begin{tablenotes}[flushleft]
\footnotesize
\item Note: All models include hospital and month fixed effects.
\item Standard errors clustered at the hospital level are in parentheses.
\end{tablenotes}
\end{threeparttable}
\end{table}
\clearpage

\end{document}